\chapter{Agujeros Negros}

Las ecuaciones de Einstein son la teoría moderna por la cual describimos la gravedad 
\begin{equation}
    R_{\mu \nu}-\frac{1}{2} R g_{\mu \nu}-\Lambda g_{\mu \nu} = \frac{8 \pi G}{c^4} T_{\mu \nu},
    \label{EinsteinFieldEquations}
\end{equation}
estas describen como se curva el espacio dada una distribución de materia. Para los casos donde no estemos en presencia de materia se deben de cumplir las ecuaciones de vació
\begin{equation}
    R_{\mu \nu} = 0.
    \label{vacuumFieldEquations}
\end{equation}
De manera genérica las ecuaciones de campo son un sistema de 16 ecuaciones parciales acopladas no lineales usualmente en un espacio-tiempo 3+1,  y aunque la ecuaciones tienden a ser muy complicadas, existen métodos de aproximación, para tratar con casos mas generales, pero de momento nos cenaremos en algunas de las  soluciones analíticas que se han encontrado. Estas soluciones representan casos idealizados que solo  nos permiten entender cómo se comporta la gravedad en situaciones específicas. Algunas de las soluciones más importantes son:

\begin{itemize}
    \item La métrica de Schwarzschild, que describe el campo gravitacional de un objeto esféricamente simétrico y no rotante.
    \item La métrica de Kerr, que describe el campo gravitacional de un objeto esféricamente simétrico y rotante.
    \item La métrica de Reissner-Nordström, que describe el campo gravitacional de un objeto cargado.
    \item La métrica de Kerr-Newman, que describe el campo gravitacional de un objeto cargado y rotante.
\end{itemize}
Este capítulo se partirá de la métrica de Schwarzschild, ya que esta es la solución base para el desarrollo de conceptos más avanzados, ademas de que representa la forma que debe de representarse en cierto limite de las otras soluciones.

\begin{task}{Redaccion}{}
Corregir ete párrafo
\end{task}



\section{Solución de Schwarzschild}
\label{sec:solucionSchwarzschild}
La métrica de Schwarzschild fue la primera solución analítica a las ecuaciones de campo de Einstein. Esta solución comprende el caso mas sencillo posible, el de un objeto esféricamente simétrico y no rotante (Se puede ver la una traducción de la propuesta original en \cite{schwarzschild1999gravitationalfieldmasspoint}, en este texto haremos una derivación inspirada en \cite{eigenchris-2021}).
\begin{definition}{Condiciones de Schwarzschild}{}
    \begin{itemize}
        \item Tomamos un universo estático y esféricamente simétrico, es decir, un universo que no cambia con el tiempo y que tiene la misma forma en todas las direcciones.
        \item Se usan coordenadas esféricas $(t,r,\theta,\phi)$.
        \item Se Toma una masa puntual M en el origen de coordenadas.
        \item Se asume que no hay materia en el espacio-tiempo, es decir, que el tensor de energía-momento es cero.
        \item lejos de la masa puntual, el espacio-tiempo debe ser plano, es decir, la métrica debe ser la métrica de Minkowski.
    \end{itemize}
\end{definition}
Al saber que usaremos una simetría esférica, la métrica de Minkowski pasa de las componentes cartesianas a las componentes esféricas, es decir, la métrica de Minkowski en coordenadas esféricas es:
\begin{equation}
    \eta_{\mu \nu}=\left(\begin{array}{cccc}
            -1 & 0 & 0 & 0 \\
            0  & 1 & 0 & 0 \\
            0  & 0 & 1 & 0 \\
            0  & 0 & 0 & 1
        \end{array}\right)=\left(\begin{array}{cccc}
            -1 & 0 & 0     & 0                      \\
            0  & 1 & 0     & 0                      \\
            0  & 0 & r^{2} & 0                      \\
            0  & 0 & 0     & r^{2} \sin ^{2} \theta
        \end{array}\right)
\end{equation}
y
\begin{equation}
    \eta^{\mu \nu}=\left(\begin{array}{cccc}
            -1 & 0 & 0 & 0 \\
            0  & 1 & 0 & 0 \\
            0  & 0 & 1 & 0 \\
            0  & 0 & 0 & 1
        \end{array}\right)=\left(\begin{array}{cccc}
            -1 & 0 & 0               & 0                                \\
            0  & 1 & 0               & 0                                \\
            0  & 0 & \frac{1}{r^{2}} & 0                                \\
            0  & 0 & 0               & \frac{1}{r^{2} \sin ^{2} \theta}
        \end{array}\right),
\end{equation}
se considera que cuando la variable $r$ tiende a infinito la métrica se acerca asintóticamente a Minkowski.

\subsection{Estático en el tiempo }
\begin{definition}{Estático en el tiempo}{}
    \begin{itemize}
        \item Tiene un vector de Killing temporal.
        \item Es irrotacional.
    \end{itemize}
\end{definition}
Estático (en el tiempo):
- $\partial_t g_{\mu \nu}=0$
- $t \rightarrow -t$ no cambia $g_{\mu \nu}$

\begin{align}
     & g_{t i}=\overrightarrow{e_t} \cdot \overrightarrow{e_i} \rightarrow\left(-\overrightarrow{e_t}\right) \cdot \overrightarrow{e_i}=-g_{t i}\rightarrow g_{t i}=0 \\
     & g_{t t}=\overrightarrow{e_t} \cdot \overrightarrow{e_t} \rightarrow\left(-\overrightarrow{e_t}\right) \cdot\left(-\overrightarrow{e_t}\right)=+g_{t t}
\end{align}
por lo tanto la métrica solo depende de $r, \theta, \phi$.
\begin{equation}
    g_{\mu \nu} \left[\begin{array}{cccc}
            g_{t t} & 0            & 0                 & 0      \\
            0       & g_{r r}      & g_{r \theta}      & g_{r \phi}      \\
            0       & g_{\theta r} & g_{\theta \theta} & g_{\theta \phi} \\
            0       & g_{\phi r}   & g_{\phi \theta}   & g_{\phi \phi}
        \end{array}\right]
\end{equation}

\subsection{Simetría esférica}
Debido a que estamos usando una simetría esférica se debe cumplir:
\begin{itemize}
    \item Los componentes $\theta, \phi$ deben usar la métrica para una esfera de radio $r$.
    \item Permitir una función de escala radial $C(r)$.
    \item Usar signos negativos (convención espacial).
\end{itemize}

\begin{equation}
    \begin{array}{ll}
        g_{\theta \theta} & g_{\theta \phi} \\
        g_{\phi \theta}   & g_{\phi \phi}
    \end{array} = \left[\begin{array}{cc}
            -C(r) r^2 & 0                        \\
            0         & -C(r) r^2(\sin \theta)^2
        \end{array}\right].
\end{equation}
Ahora para los demás componentes de la métrica 
\begin{equation}
    \left[\begin{array}{cccc}
            g_{t t} & 0            & 0            & 0                        \\
            0       & g_{r r}      & g_{r \theta} & g_{r \phi}               \\
            0       & g_{\theta r} & -C(r) r^2    & 0                        \\
            0       & g_{\phi r}   & 0            & -C(r) r^2(\sin \theta)^2
        \end{array}\right]
\end{equation}
Ahora debido a la ortogonalidad de los vectores base 
\begin{equation}
    \begin{aligned}
         & \overrightarrow{e_\theta} \cdot \overrightarrow{e_r}=0=g_{\theta r} \\
         & \overrightarrow{e_\phi} \cdot \overrightarrow{e_r}=0=g_{\phi r},
    \end{aligned}
\end{equation}
es posible escribir la métrica de forma diagonal 
\begin{equation}
    \left[\begin{array}{cccc}
            g_{t t}(r, \theta, \phi) & 0                        & 0         & 0                        \\
            0                        & g_{r r}(r, \theta, \phi) & 0         & 0                        \\
            0                        & 0                        & -C(r) r^2 & 0                        \\
            0                        & 0                        & 0         & -C(r) r^2(\sin \theta)^2
        \end{array}\right].
\end{equation}
Una noción clave para facilitar los cálculos es que estamos tomando en consideración una masa puntual en un universo vacío e isotrópico, así que es de esperarse que la métrica no dependa de las coordenadas $\theta$ y $\phi$, permitiéndonos escribirla solo en términos de la distancia al origen $r$ tal que
\begin{equation}
    \left[\begin{array}{cccc}
            A(r) & 0     & 0         & 0                        \\
            0    & -B(r) & 0         & 0                        \\
            0    & 0     & -C(r) r^2 & 0                        \\
            0    & 0     & 0         & -C(r) r^2(\sin \theta)^2
        \end{array}\right].
\end{equation}
A continuación se tomará el siguiente cambio de variable:

\begin{equation}
    \tilde{r}=\sqrt{C(r)} \, r
\end{equation}
La forma de proceder va a ser  que cuando estemos lejos de la masa $M$, debemos tener las ecuaciones de campo débil a bajas velocidades y recuperar las ecuaciones de Newton.
\begin{equation*}
    g_{\mu \nu} \rightarrow \Gamma_{\mu \nu}^\sigma \rightarrow R_{\mu \nu} \rightarrow \text{Gravedad Newtoniana}
\end{equation*}


\subsection{Christoffel symbols}
los símbolos de Christoffel
\begin{equation}
    \begin{gathered}
        \Gamma_{\mu \nu}^0 \rightarrow\left[\begin{array}{cccc}
                0                        & \frac{\partial_r A}{2 A} & 0 & 0 \\
                \frac{\partial_r A}{2 A} & 0                        & 0 & 0 \\
                0                        & 0                        & 0 & 0 \\
                0                        & 0                        & 0 & 0
            \end{array}\right]
        \quad
        \Gamma_{\mu \nu}^1 \rightarrow\left[\begin{array}{cccc}
                \frac{\partial_r A}{2 B} & 0                        & 0            & 0                           \\
                0                        & \frac{\partial_r B}{2 B} & 0            & 0                           \\
                0                        & 0                        & -\frac{r}{B} & 0                           \\
                0                        & 0                        & 0            & -\frac{r(\sin \theta)^2}{B}
            \end{array}\right] \\
        \Gamma_{\mu \nu}^2 \rightarrow\left[\begin{array}{cccc}
                0 & 0           & 0           & 0                        \\
                0 & 0           & \frac{1}{r} & 0                        \\
                0 & \frac{1}{r} & 0           & 0                        \\
                0 & 0           & 0           & -\sin \theta \cos \theta
            \end{array}\right]
        \quad
        \Gamma_{\mu \nu}^3 \rightarrow\left[\begin{array}{cccc}
                0 & 0           & 0           & 0           \\
                0 & 0           & 0           & \frac{1}{r} \\
                0 & 0           & 0           & \cot \theta \\
                0 & \frac{1}{r} & \cot \theta & 0
            \end{array}\right]
    \end{gathered}
\end{equation}

\begin{equation}
    \begin{gathered}
        \Gamma_{01}^0=\Gamma_{10}^0=\frac{1}{2} \frac{1}{A}\left(\partial_r A\right), \quad \Gamma_{00}^1=\frac{1}{2} \frac{1}{B}\left(\partial_r A\right) \\
        \Gamma_{11}^1=\frac{11}{2} \frac{1}{B}\left(\partial_r B\right), \quad \Gamma_{22}^1=-\frac{r}{B}, \quad \Gamma_{33}^1=-\frac{r(\sin \theta)^2}{B} \\
        \Gamma_{12}^2=\Gamma_{21}^2=\Gamma_{13}^3=\Gamma_{31}^3=\frac{1}{r}, \quad \Gamma_{33}^2=-\sin \theta \cos \theta, \quad \Gamma_{23}^3=\Gamma_{32}^3=\cot \theta
    \end{gathered}
\end{equation}

Calculate:
- $R_{00}=0$
- $R_{11}=0$
$R_{\mu \nu}=0$
- $R_{22}=0$

\begin{equation}
    R_{00}=\frac{\partial_r \partial_r A}{2 B}
    -\frac{\partial_r A \,\partial_r B}{4 B^2}
    +\frac{\partial_r A}{B r}
    -\frac{\left(\partial_r A\right)^2}{4 A B}=0
\end{equation}

\begin{equation}
    R_{00}+R_{11}=0
\end{equation}

\begin{equation}
    \begin{aligned}
        g_{\mu \nu} & \rightarrow\left[\begin{array}{cccc}
                A(r) & 0     & 0    & 0                   \\
                0    & -B(r) & 0    & 0                   \\
                0    & 0     & -r^2 & 0                   \\
                0    & 0     & 0    & -r^2(\sin \theta)^2
            \end{array}\right] \\
        g_{\mu \nu} & \rightarrow\left[\begin{array}{cccc}
                1 & 0  & 0    & 0                   \\
                0 & -1 & 0    & 0                   \\
                0 & 0  & -r^2 & 0                   \\
                0 & 0  & 0    & -r^2(\sin \theta)^2
            \end{array}\right]
    \end{aligned}
\end{equation}

\begin{equation}
    \begin{gathered}
        B A^{\prime}+A B^{\prime}=0 \\
        \partial_r(A B)=0 \\
        \Rightarrow A B=K
    \end{gathered}
\end{equation}
\begin{equation}
    \begin{aligned}
         & A(r) \rightarrow 1 \\
         & B(r) \rightarrow 1
    \end{aligned}
\end{equation}
\begin{equation}
    (1)(1)=K
\end{equation}

\begin{equation}
    \Rightarrow B(r)=\frac{1}{A(r)} \text { para todo } r
\end{equation}

\begin{equation}
    B^{\prime}=\partial_r\left(A^{-1}\right)=-\frac{A^{\prime}}{A^2}
\end{equation}


\begin{equation}
    \begin{aligned}
         & R_{22}=-2 A B+2 A B^2-r A^{\prime} B+r A B^{\prime}                                                                  \\
         & 0=-2 A \frac{1}{A}+2 A\left(\frac{1}{A}\right)^2-r A^{\prime} \frac{1}{A}+r A\left(-\frac{A^{\prime}}{A^2}\right)
    \end{aligned}
\end{equation}
\begin{equation}
    \Rightarrow r A^{\prime}=1-A
\end{equation}

\begin{equation}
    \begin{aligned}
         & \frac{A(r)}{\partial} \equiv 1-\frac{k^{\prime}}{r}                                 \\
         & \frac{\partial}{\partial r} A=\frac{\partial}{\partial r}\left(1-\frac{k}{r}\right) \\
         & A^{\prime}=\frac{\partial}{\partial r} 1-k \frac{\partial}{\partial r} r^{-1}       \\
         & A^{\prime}=0                                                                        \\
         & A^{\prime}=\frac{k}{r^2}
    \end{aligned}
\end{equation}

\begin{equation}
    g_{\mu \nu} \rightarrow\left[\begin{array}{cccc}
            1-\frac{k}{r} & 0                                & 0    & 0                   \\
            0             & -\left(1-\frac{k}{r}\right)^{-1} & 0    & 0                   \\
            0             & 0                                & -r^2 & 0                   \\
            0             & 0                                & 0    & -r^2(\sin \theta)^2
        \end{array}\right]
\end{equation}

\begin{task}{}{}
    Terminar la derivación
\end{task}
La métrica de Schwarzschild es:
\begin{equation}
    \boxed{g_{\mu \nu}=\left(\begin{array}{cccc}
        -1+\frac{2 G m}{r c^2} & 0 & 0 & 0 \\
        0 & \left(1-\frac{2 G m}{r c^2}\right)^{-1} & 0 & 0 \\
        0 & 0 & r^2 & 0 \\
        0 & 0 & 0 & r^2 \sin ^2 \theta
        \end{array}\right)}
    \end{equation}
\subsection{Geodésicas en Schwarzschild tipo tiempo y nulas}

\subsection{Cambio de coordenadas}
Coordenadas de Eddington-Finkelstein (es decir, el componente $g_{r r}$ se elimina mediante un cambio de coordenadas y se introduce un término cruzado)

$$
\mathrm{d} s^2=\left(1-\frac{2 m}{r}\right) \mathrm{du}^2+2 \mathrm{dudr}-\mathrm{r}^2\left(\mathrm{~d} \vartheta^2+\sin ^2 \vartheta \mathrm{~d} \varphi^2\right)
$$



\section{Algoritmo de Newman-Janis y agujeros negros rotantes}

El algoritmo de Newman-Janis es una técnica en relatividad general que permite generar soluciones exactas a las ecuaciones de Einstein. En 1965, Ezra T. Newman y Alfred I. Janis \cite{newman-1965} descubrieron que, mediante una transformación de coordenadas complejas aplicada a la métrica de Schwarzschild, podían obtener la métrica de Kerr, que describe un agujero negro en rotación.

Particularmente nos centraremos en usar el desarrollo hecho por \cite{drake-2000} donde da una versión de este algoritmo donde remueve algunas de las ambigüedades presentes en el trabajo original de \cite{newman-1965}.

El algoritmo se puede resumir en 5 pasos:

\begin{enumerate}[1.]
    \item Escribe un elemento de línea estático y esféricamente simétrico en coordenadas nulas avanzadas $\{u, r, \theta, \phi\}$:

          \begin{equation}
              ds^2 = e^{2 \Phi(r)} du^2 + e^{\Phi(r) + \lambda(r)} du dr - r^2 (d\theta^2 + \sin^2 \theta d\phi^2)
          \end{equation}

    \item Expresar la forma contravariante de la métrica en términos de un tetrad nulo:

          \begin{equation}
              g^{\mu \nu} = l^\mu n^\nu + l^\nu n^\mu - m^\mu \bar{m}^\nu - m^\nu \bar{m}^\mu
              \label{MetricaTetrada}
          \end{equation}

          donde

          \begin{equation}
              l_\mu l^\mu = m_\mu m^\mu = n_\mu n^\mu = 0, \quad l_\mu n^\mu = -m_\mu \bar{m}^\mu = 1, \quad l_\mu m^\mu = n_\mu m^\mu = 0
          \end{equation}

          Donde los tetrads nulos tendrán la forma inicial:

          \begin{equation}
              \begin{aligned}
                  l^\mu & = \delta_1^\mu                                                                          \\
                  n^\mu & = e^{-\lambda(r) - \Phi(r)} \delta_0^\mu - \frac{1}{2} e^{-2 \lambda(r)} \delta_1^\mu   \\
                  m^\mu & = \frac{1}{\sqrt{2} r} \left( \delta_2^\mu + \frac{i}{\sin \theta} \delta_3^\mu \right)
              \end{aligned}
          \end{equation}

          En este punto es conveniente usar la notación de tetradas:

          \begin{equation}
              Z_a^\mu = (l^\mu, n^\mu, m^\mu, \bar{m}^\mu), \quad a = 1, 2, 3, 4
          \end{equation}

    \item El siguiente paso es hacer una transformación compleja directa que extiende las coordenadas $x^\rho$ a un nuevo conjunto de coordenadas complejas $\tilde{x}^\rho$:

          \begin{equation}
              x^\rho \rightarrow \tilde{x}^\rho = x^\rho + i y^\rho(x^\sigma)
          \end{equation}

          donde $y^\rho(x^\sigma)$ son funciones analíticas de las coordenadas reales $x^\sigma$, y simultáneamente dejamos que los vectores tetrad nulos $Z_a^\mu$ sufran una transformación:

          \begin{equation}
              Z_a^\mu(x^\rho) \rightarrow \tilde{Z}_a^\mu(\tilde{x}^\rho, \overline{\tilde{x}}^\rho)
          \end{equation}

          Finalmente requerimos como condición que la transformación recupere el tetrad y la métrica antiguos cuando $\tilde{x}^\rho = \overline{\tilde{x}}^\rho$. En resumen, el efecto de esta "transformación tilde" es crear una nueva métrica cuyos componentes (reales) son funciones de variables complejas:

          \begin{equation}
              g_{\mu \nu} \rightarrow \tilde{g}_{\mu \nu}: \tilde{x} \times \tilde{x} \mapsto \mathbb{R}
          \end{equation}

          mientras que se debe de cumplir también:

          \begin{equation}
              \left. \tilde{Z}_a^\mu(\tilde{x}^\rho, \overline{\tilde{x}}^\rho) \right|_{\overline{x} = \tilde{x}} = Z_a^\mu(x^\rho)
          \end{equation}

          La transformación tilde claramente no es única, ya que existen muchas opciones diferentes para los coeficientes de los vectores tetrad nulos que satisfacen las condiciones. En el paper original \cite{newman-1965} se elige la transformación tilde como:

          \begin{equation}
              \tilde{x}^\rho = x^\rho + i a \cos x^2 (\delta_0^\rho - \delta_1^\rho) = x^\rho + i a \cos \theta (\delta_0^\rho - \delta_1^\rho)
          \end{equation}

          En este caso, la transformación tilde se escribe para cada coordenada $\rho$ como:

          \begin{itemize}
              \item Para $\rho = 0$ (coordenada $u$):
                    \begin{equation}
                        \tilde{u} = u + i a \cos \theta
                    \end{equation}
                    porque $\delta_0^0 = 1$ y $\delta_1^0 = 0$.

              \item Para $\rho = 1$ (coordenada $r$):
                    \begin{equation}
                        \tilde{r} = r - i a \cos \theta
                    \end{equation}
                    ya que $\delta_0^1 = 0$ y $\delta_1^1 = 1$.

              \item Para $\rho = 2, 3$ (coordenadas $\theta$ y $\phi$):
                    \begin{equation}
                        \tilde{\theta} = \theta, \quad \tilde{\phi} = \phi
                    \end{equation}
                    pues $\delta_0^\rho - \delta_1^\rho = 0$ para $\rho = 2, 3$.
          \end{itemize}

          Después de aplicar la transformación, las coordenadas $\tilde{x}^\rho$ son complejas. El radio $r$ se reemplaza por $\tilde{r} = r - i a \cos \theta$, lo que permitirá que las funciones métricas (por ejemplo, $e^{2 \Phi}$) pasen a tener una dependencia combinada en $r$ y $\theta$.

          \begin{itemize}
              \item $u \rightarrow \tilde{u} = u + ia \cos \theta$ (coordenada tipo luz)
              \item $r \rightarrow \tilde{r} = r - i a \cos \theta$
              \item $\theta \rightarrow \tilde{\theta} = \theta$
              \item $\phi \rightarrow \tilde{\phi} = \phi$
          \end{itemize}

          \begin{equation}
              \begin{aligned}
                  l^\mu & \rightarrow \tilde{l}^\mu = \delta_1^\mu                                                                                                                                                                  \\
                  n^\mu & \rightarrow \tilde{n}^\mu = e^{-\lambda(\tilde{r}, \overline{\tilde{r}}) - \phi(\tilde{r}, \overline{\tilde{r}})} \delta_0^\mu - \frac{1}{2} e^{-2 \lambda(\tilde{r}, \overline{\tilde{r}})} \delta_1^\mu \\
                  m^\mu & \rightarrow \tilde{m}^\mu = \frac{1}{\sqrt{2} \overline{\tilde{r}}} \left( \delta_2^\mu + \frac{i}{\sin \tilde{\theta}} \delta_3^\mu \right)
              \end{aligned}
          \end{equation}

          En este paso se extiende la variable radial $r$ a una variable compleja. La idea es reemplazar $r$ por $\tilde{r}$ (y análogamente para $\theta$ si fuera necesario) de modo que las funciones métricas pasen a depender de dos variables: $r$ y $\theta$. De esta forma, se escriben las funciones como $\phi(\tilde{r}, \overline{\tilde{r}})$ y $\lambda(\tilde{r}, \overline{\tilde{r}})$.

          \begin{itemize}
              \item $\tilde{r} \rightarrow r$,
              \item $\overline{\tilde{r}} \rightarrow r$,
              \item $\tilde{\theta} \rightarrow \theta$,
              \item $\tilde{r} + \overline{\tilde{r}} = 2r$,
              \item $\tilde{r} - \overline{\tilde{r}} = 2ia \cos \theta$.
          \end{itemize}
          \begin{note}[Problema de la Complejidad]
              Una métrica compleja no tiene sentido físico en relatividad general, ya que el espacio-tiempo es real. Por lo tanto, se requiere eliminar las partes imaginarias introducidas por la complejificación. Esto se logra mediante la condición de realidad.

              \textbf{Mecanismo de la Condición de Realidad}

              La condición de realidad opera en dos niveles:
              \begin{enumerate}[a.]
                  \item \textbf{Sustitución de $\tilde{r}$ y $\overline{\tilde{r}}$}

                        Se reemplazan las combinaciones de $\tilde{r}$ (complejo) y su conjugado $\overline{\tilde{r}} = r - ia \cos \theta$ por expresiones reales que dependen de $r$ y $\theta$. Por ejemplo:

                        $$
                            \tilde{r} + \overline{\tilde{r}} = 2r, \quad \tilde{r} \overline{\tilde{r}} = r^2 + a^2 \cos^2 \theta.
                        $$

                        Ejemplo:
                        Si una función compleja es $f(\tilde{r}) = \frac{1}{\tilde{r}}$, la condición de realidad la convierte en:

                        $$
                            f(r, \theta) = \frac{1}{r^2 + a^2 \cos^2 \theta} \cdot (r - ia \cos \theta)
                        $$

                        pero solo se conserva la parte real relevante para la métrica.
                  \item  \textbf{Simetrización de Funciones}

                        Las funciones que dependían originalmente de $r$ en la métrica estática se redefinen como funciones simétricas en $\tilde{r}$ y $\overline{\tilde{r}}$. Por ejemplo:
                        - En la métrica de Schwarzschild, $1 - \frac{2M}{r}$ se convierte en $1 - \frac{2M}{\tilde{r}} \rightarrow 1 - \frac{2Mr}{r^2 + a^2 \cos^2 \theta}$.

              \end{enumerate}



              \textbf{Aplicación en las Tetradas}

              Las tetradas nulas $l^\mu, n^\mu, m^\mu$ también se afectan por la complejificación. Para garantizar su realidad:

              \begin{enumerate}[a.]
                  \item \textbf{Tetrada $m^\mu$}

                        La parte angular $m^\mu$ adquiere un término adicional $ia \sin \theta (\delta_0^\mu - \delta_1^\mu)$, que cancela las contribuciones imaginarias del denominador complejo $\tilde{r} = r + ia \cos \theta$.
                        Ejemplo:

                        $$
                            m^\mu = \frac{1}{\sqrt{2}(r + ia \cos \theta)} \left( ia \sin \theta (\delta_0^\mu - \delta_1^\mu) + \delta_2^\mu + \frac{i}{\sin \theta} \delta_3^\mu \right)
                        $$

                  \item \textbf{Tetradas $l^\mu$ y $n^\mu$}

                        Se mantienen reales al reemplazar $\tilde{r} \rightarrow r$ en sus componentes radiales, pero incorporando $a$ en las funciones $\phi(r, \theta)$ y $\lambda(r, \theta)$.
              \end{enumerate}
          \end{note}




    \item Se obtiene una nueva métrica al realizar una transformación de coordenadas complejas en los vectores tetrad nulos $\tilde{Z}_a^\mu$. Los vectores tetrad nulos se transforman de la manera habitual:

          \begin{equation}
              Z_a^\mu = \tilde{Z}_a^\nu \frac{\partial x^\mu}{\partial \tilde{x}^\nu}
          \end{equation}

          mediante esta transformación, de forma genérica podemos efectuar la de las tetradas $m^\mu = \tilde{m}^\nu \frac{\partial x^\mu}{\partial \tilde{x}^\nu}$ y $l^\mu = \tilde{l}^\nu \frac{\partial x^\mu}{\partial \tilde{x}^\nu}$:
          \begin{note}
            Las derivadas parciales $\pd{x^\mu}{\tilde{x}^\nu}$ las tomamos a partir de las transformaciones inversas:
            $$\begin{aligned} u & =\tilde{u}-i a \cos \tilde{\theta} \\ r & =\tilde{r}+i a \cos \tilde{\theta} \\ \theta & =\tilde{\theta} \\ \phi & =\tilde{\phi}\end{aligned}$$
            y la matriz jacobiana de la transformación es:
            \begin{equation}
               \left(\pd{x^\mu}{\tilde{x}^\nu}\right) =\left(\begin{array}{cccc}
                1 & 0 & +i a \sin \theta & 0\\
                0 & 1 & -i a \sin \theta & 0\\
                0 & 0 & 1 & 0\\
                0 & 0 & 0& 1
                \end{array}\right)
                \end{equation}
                Lo cual nos da que:
                \begin{align*}
                \pd{x^\mu}{\tilde{x}^0}&= \delta^\mu_0\\
                \pd{x^\mu}{\tilde{x}^1}&= \delta^\mu_1\\
                \pd{x^\mu}{\tilde{x}^2}&= ia (\delta^\mu_0 - \delta^\mu_1)\sin \theta + \delta^\mu_2\\
                \pd{x^\mu}{\tilde{x}^3}&=\delta^\mu_3
                \end{align*}
          \end{note}
          \begin{equation}
              \begin{aligned}
                  l^\mu & = \tilde{l}^\nu \frac{\partial x^\mu}{\partial \tilde{x}^\nu}                                                                                                                                                                           \\
                        & = \tilde{l}^0 \frac{\partial x^\mu}{\partial \tilde{x}^0} + \tilde{l}^1 \frac{\partial x^\mu}{\partial \tilde{x}^1} + \tilde{l}^2 \frac{\partial x^\mu}{\partial \tilde{x}^2} + \tilde{l}^3 \frac{\partial x^\mu}{\partial \tilde{x}^3} \\
                        & = \delta_1^0 \frac{\partial x^\mu}{\partial \tilde{x}^0} + \delta_1^1 \frac{\partial x^\mu}{\partial \tilde{x}^1} + \delta_1^2 \frac{\partial x^\mu}{\partial \tilde{x}^2} + \delta_1^3 \frac{\partial x^\mu}{\partial \tilde{x}^3}     \\
                        & = \delta_1^1 \frac{\partial x^\mu}{\partial \tilde{x}^1} = \delta_1^\mu
              \end{aligned}
          \end{equation}

          el caso $m^\mu = \tilde{m}^\nu \frac{\partial x^\mu}{\partial \tilde{x}^\nu}$:

          \begin{equation}
              \begin{aligned}
                  m^\mu & = \tilde{m}^\nu \frac{\partial x^\mu}{\partial \tilde{x}^\nu}                                                                                                                                                                                                                           \\
                        & = \tilde{m}^0 \frac{\partial x^\mu}{\partial \tilde{x}^0} + \tilde{m}^1 \frac{\partial x^\mu}{\partial \tilde{x}^1} + \tilde{m}^2 \frac{\partial x^\mu}{\partial \tilde{x}^2} + \tilde{m}^3 \frac{\partial x^\mu}{\partial \tilde{x}^3}                                                 \\
                        & = 0 * \frac{\partial x^\mu}{\partial \tilde{x}^0} + 0 * \frac{\partial x^\mu}{\partial \tilde{x}^1} + \frac{1}{\sqrt{2} \tilde{r}} \frac{\partial x^\mu}{\partial \tilde{x}^2} + \frac{1}{\sqrt{2} \tilde{r}} \frac{i}{\sin \tilde{\theta}} \frac{\partial x^\mu}{\partial \tilde{x}^3} \\
                        & = \frac{1}{\sqrt{2}(r + ia \cos \theta)} \left( ia \sin \theta (\delta_0^\mu - \delta_1^\mu) + \delta_2^\mu + \frac{i}{\sin \theta} \delta_3^\mu \right)
              \end{aligned}
          \end{equation}

          dado que la tetrada $n^\mu$ es particular para cada métrica semilla que se use, el cálculo de esta debe de hacerse de forma particular. Al efectuar la transformación del tetrad, se obtiene:

          \begin{equation}
              \begin{aligned}
                  l^\mu & = \delta_1^\mu     \\
                  m^\mu & = \frac{1}{\sqrt{2}(r + ia \cos \theta)} \left[ ia \sin \theta (\delta_0^\mu - \delta_1^\mu) + \delta_2^\mu + \frac{i}{\sin \theta} \delta_3^\mu \right]
              \end{aligned}
          \end{equation}

    \item \begin{task}
        Convertir la métrica obtenida a coordenadas de Boyer-Lindquist, que son las coordenadas más comunes para describir agujeros negros rotantes. 
    \end{task}
\end{enumerate}

\subsection{Aplicación a la métrica de Schwarzschild}
La métrica de Schwarzschild en coordenadas avanzadas de Eddington-Finkelstein es:
\begin{equation}
    ds^2=\left(1-\frac{2 m}{r}\right) d u^2+2 d u d r-r^2\left(d \theta^2+\sin ^2 \theta d \phi^2\right) .
\end{equation}

El "seed metric" propuesto es

\begin{equation}
    d s^2=e^{2 \Phi(r)} d u^2+e^{\Phi(r)+\lambda(r)} d u d r-r^2\left(d \theta^2+\sin ^2 \theta d \phi^2\right)
\end{equation}

Comparando término a término con la métrica obtenida:
Tenemos

\begin{equation}
    e^{2 \Phi(r)}=1-\frac{2 m}{r}
\end{equation}

lo que implica

\begin{equation}
    \Phi(r)=\frac{1}{2} \ln \left(1-\frac{2 M}{r}\right) .
\end{equation}

En la métrica obtenida aparece el coeficiente +2 , por lo que se identifica

\begin{equation}
    e^{\Phi(r)+\lambda(r)}=2
\end{equation}

Despejando para $\lambda(r)$,

\begin{equation}
    \begin{aligned}
        \lambda(r)= & \ln 2-\Phi(r)=\ln 2-\frac{1}{2} \ln \left(1-\frac{2 M}{r}\right)   \\
        =           & \frac{1}{2}\ln 4 -\frac{1}{2} \ln \left(1-\frac{2 M}{r}\right)     \\
        =           & \frac{1}{2} \left(\ln 4 -  \ln \left(1-\frac{2 M}{r}\right)\right) \\
        =           & \frac{1}{2} \ln \left(\frac{4}{1-\frac{2 M}{r}}\right)             \\
        =           & \frac{1}{2} \ln \left(\frac{4 r}{r-2 M}\right)
    \end{aligned}
\end{equation}
como lo comentamos en el caso genérico solo hace falta preocuparse por el caso de la tetrada $n^\mu$ que es particular para cada métrica semilla, en este caso se obtiene
\begin{equation}
    n^\mu=e^{-\lambda(r)-\Phi(r)} \delta_0^\mu-\frac{1}{2} e^{-2 \lambda(r)} \delta_1^\mu
\end{equation}
Para el caso particular de la métrica de Schwarzschild tenemos
\begin{equation}
   \begin{aligned}
    n^\mu&=e^{- \frac{1}{2} \ln \left(\frac{4 r}{r-2 m}\right) -\frac{1}{2} \ln \left(1-\frac{2 m}{r}\right)  }\delta_0^\mu-\frac{1}{2} e^{-2 \frac{1}{2} \ln \left(\frac{4 r}{r-2 m}\right)} \delta_1^\mu \\
    &=e^{\ln \left(\frac{4 r}{r-2 m}\right)^{-1/2}}e^{\ln \left(1-\frac{2 m}{r}\right)^{-1/2}} \delta_0^\mu-\frac{1}{2} e^{\ln \left(\frac{4 r}{r-2 m}\right)^{-1}} \delta_1^\mu \\
    &=\left(\frac{4 r}{r-2 m}\right)^{-1/2}\left(1-\frac{2 m}{r}\right)^{-1/2} \delta_0^\mu-\frac{1}{2} \left(\frac{4 r}{r-2 m}\right)^{-1} \delta_1^\mu \\ 
    &=\left(\frac{4 r}{r-2 m}\frac{r - 2 m}{r}\right)^{-1/2}\delta_0^\mu-\frac{1}{2} \left(\frac{r-2 m}{4r}\right)\delta_1^\mu \\
    &= \frac{1}{2}\delta_0^\mu-\frac{1}{8} \left(1 - \frac{2 m}{r}\right)\delta_1^\mu \\
   \end{aligned}
\end{equation}
procederemos a la hacer la transformación compleja de la tetrada $n^\mu \to \tilde{n}^\mu$:
\begin{equation}
    \tilde{n}^\mu=\frac{1}{2}\delta_0^\mu-\frac{1}{8} \left(1 - \frac{2 m}{\tilde{\bar{r} }}\right)\delta_1^\mu 
\end{equation}
Para sacar la nueva tetrada $n^\mu$ se hace el siguiente procedimiento:
\begin{equation}
\begin{aligned}
    n^\mu&= \tilde{n}^\mu \pd{x^\mu}{\tilde{x}^\nu} \\
&= \tilde{n}^0 \pd{x^\mu}{\tilde{x}^0} + \tilde{n}^1 \pd{x^\mu}{\tilde{x}^1} + \tilde{n}^2 \pd{x^\mu}{\tilde{x}^2} + \tilde{n}^3 \pd{x^\mu}{\tilde{x}^3} \\
&= \tilde{n}^0 \pd{x^\mu}{\tilde{x}^0} + \tilde{n}^1 \pd{x^\mu}{\tilde{x}^1} +0 \cdot \pd{x^\mu}{\tilde{x}^2} + 0 \cdot \pd{x^\mu}{\tilde{x}^3} \\
& = \frac{1}{2}\delta^{\mu}_0 -\frac{1}{8} \left(1 - \frac{2 m\tilde{r}}{\tilde{\bar{r}}\tilde{r}}\right)\delta_1^\mu \\
& = \frac{1}{2}\delta^{\mu}_0 -\frac{1}{8} \left(1 - \frac{2 m(r-ia\cos\theta)}{r^2+a^2 \cos^2 \theta }\right)\delta_1^\mu \\
\end{aligned}    
\end{equation}
de forma que todas las tetradas quedan como:
\begin{equation}
    \begin{aligned}
        l^\mu & = \delta_1^\mu     \\
        n^\mu & = \frac{1}{2}\delta^{\mu}_0 -\frac{1}{8} \left(1 - \frac{2 m(r-ia\cos\theta)}{r^2+a^2 \cos^2 \theta }\right)\delta_1^\mu  \\
        m^\mu & = \frac{1}{\sqrt{2}(r + ia \cos \theta)} \left[ ia \sin \theta (\delta_0^\mu - \delta_1^\mu) + \delta_2^\mu + \frac{i}{\sin \theta} \delta_3^\mu \right]
    \end{aligned}
\end{equation}
Y usando la transformación (\ref{MetricaTetrada}) obtenemos una forma de la métrica de Kerr
\begin{equation}
   g^{\mu\nu} =  \left(\begin{matrix}
- \frac{a^{2} \sin^{2}{\left(\theta \right)}}{a^{2} \cos^{2}{\left(\theta \right)} + r^{2}} & \frac{0.5 \left(a^{2} \sin^{2}{\left(\theta \right)} + a^{2} + r^{2}\right)}{a^{2} \cos^{2}{\left(\theta \right)} + r^{2}} & 0 & - \frac{a}{a^{2} \cos^{2}{\left(\theta \right)} + r^{2}} \\
\frac{0.5 \left(a^{2} \sin^{2}{\left(\theta \right)} + a^{2} + r^{2}\right)}{a^{2} \cos^{2}{\left(\theta \right)} + r^{2}} & \frac{- 3 a^{2} \sin^{2}{\left(\theta \right)} - a^{2} - 2 i a m \cos{\left(\theta \right)} + 2 m r - r^{2}}{4 \left(a^{2} \cos^{2}{\left(\theta \right)} + r^{2}\right)} & 0 & \frac{a}{a^{2} \cos^{2}{\left(\theta \right)} + r^{2}} \\
0 & 0 & - \frac{1}{a^{2} \cos^{2}{\left(\theta \right)} + r^{2}} & 0 \\
- \frac{a}{a^{2} \cos^{2}{\left(\theta \right)} + r^{2}} & \frac{a}{a^{2} \cos^{2}{\left(\theta \right)} + r^{2}} & 0 & - \frac{1}{\left(a^{2} \cos^{2}{\left(\theta \right)} + r^{2}\right) \sin^{2}{\left(\theta \right)}} \\
\end{matrix}\right),
\end{equation}

por conveniencia se definen las siguientes variables:
\begin{equation}
    \Sigma = r^2 + a^2 \cos^2 \theta, \quad \Delta = r^2 - 2 m r + a^2
\end{equation}



\begin{task}
    Verificar que la métrica obtenida es equivalente a la métrica de Kerr en coordenadas de Boyer-Lindquist.

\end{task}
La mewtrica en coordenadas de Boyer-Lindquist es:

\begin{equation}
    g_{\mu \nu}=\left(\begin{array}{cccc}
-\left(1-\frac{2 M r}{\Sigma}\right) c^2 & 0 & 0 & -\frac{2 M a r \sin ^2 \theta}{\Sigma} c \\
0 & \frac{\Sigma}{\Delta} & 0 & 0 \\
0 & 0 & \Sigma & 0 \\
-\frac{2 M a r \sin ^2 \theta}{\Sigma} c & 0 & 0 & \left(r^2+a^2+\frac{2 M r a^2 \sin ^2 \theta}{\Sigma}\right) \sin ^2 \theta
\end{array}\right)
    \label{eq:MetricaKerrBL}
\end{equation}
Donde funciones $\Sigma$ y $\Delta$ son:
- $\Sigma=r^2+a^2 \cos ^2 \theta$
- $\Delta=r^2-2 M r+a^2$

\section{Análisis de la métrica de Kerr}
\subsection{Símbolos de Christoffel}
Se uso el algoritmo que aparece en los anexos para calcular los símbolos de Christoffel no cero de la métrica de Kerr

\section{Diagramas de Penrose y estructura causal.}
la idea detrás de los diagramas espacio-tiempo  de Penrose es representar la estructura causal de un espacio-tiempo en un diagrama bidimensional finito, mediante una compactación conforme , estos muestran la propagación de señales y partículas en el espacio-tiempo.

En un diagrama de Penrose, el espacio-tiempo se representa en un plano bidimensional, donde el eje vertical representa el tiempo y el eje horizontal representa el espacio. En esta representación se mantienen los conos de luz formando un angulo de $45$ grados . Las líneas horizontales y verticales representan  tiempo constante y espacio constante respectivamente .


Con el propósito de construir los diagramas de Penrose, podemos seguir la siguiente receta,
(Es una receta en términos de que no hay un algoritmo perfectamente delimitado, siempre nosotros debemos de agregar algo mas )


\begin{enumerate}
  \item Empieza con alguna métrica en algún sistema de coordenadas no compacto (nota el rango de las coordenadas)
  \item Encuentra una coordenadas tales que las anteriores sean reemplazadas por coordenadas nulas
  \item Compacta las dos coordenadas nulas de forma separada
        Para esto hay varias maneras y varias funciones que cumplen este requisito, por ejemplo $sinh(x)$ o $tanh(x)$ o $tan^{-1}(x)$, cada una de estas son funciones biyectivas de los reales $\mathbb{R} $ a un dominio compacto.


        En este caso se introduce las nuevas coordenadas como:
        \begin{equation}
          \begin{array}{l}
            p:=\tan ^{-1}(v) \\
            q:=\tan ^{-1}(w)
          \end{array}
        \end{equation}
        Donde el par ordenado $(p,q)$ tomara valores dentro de un subconjunto de rango
        $(\frac{-\pi}{2},\frac{\pi}{2})X(\frac{-\pi}{2},\frac{\pi}{2})$
        (La elección  de la función tangente inversa es arbitraria, pero es una elección común, para el caso de Minkowski es posible también hacer  esta compactación con la función $sinh(x)$)
  \item Define de nuevo coordenadas espaciales y temporales
        \begin{equation}
          \begin{array}{l}
            T:=p+q \\
            X:=p-q
          \end{array}
        \end{equation}
        (Mantén una "bitácora" de los rangos de estas coordenadas, ya que la información de esta transformación esta dada por donde se encuentran las nuevas fronteras finitas)
\end{enumerate}
\subsection{Diagrama en el espacio-tiempo de Minkowski}
\begin{figure}[H]
    \centering
    
% PENROSE DIAGRAM of Minkowski space - 45 rotation
\begin{tikzpicture}[scale=3.2]
  \message{Penrose diagram (45 rotation)^^J}
  
  \def\R{0.10} % size lightcone
  \def\Nlines{6} % number of world lines (at constant r/t)
  \pgfmathsetmacro\d{0.92/\Nlines} % grid size
  
  \coordinate (O) at (0,0);
  \coordinate (W) at (-1.05,0);
  \coordinate (E) at (1.15,0);
  \coordinate (S) at (0,-1.05);
  \coordinate (N) at (0,1.15);
  \coordinate (SW) at (-135:1.45);
  \coordinate (SE) at (-45:1.45);
  \coordinate (NW) at (135:1.45);
  \coordinate (NE) at (45:1.45);
  \coordinate (X0) at (-0.41,-1);
  \coordinate (X1) at (-\d,-3*\d);
  \coordinate (X2) at (2*\d,2*\d);
  \coordinate (X3) at (0.54,1);
  
  % WORLD LINES GRID
  \message{Making world lines...^^J}
  \foreach \i [evaluate={\x=\i*\d;}] in {1,...,\Nlines}{
    \message{  Running i/N=\i/\Nlines, x=\x...^^J}
    \draw[world line]   (-\x,-1) -- (-\x,1);
    \draw[world line]   ( \x,-1) -- ( \x,1);
    \draw[world line t] (-1,-\x) -- (1,-\x);
    \draw[world line t] (-1, \x) -- (1, \x);
  }
  
  % AXES
  \draw[->,thick,mydarkblue!70!black]
    (W) -- (E) node[left=4,below=0] {$x$};
  \draw[->,thick,mydarkpurple!70!black]
    (S) -- (N) coordinate (N) node[below=4,left=0] {$t$};
  \draw[->,thick,mydarkred] (SW) -- (NE) node[below right=-2] {$u=x+t$};
  \draw[->,thick,mydarkred] (SE) -- (NW) node[below left=-2] {$x-t=v$};
  
  \draw pic[->,"$45^\circ$"{above,scale=0.9},draw=black,angle radius=16,
            angle eccentricity=1.0] {angle = N--O--NW};
  
  % INFINITY LABELS
  \node[above=1,left=1,mydarkblue] at (W) {$i^0$};
  \node[above=1,right=1,mydarkblue] at (E) {$i^0$};
  \node[right=3,below=2,mydarkpurple] at (0,-1) {$i^-$};
  \node[right=3,above=0,mydarkpurple] at (N) {$i^+$};
  \node[mydarkblue,left=5,below right=-1] at (SE) {$\calI^-$};
  \node[mydarkblue,right=8,below left=-1] at (SW) {$\calI^-$};
  \node[mydarkblue,left=5,above right=-1] at (NE) {$\calI^+$};
  \node[mydarkblue,right=8,above left=-1] at (NW) {$\calI^+$};
  
  % LIGHT CONE BACK
  \coneback{X1};
  \coneback{X2};
  
  % PARTICLE
  \draw[particle,decoration={markings,mark=at position 0.170 with {\arrow{latex}},
                                      mark=at position 0.505 with {\arrow{latex}},
                                      mark=at position 0.860 with {\arrow{latex}}},postaction={decorate}]
    (X0) to[out=70,in=-110] (X1) to[out=70,in=-110] (X2) to[out=70,in=-120] (X3);
  
  % LIGHT CONE FRONT
  \conefront{X1};
  \conefront{X2};
  
\end{tikzpicture}
    \caption{Caption}
\end{figure}

El espacio-tiempo de Minkowski es la solución mas sencilla de las ecuaciones de campo, donde las coordenadas poseen los rangos:
\begin{equation}
  \begin{array}{c|c|c|c}
    t                  & r           & \theta   & \varphi   \\
    ( -\infty, \infty) & (0, \infty) & (0, \pi) & (0, 2\pi)
  \end{array}
\end{equation}
y la métrica es:
\begin{equation}
  ds^2 = -c^2dt^2 + dr^2 + r^2 d\Omega^2,
\end{equation}
Definimos las coordenadas nulas $v$ y $w$ como:

\begin{equation}
  \begin{array}{rcl|rcl}
    v & := & t+r & r & = & \frac{1}{2}(v-w) \\
    w & := & t-r & t & = & \frac{1}{2}(v+w)
  \end{array}
\end{equation}

Donde los rangos de las nuevas coordenadas son  $r=\frac{1}{2}(v-w)>0 \longrightarrow v>w$.
Las coordenadas nulas $v$ y $w$ son compactificadas con la función tangente inversa, obteniendo las coordenadas $p$ y $q$:
\begin{equation}
  \begin{aligned}
    p:=\tan ^{-1}(v) \\
    q :=\tan ^{-1}(w)
  \end{aligned}
\end{equation}
Debido a que la función tangente inversa es monótona creciente, se tiene que $v>w \longrightarrow p>q$ y cambia el rango del par ordenado $v,w \in (-\infty, \infty) \longrightarrow p,q \in \left(-\frac{\pi}{2}, \frac{\pi}{2}\right)$.

Una vez que las coordenadas nulas han sido compactificadas, se definen las coordenadas temporales y espaciales $T$ y $X$ como:
\begin{equation}
  \begin{array}{rcl|rcl}
    T & := & p + q, & p & = & \tfrac{1}{2}(T + X) \\[4pt]
    X & := & p - q, & q & = & \tfrac{1}{2}(T - X)
  \end{array}
\end{equation}
Ya que los argumentos de $tan(\theta)$ esta entre $(-\frac{\pi}{2}, \frac{\pi}{2})$ y $p>q$, los nuevos rangos de estas coordenadas son:

\begin{equation}
  \left[\begin{array}{c}
      -\frac{\pi}{2}<\frac{1}{2}(T+X)<\frac{\pi}{2} \\
      -\frac{\pi}{2}<\frac{1}{2}(T-X)<\frac{\pi}{2} \\
      \frac{1}{2}(T+X)>\frac{1}{2}(T-X)
    \end{array}\right] \Leftrightarrow\left[\begin{array}{l}
      -\pi<T+X<\pi \\
      -\pi<T-X<\pi \\
      X>0
    \end{array}\right]
\end{equation}
En base  a estas desigualdades  la región permitida
\begin{figure}[H]
  \centering
  \begin{tikzpicture}[scale=0.8]
    % Definimos la constante pi
    \pgfmathsetmacro{\mypi}{3.1416}

    % Fondo gris para toda el área de dibujo (ajustable)
    %\fill[gray!30] (-4,-4) rectangle (4,4);

    % Región permitida (blanca): para x>0, T entre -pi+x y pi-x
    \fill[pattern=north east lines, pattern color=black!60]
    (0,-\mypi) -- (0,\mypi) -- (\mypi,0) -- cycle;
    % Límites sólidos de la región permitida:
    % T+x=\pi  =>  T=\pi-x, que une (0,\pi) con (\mypi,0)
    % T-x=-\pi =>  T=-\pi+x, que une (0,-\mypi) con (\mypi,0)
    % y la recta x=0 (eje vertical) que une (0,-\mypi) con (0,\mypi)
    \draw[line width=1.5pt] (0,-\mypi) -- (0,\mypi) -- (\mypi,0) -- cycle;

    % Límites adicionales (discontinuos) de las otras inecuaciones:
    % 1. T+x=-\pi  =>  T=-\pi-x (línea discontinua en verde)
    % Se dibuja en un rango amplio de x para visualizarla.
    \draw[dashed,green] (-4, {4 - \mypi}) -- (0.9, {-0.9 - \mypi});

    % 2. T-x=\pi   =>  T=\pi+x (línea discontinua en magenta)
    \draw[dashed,magenta] (-4, { \mypi - 4 }) -- (0.9, { \mypi + 0.9 });

    % Etiquetas para los límites (se pueden ajustar las posiciones):
    % Etiqueta de la frontera T+x=\pi (límite sólido superior de la región)
    \node[above, black] at ({0.8*\mypi}, {0.5*\mypi}) {$T+x=\pi$};

    % Etiqueta de la frontera T-x=-\pi (límite sólido inferior de la región)
    \node[below, black] at ({0.8*\mypi}, {-0.5*\mypi}) {$T-x=-\pi$};

    % Etiqueta para la frontera T+x=-\pi (línea discontinua, verde)
    % Se coloca en un punto representativo, por ejemplo, en x=-2.
    \node[above left, black] at (-2, { -\mypi  }) {$T+x=-\pi$};

    % Etiqueta para la frontera T-x=\pi (línea discontinua, magenta)
    \node[above left, black] at (-2, { \mypi - 2 }) {$T-x=\pi$};

    % Etiqueta para la frontera x=0 (eje vertical)
    \node[above left] at (0,0) {$x=0$};

    % Se dibujan los ejes coordenados
    \draw[->] (-4,0) -- (4,0) node[right] {$X$};
    \draw[->] (0,-4) -- (0,4) node[above] {$T$};

    % Etiquetas de los vértices de la región permitida
    \node[below left] at (0,-\mypi) {$(0,-\pi)$};
    \node[above left] at (0,\mypi) {$(0,\pi)$};
    \node[below right] at (\mypi,0) {$(\pi,0)$};

  \end{tikzpicture}
  \caption{Regiones definidas por \(-\pi<T+x<\pi\), \(-\pi<T-x<\pi\) y \(x>0\). Límites sólidos: región permitida (\(T+x=\pi\) y \(T-x=-\pi\)); discontinuos: fronteras de las inecuaciones restantes.}
\end{figure}
Para el caso de Minkowski, la región permitida es un rombo a la derecha, donde las líneas diagonales representan la luz, las líneas horizontales representan el espacio constante y las líneas verticales representan el tiempo constante, mediante la transformación de coordenadas.
\begin{equation}
  \begin{aligned}
    X = \tan^{-1}(t+r) + \tan^{-1}(t-r) \\
    Y = \tan^{-1}(t+r) - \tan^{-1}(t-r)
  \end{aligned}
\end{equation}
con el rango $r>0$, $-\infty<t<\infty$.
\begin{figure}[H]
  \centering % Centra el diagrama en la página
  % PENROSE DIAGRAM of Minkowski space - radius r
  \begin{tikzpicture}[scale=4]
    \message{Penrose diagram (radius r)^^J}

    \def\Nlines{4} % number of world lines (at constant r/t)
    \def\ta{tan(90*1.0/(\Nlines+1))} % constant r/t value 1
    \def\tb{tan(90*2.0/(\Nlines+1))} % constant r/t value 2
    \coordinate (O) at ( 0, 0); % center: origin (r,t) = (0,0)
    \coordinate (S) at ( 0,-1); % south: t=-infty, i-
    \coordinate (N) at ( 0, 1); % north: t=+infty, i+
    \coordinate (E) at ( 1, 0); % east:  r=+infty, i0
    \coordinate (X) at ({penroseu(\tb,\tb)},{penrosev(\tb,\tb)});
    \coordinate (X0) at ({penroseu(\ta,-\tb)},{penrosev(\ta,-\tb)});

    % AXES
    \fill[mylightblue] (N) -- (E) -- (S) -- cycle;
    \draw[->,thick] (-0.1,0) -- (1.2,0) node[below right=-2] {$u$};
    \draw[->,thick] (0,-1.1) -- (0,1.2) node[left=-1] {$v$};

    % INFINITY LABELS
    \node[above=1,above left=0,mydarkblue,align=center] at (O)
    {$r=0$};
    \node[left=6,above right=-2,mydarkblue,align=center] at (1,0.04)
    {spacelike\\[-2]infinity ($i^0$)\\[-2]$r=+\infty$};
    \node[above=6,below right=0,mydarkpurple,align=left] at (0.04,-1)
    {$t=-\infty$\\[-2]past timelike\\[-2]infinity ($i^-$)};
    \node[below=6,above right=0,mydarkpurple,align=left] at (0.04,1)
    {future timelike\\[-2]infinity ($i^+$)\\[-2]$t=+\infty$};
    \node[mydarkblue,above right,align=right] at (57:0.68)
    {future lightlike\\[-2]infinity ($\calI^+$)};
    \node[mydarkblue,below right,align=right] at (-60:0.68)
    {past lightlike\\[-2]infinity ($\calI^-$)};

    % CONE BACK
    \coneback{X};
    \coneback{X0};

    % WORLD LINES
    \draw[world line] (N) -- (S);
    \draw[world line] (O) -- (E);
    \message{Making world lines...^^J}

    \foreach \i [evaluate={\c=\i/(\Nlines+1); \ct=tan(90*\c);}] in {1,...,\Nlines}{
        \message{  Running i/N=\i/\Nlines, c=\c, tan(90*\c)=\ct...^^J}
        \draw[world line t,samples=\Nsamples,smooth,variable=\t,domain=0.001:1] % constant t
        plot(\t,{-penrose(\t*pi/2,\ct)})
        plot(\t,{ penrose(\t*pi/2,\ct)});
        \draw[world line,samples=\Nsamples,smooth,variable=\r,domain=-1:1] % constant r
        plot({penrose(\r*pi/2,\ct)},\r);
      }
    \draw[thick,mydarkblue] (N) -- (E) -- (S) -- cycle;

    % CONSTANT
    \draw[->,mydarkpurple!80!black,shorten <=0.4] % constant r
    (0.66,{-penrose(0.66*pi/2,tan(90*3/(\Nlines+1)))}) to[out=-70,in=150]++ (-45:0.23)
    node[right=-1] {$t=\text{constant}$};

    \draw[->,mydarkblue!80!black,shorten <=0.4] % constant t
    ({penrose(-0.27*pi/2,tan(90*3/(\Nlines+1)))},-0.27) to[out=-55,in=170]++ (-35:0.3)
    node[right=-1] {$r=\text{constant}$};

    % PARTICLE
    \draw[particle,decoration={markings,mark=at position 0.24 with {\arrow{latex}},
          mark=at position 0.55 with {\arrow{latex}},
          mark=at position 0.82 with {\arrow{latex}}},postaction={decorate}]
    (S) to[out=90,in=-80] (X0) to[out=100,in=-95] (X) to[out=85,in=-90] (N);

    % LIGHT CONE FRONT
    \conefront{X};
    \conefront{X0};

    % PHOTON
    \draw[->,photon] (O) -- (0.5,0.5) node[above=2,right=1] {photon};

    % TICKS
    \tick{E}{90} node[right=4,below=-1] {$+\pi$};
    \tick{S}{ 0} node[left=-1] {$-\pi$};
    \tick{N}{ 0} node[left=-1] {$+\pi$};

  \end{tikzpicture}
  \caption{Diagrama de Penrose para el espacio-tiempo de Minkowski, mostrando la estructura causal compactada. Las líneas curvas representan hipersuperficies de tiempo constante (Recuperado de \cite{neutelings-2024}).}
\end{figure}
\subsection{Diagrama de espacio-tiempo de Schwarzschild}

Comenzamos con la métrica de Schwarzschild en las coordenadas de Kruskal-Szekeres $(T, X)$. La parte radial-temporal de la métrica es conformemente plana, es decir, proporcional a la métrica de Minkowski:
\begin{equation}
  ds^2 = \frac{4r_s^3}{r} e^{-r/r_s}(-dT^2 + dX^2) + r^2 d\Omega^2
\end{equation}
La estructura causal está determinada por el término $-dT^2 + dX^2$. Las coordenadas $(T, X)$ definen las regiones de interés a través de las siguientes relaciones:

\begin{itemize}
  \item \textbf{Región I (Exterior, $r > r_s$):} Se caracteriza por la condición $X^2 - T^2 > 0$ con $X > 0$. Geométricamente, esta es la región a la derecha del origen, contenida entre las hipérbolas nulas $T = \pm X$.

  \item \textbf{Región II (Interior, $0 < r < r_s$):} Se caracteriza por la condición $T^2 - X^2 > 0$ con $T > 0$. Esta es la región futura, contenida entre las hipérbolas nulas $T = \pm X$.
\end{itemize}

De forma análoga al espaciotiempo de Minkowski, definimos las coordenadas nulas $(U, V)$ a partir de $(T, X)$:
\begin{equation}
  \begin{array}{l}
    V := T+X \\
    U := T-X
  \end{array}
\end{equation}
A continuación, determinamos el rango de estas coordenadas para cada región.

\begin{itemize}
  \item \textbf{Para la Región I ($X > |T|$):}
        \begin{itemize}
          \item $V = T+X > 0 \implies V \in (0, \infty)$
          \item $U = T-X < 0 \implies U \in (-\infty, 0)$
        \end{itemize}

  \item \textbf{Para la Región II ($T > |X|$):}
        \begin{itemize}
          \item $V = T+X > 0 \implies V \in (0, \infty)$
          \item $U = T-X > 0 \implies U \in (0, \infty)$
        \end{itemize}
\end{itemize}



Utilizamos la función tangente inversa para mapear los rangos infinitos de $U$ y $V$ a un intervalo finito. Definimos las nuevas coordenadas compactificadas $(p, q)$:
\begin{equation}
  \begin{array}{l}
    p := \tan^{-1}(V) \\
    q := \tan^{-1}(U)
  \end{array}
\end{equation}
Los rangos de $(p, q)$ se transforman de la siguiente manera:

\begin{itemize}
  \item \textbf{Para la Región I ($V>0, U<0$):}
        \begin{itemize}
          \item $p = \tan^{-1}(V)$ toma valores en $\left(0, \frac{\pi}{2}\right)$.
          \item $q = \tan^{-1}(U)$ toma valores en $\left(-\frac{\pi}{2}, 0\right)$.
        \end{itemize}

  \item \textbf{Para la Región II ($V>0, U>0$):}
        \begin{itemize}
          \item $p = \tan^{-1}(V)$ toma valores en $\left(0, \frac{\pi}{2}\right)$.
          \item $q = \tan^{-1}(U)$ toma valores en $\left(0, \frac{\pi}{2}\right)$.
        \end{itemize}
\end{itemize}



Finalmente, definimos las coordenadas del diagrama de Penrose, que llamaremos $(T', X')$, a partir de $(p, q)$:
\begin{equation}
  \begin{array}{l}
    T' := p+q \\
    X' := p-q
  \end{array}
\end{equation}
Ahora, calculamos los límites de $(T', X')$ para cada región para determinar su forma en el diagrama final.

\begin{itemize}
  \item \textbf{Límites para la Región I:} Con $p \in (0, \frac{\pi}{2})$ y $q \in (-\frac{\pi}{2}, 0)$:
        \begin{align*}
          T' = p+q \quad & \implies \quad T' \in (-\pi/2, \pi/2) \\
          X' = p-q \quad & \implies \quad X' \in (0, \pi)
        \end{align*}
        Podemos expresar estas condiciones de forma más restrictiva. Sabemos que $p = \frac{1}{2}(T'+X')$ y $q = \frac{1}{2}(T'-X')$. Aplicando los rangos de $p$ y $q$:
        \begin{equation}
          \left[\begin{array}{c}
              0 < \frac{1}{2}(T'+X') < \frac{\pi}{2} \\
              -\frac{\pi}{2} < \frac{1}{2}(T'-X') < 0
            \end{array}\right] \Leftrightarrow \left[\begin{array}{c}
              0 < T'+X' < \pi \\
              -\pi < T'-X' < 0
            \end{array}\right] \Leftrightarrow
          \left[\begin{array}{c}
              T' > -X'    \\
              T' < \pi-X' \\
              T' < X'     \\
              T' > X'-\pi
            \end{array}\right]
        \end{equation}
        Estas desigualdades definen un triángulo con vértices en $(0,0)$, $(\pi/2, \pi/2)$ y $(\pi/2, -\pi/2)$.

  \item \textbf{Límites para la Región II:} Con $p \in (0, \frac{\pi}{2})$ y $q \in (0, \frac{\pi}{2})$:
        \begin{align*}
          T' = p+q \quad & \implies \quad T' \in (0, \pi)        \\
          X' = p-q \quad & \implies \quad X' \in (-\pi/2, \pi/2)
        \end{align*}
        Aplicando los rangos de $p$ y $q$ de manera similar:
        \begin{equation}
          \left[\begin{array}{c}
              0 < \frac{1}{2}(T'+X') < \frac{\pi}{2} \\
              0 < \frac{1}{2}(T'-X') < \frac{\pi}{2}
            \end{array}\right] \Leftrightarrow \left[\begin{array}{c}
              0 < T'+X' < \pi \\
              0 < T'-X' < \pi
            \end{array}\right] \Leftrightarrow
          \left[\begin{array}{c}
              T' > -X'    \\
              T' < \pi-X' \\
              T' > X'     \\
              T' < \pi+X'
            \end{array}\right]
        \end{equation}
        Estas desigualdades definen un triángulo adyacente al anterior, con vértices en $(0,0)$, $(\pi/2, \pi/2)$ y $(-\pi/2, \pi/2)$.
\end{itemize}

Las regiones permitidas son
\begin{figure}[H]
  \begin{subfigure}{0.5\textwidth}
    \centering
    \begin{tikzpicture}[scale=0.6]
      \pgfmathsetmacro{\mypi}{3.1416}

      %===========================
      % Región permitida (relleno gris)
      %===========================
      \draw[line width=1.5pt] (0,0) -- (\mypi /2 ,\mypi/2) -- (\mypi,0) -- (\mypi /2 ,-\mypi/2) -- cycle;
      \fill[pattern=north east lines, pattern color=black!60](0,0) -- (\mypi /2 ,\mypi/2) -- (\mypi,0) -- (\mypi /2 ,-\mypi/2) -- cycle;



      %===========================
      % Líneas límites
      %===========================
      % T' = X'
      \draw[dashed,red,thick] (-3,-3) -- (3,3) node[above right] {$T' = X'$};

      % T' = -X'
      \draw[dashed,blue,thick] (-3,3) -- (3,-3) node[below right] {$T' = -X'$};

      % T' = \pi - X'
      \draw[dashed,green!70!black,thick] (0, {3.1416}) -- (3, {3.1416-3}) node[above right] {$T' = \pi - X'$};

      % T' = X' - \pi
      \draw[dashed,orange!90!black,thick] (0, {-3.1416}) -- (3, {-3.1416+3}) node[below right] {$T' = X' - \pi$};

      %===========================
      % Ejes coordenados
      %===========================
      \draw[->] (-4,0) -- (4,0) node[right] {$X'$};
      \draw[->] (0,-4) -- (0,4) node[above] {$T'$};

      %===========================
      % Etiqueta de la región
      %===========================


    \end{tikzpicture}
    \caption{Región definida por el sistema  $\;T' > -X',\; T' < \pi - X',\; T' < X',\; T' > X' - \pi$. Las líneas discontinuas marcan los límites del dominio.}
  \end{subfigure}
  \begin{subfigure}{0.5\textwidth}
    \centering
    \begin{tikzpicture}[scale=0.6]
      \pgfmathsetmacro{\mypi}{3.1416}

      %===========================
      % Región II (relleno con patrón)
      %===========================
      \fill[pattern=north east lines, pattern color=black!60, opacity=0.6]
      (-\mypi/2,\mypi/2)
      -- (\mypi/2,\mypi/2)
      -- (0,0)
      -- cycle;
      \draw[line width=1.5pt]    (-\mypi/2,\mypi/2)
      -- (\mypi/2,\mypi/2)
      -- (0,0)
      -- cycle;
      %===========================
      % Líneas límites
      %===========================
      % T' = X'
      \draw[dashed,red,thick] (-3,-3) -- (3,3) node[above right] {$T' = X'$};

      % T' = -X'
      \draw[dashed,blue,thick] (-3,3) -- (3,-3) node[below right] {$T' = -X'$};

      % T' = \pi - X'
      \draw[dashed,green!70!black,thick] (-0.5, {3.1416+0.5}) -- (3, {3.1416-3}) node[above right] {$T' = \pi - X'$};

      % T' = \pi + X'
      \draw[dashed,orange!90!black,thick] (-3, {3.1416-3}) -- (0+0.5, {3.1416+0.5}) node[above right] {$T' = \pi + X'$};
      % T' = \pi
      \draw[dashed,orange!90!black,thick]  (-3.5, {\mypi/2}) -- (3.5, {\mypi/2}) node[above right] {$T' = \pi/2 $};


      %===========================
      % Ejes coordenados
      %===========================
      \draw[->] (-4,0) -- (4,0) node[right] {$X'$};
      \draw[->] (0,-4) -- (0,4) node[above] {$T'$};

      %===========================
      % Etiqueta de la región
      %===========================

    \end{tikzpicture}
    \caption{Región II definida por $\;T' > -X',\; T' < \pi - X',\; T' > X',\; T' < \pi + X'$. Se ubica a la izquierda del eje \(X'=0\), entre las líneas nulas y sus traslaciones en \(\pi\).}
  \end{subfigure}
  \caption{Regiones permitidas en el diagrama de Penrose para un agujero de Schwarzschild.}
\end{figure}
Por composición de constantes, las funciones las funciones
$$
  \begin{array}{l}
    T^{\prime}=p+q=\tan ^{-1}(T+X)+\tan ^{-1}(T-X) \\
    X^{\prime}=p-q=\tan ^{-1}(T+X)-\tan ^{-1}(T-X)
  \end{array}
$$

Con el dominio interior definido por $-\frac{\pi}{2}<X^{\prime}<\frac{\pi}{2}$, $0<T^{\prime}<\frac{\pi}{2}$, y el exterior $0<X^{\prime}<\pi$, $-\frac{\pi}{2}<T^{\prime}<\frac{\pi}{2}$.

Sustituyendo las expresiones de $T$ y $X$ en estas ecuaciones se obtienen $T^{\prime}$ y $X^{\prime}$ en términos de $r$ y $t$.
- Para la región exterior ( $r>r_s$ ):

$$
  \begin{aligned}
    T & =\sqrt{\frac{r}{r_s}-1} e^{r /\left(2 r_s\right)} \sinh \left(\frac{t}{2 r_s}\right) \\
    X & =\sqrt{\frac{r}{r_s}-1} e^{r /\left(2 r_s\right)} \cosh \left(\frac{t}{2 r_s}\right)
  \end{aligned}
$$

- Para la región interior ( $0<r<r_s$ ):

$$
  \begin{aligned}
    T & =\sqrt{1-\frac{r}{r_s}} e^{r /\left(2 r_s\right)} \cosh \left(\frac{t}{2 r_s}\right) \\
    X & =\sqrt{1-\frac{r}{r_s}} e^{r /\left(2 r_s\right)} \sinh \left(\frac{t}{2 r_s}\right)
  \end{aligned}
$$



Donde las funciones para las lineas de $r, t$ constantes son:
\begin{equation}
  \begin{aligned}
    \tan\left(\frac{T'+X'}{2}\right) \tan\left(\frac{T'-X'}{2}\right)                                                                               & = \left(1-\frac{r}{r_s}\right)e^{r/r_s} \\
    \frac{\tan\left(\frac{T'+X'}{2}\right) + \tan\left(\frac{T'-X'}{2}\right)}{\tan\left(\frac{T'+X'}{2}\right) - \tan\left(\frac{T'-X'}{2}\right)} & = \tanh\left(\frac{t}{2r_s}\right)
  \end{aligned}
\end{equation}
A partir de estas ecuaciones, podemos identificar las superficies más importantes del espacio-tiempo de Schwarzschild en el diagrama de Penrose:

\begin{itemize}
  \item \textbf{El horizonte de eventos ($r=r_s$):} Si sustituimos $r=r_s$ en la primera ecuación, el lado derecho se anula:
        \begin{equation}
          \tan\left(\frac{T'+X'}{2}\right) \tan\left(\frac{T'-X'}{2}\right) = \left(1-\frac{r_s}{r_s}\right)e^{r_s/r_s} = 0
        \end{equation}
        Esto implica que $\tan\left(\frac{T'+X'}{2}\right)=0$ o $\tan\left(\frac{T'-X'}{2}\right)=0$. Dentro de los rangos de las coordenadas del diagrama, esto se cumple cuando $T'+X'=0$ o $T'-X'=0$. Por lo tanto, el horizonte de eventos corresponde a las líneas diagonales $T' = \pm X'$.

  \item \textbf{La singularidad ($r=0$):} Al sustituir $r=0$ en la misma ecuación, obtenemos:
        \begin{equation}
          \tan\left(\frac{T'+X'}{2}\right) \tan\left(\frac{T'-X'}{2}\right) = (1-0)e^0 = 1
        \end{equation}
        Esta condición es equivalente a $\cos(T')=0$. En el diagrama, esto corresponde a las líneas horizontales $T' = \pi/2$ (la singularidad futura, hacia la que todo cae) y $T' = -\pi/2$ (la singularidad pasada, de la que todo emerge en el caso de un agujero blanco).
\end{itemize}

De forma similar a las coordenadas de Kruskal-Szekeres, el diagrama de Penrose revela la estructura causal del espacio-tiempo de Schwarzschild. Las líneas diagonales representan los horizontes de eventos, mientras que las líneas horizontales representan las singularidades. La región exterior (Región I) está separada de la región interior (Región II) por estas líneas diagonales, mostrando claramente que nada puede escapar del interior una vez que cruza el horizonte de eventos.

\begin{figure}[H]
  \centering % Centra el diagrama en la página
  % PENROSE DIAGRAM of a Schwarzschild black hole
  \begin{tikzpicture}[scale=3.2]
    \message{Extended Penrose diagram: Schwarzschild black hole^^J}

    \def\R{0.08} % size lightcone
    \def\Nlines{3} % number of world lines (at constant r/t)
    \pgfmathsetmacro\ta{1/sin(90*1/(\Nlines+1))} % constant r/t value 1
    \pgfmathsetmacro\tb{sin(90*2/(\Nlines+1))}   % constant r/t value 2
    \pgfmathsetmacro\tc{1/sin(90*2/(\Nlines+1))} % constant r/t value 3
    \pgfmathsetmacro\td{sin(90*1/(\Nlines+1))}   % constant r/t value 4
    \coordinate (-O) at (-1, 0); % center III: origin (r,t) = (0,0)
    \coordinate (-N) at (-1, 1); % north III: t=+infty, i+
    \coordinate (O)  at ( 1, 0); % center I: origin (r,t) = (0,0)
    \coordinate (S)  at ( 1,-1); % south I: t=-infty, i-
    \coordinate (N)  at ( 1, 1); % north I: t=+infty, i+
    \coordinate (E)  at ( 2, 0); % east I:  r=-infty, i0
    \coordinate (W)  at ( 0, 0); % west I:  r=+infty, i0
    \coordinate (B)  at ( 0,-1); % singularity bottom
    \coordinate (X0) at ({asin(sqrt((\ta^2-1)/(\ta^2-\tb^2)))/90},
    {-acos(\ta*sqrt((1-\tb^2)/(\ta^2-\tb^2)))/90}); % particle 1
    \coordinate (X1) at ({asin(sqrt((\tc^2-1)/(\tc^2-\td^2)))/90},
    {acos(\tc*sqrt((1-\td^2)/(\tc^2-\td^2)))/90}); % particle 2
    \coordinate (X2) at (45:0.87); % particle falling in BH horizon
    \coordinate (X3) at (0.60,1.05); % particle falling in BH singularity

    % AXES
    \draw[->,thick] (0,-0.1) -- (0,1.15) node[above=1,left=-1] {$v$};
    \draw[->,thick] (-0.1,0) -- (2.15,0) node[left=1,above=0] {$u$};

    \begin{scope}

      % CLIP to fill inside zigzag lines
      \clip[decorate,decoration={zigzag,amplitude=2,segment length=6.17}]
      (-N) -- (N) --++ (1.1,0.1) |-++ (-3.1,-2.3) -- cycle;

      % REGIONS FILLS
      %\fill[mylightpurple] (-N) |-++ (2,0.1) -- (N) -- (W) -- cycle;
      %\fill[mylightblue] (N) -- (E) -- (S) -- (W) -- cycle;

      % CONE BACK
      \coneback{X0};
      \coneback{X1};
      \coneback{X2};

      % WORLD LINES
      \draw[world line] (N) -- (S);
      \draw[world line t] (W) -- (E) (W) -- (0,1.1);
      \message{Making world lines...^^J}
      \foreach \i [evaluate={\c=\i/(\Nlines+1); \cs=sin(90*\c);}] in {1,...,\Nlines}{
          \message{  Running i/N=\i/\Nlines, c=\c, cs=\cs...^^J}
          \draw[world line t,samples=\Nsamples,smooth,variable=\x,domain=0:2] % region I, constant t
          plot(\x,{-kruskal(\x*pi/4,\cs)})
          plot(\x,{ kruskal(\x*pi/4,\cs)});
          \draw[world line,samples=\Nsamples,smooth,variable=\y,domain=0:2] % region I, constant r
          plot({1-kruskal(\y*pi/4,\cs)},\y-1)
          plot({1+kruskal(\y*pi/4,\cs)},\y-1);
          \draw[world line,samples=\Nsamples,smooth,variable=\x,domain=0:2] % region II, constant r
          plot(\x-1,{1-kruskal(\x*pi/4,\cs)});
          \draw[world line t,samples=\Nsamples,smooth,variable=\y,domain=0:1.05] % region II constant t
          plot({-kruskal(\y*pi/4,\cs)},\y)
          plot({ kruskal(\y*pi/4,\cs)},\y);
        }

      % PARTICLE WORLD LINE
      \draw[particle,decoration={markings,mark=at position 0.16 with {\arrow{latex}},
            mark=at position 0.45 with {\arrow{latex}},
            mark=at position 0.72 with {\arrow{latex}},
            mark=at position 0.90 with {\arrow{latex}}},postaction={decorate}]
      (S) to[out=77,in=-70] (X0) to[out=110,in=-80] (X1)
      to[out=100,in=-90] (X2) to[out=75,in=-80] (X3);

    \end{scope}

    % LIGHT CONE FRONT
    \conefront{X0};
    \conefront{X1};
    \conefront{X2};

    % ESCAPING PHOTONS
    \draw[photon] (X0) ++ (45:0.1) --++ (45:0.3);
    \draw[photon] (X1) ++ (45:0.1) --++ (45:0.3);

    % REGIONS
    \node[fill=mylightblue,inner sep=2] at (O) {I};
    \node[fill=mylightpurple,inner sep=2] at (0,0.64) {II};

    % BOUNDARIES
    \draw[singularity] (-N) -- node[pos=0.46,above left=-2] {\strut singularity} (N);
    \draw[singularity] (-N) -- node[pos=0.54,above right=-2] {\strut $r=0$} (N);
    \path (S) -- (W) node[mydarkblue,pos=0.50,below=-2.5,rotate=-45,scale=0.85]
    {anti-horizon $r=2GM$};
    \path (W) -- (N) node[mydarkblue,pos=0.32,above=-2.5,rotate=45,scale=0.85]
    {\contour{mylightpurple}{horizon $r=2GM$}};
    \draw[thick,mydarkblue] (N) -- (E) -- (S) --  (W) -- cycle;
    \draw[thick,mydarkblue] (W) -- (-N);

    % TICKS
    \node[below left=-1] at (W) {$0$};
    \tick{E}{90} node[right=4,below=-3] {$\pi/2$};
    \tick{S}{0} node[left=-1] {$-\pi/2$};
    \tick{N}{180} node[right=-1] {$\pi/2$};

    % INFINITY LABELS
    \node[above=1,right=1,mydarkblue] at (2.15,0) {$i^0$};
    \node[right=1,below=1,mydarkpurple] at (S) {$i^-$};
    \node[right=1,above=1,mydarkpurple] at (N) {$i^+$};
    \node[mydarkblue,above right=-1] at (1.5,0.5) {$\calI^+$};
    \node[mydarkblue,below right=-2] at (1.5,-0.5) {$\calI^-$};

  \end{tikzpicture}
  \caption{Diagrama de Penrose para el agujero negro de Schwarzschild (Recuperado de \cite{neutelings-2024})}
\end{figure}
También se puede hacer la extensión máxima del espacio-tiempo de Schwarzschild, que incluye las regiones adicionales (Región III y IV) que representan un agujero blanco y un universo paralelo. En este caso, el diagrama de Penrose se extiende para incluir estas regiones, mostrando la simetría del espacio-tiempo alrededor del horizonte de eventos.


\begin{figure}[H]
  \centering % Centra el diagrama en la página
  \begin{tikzpicture}[scale=3.2]
    \message{Extended Penrose diagram: Schwarzschild black hole^^J}

    \def\R{0.08} % size lightcone
    \def\Nlines{3} % number of world lines (at constant r/t)
    \pgfmathsetmacro\ta{1/sin(90*1/(\Nlines+1))} % constant r/t value 1
    \pgfmathsetmacro\tb{sin(90*2/(\Nlines+1))}   % constant r/t value 2
    \pgfmathsetmacro\tc{1/sin(90*2/(\Nlines+1))} % constant r/t value 3
    \pgfmathsetmacro\td{sin(90*1/(\Nlines+1))}   % constant r/t value 4
    \coordinate (-O) at (-1, 0); % center III: origin (r,t) = (0,0)
    \coordinate (-S) at (-1,-1); % south III: t=-infty, i-
    \coordinate (-N) at (-1, 1); % north III: t=+infty, i+
    \coordinate (-W) at (-2, 0); % east III:  r=-infty, i0
    \coordinate (-E) at ( 0, 0); % west III:  r=+infty, i0
    \coordinate (O)  at ( 1, 0); % center I: origin (r,t) = (0,0)
    \coordinate (S)  at ( 1,-1); % south I: t=-infty, i-
    \coordinate (N)  at ( 1, 1); % north I: t=+infty, i+
    \coordinate (E)  at ( 2, 0); % east I:  r=-infty, i0
    \coordinate (W)  at ( 0, 0); % west I:  r=+infty, i0
    \coordinate (B)  at ( 0,-1); % singularity bottom
    \coordinate (T)  at ( 0, 1); % singularity top
    \coordinate (X0) at ({asin(sqrt((\ta^2-1)/(\ta^2-\tb^2)))/90},
    {-acos(\ta*sqrt((1-\tb^2)/(\ta^2-\tb^2)))/90}); % particle 1
    \coordinate (X1) at ({asin(sqrt((\tc^2-1)/(\tc^2-\td^2)))/90},
    {acos(\tc*sqrt((1-\td^2)/(\tc^2-\td^2)))/90}); % particle 2
    \coordinate (X2) at (45:0.87); % particle falling in BH horizon
    \coordinate (X3) at (0.60,1.05); % particle falling in BH singularity

    \begin{scope}

      % CLIP to fill inside zigzag lines
      \clip[decorate,decoration={zigzag,amplitude=2,segment length=6.17}]
      (S) -- (-S) --++ (-1.1,-0.1) |-++ (4.2,2.2) |- cycle;
      \clip[decorate,decoration={zigzag,amplitude=2,segment length=6.17}]
      (-N) -- (N) --++ (1.1,0.1) |-++ (-4.2,-2.2) |- cycle;

      % REGIONS FILLS
      %\fill[mylightpurple] (-N) |-++ (2,0.1) -- (N) -- (-S) -- (S) -- cycle;
      %\fill[mylightpurple] (-S) |-++ (2,-0.1) -- (S) -- (-N) -- (N) -- cycle;

      %\fill[mylightblue] (-N) -- (-E) -- (-S) -- (-W) -- cycle;
      %\fill[mylightblue] (N) -- (E) -- (S) -- (W) -- cycle;

      % CONE BACK
      \coneback{X0};
      \coneback{X1};
      \coneback{X2};

      % WORLD LINES
      \draw[world line] (-N) -- (-S) (N) -- (S);
      \draw[world line t] (-W) -- (-E) (W) -- (E) (0,-1.1) -- (0,1.1);
      \message{Making world lines...^^J}
      \foreach \i [evaluate={\c=\i/(\Nlines+1); \cs=sin(90*\c);}] in {1,...,\Nlines}{
          \message{  Running i/N=\i/\Nlines, c=\c, cs=\cs...^^J}
          \draw[world line t,samples=2*\Nsamples,smooth,variable=\x,domain=-2:2] % region I/III, constant t
          plot(\x,{-kruskal(\x*pi/4,\cs)})
          plot(\x,{ kruskal(\x*pi/4,\cs)});
          \draw[world line,samples=\Nsamples,smooth,variable=\y,domain=0:2] % region I/III, constant r
          plot({-1-kruskal(\y*pi/4,\cs)},\y-1)
          plot({-1+kruskal(\y*pi/4,\cs)},\y-1)
          plot({1-kruskal(\y*pi/4,\cs)},\y-1)
          plot({1+kruskal(\y*pi/4,\cs)},\y-1);
          \draw[world line,samples=\Nsamples,smooth,variable=\x,domain=0:2] % region II/IV, constant r
          plot(\x-1,{kruskal(\x*pi/4,\cs)-1})
          plot(\x-1,{1-kruskal(\x*pi/4,\cs)});
          \draw[world line t,samples=\Nsamples,smooth,variable=\y,domain=-1.05:1.05] % region II/IV constant t
          plot({-kruskal(\y*pi/4,\cs)},\y)
          plot({ kruskal(\y*pi/4,\cs)},\y);
        }

      % PARTICLE WORLD LINE
      \draw[particle,decoration={markings,mark=at position 0.16 with {\arrow{latex}},
            mark=at position 0.45 with {\arrow{latex}},
            mark=at position 0.72 with {\arrow{latex}},
            mark=at position 0.90 with {\arrow{latex}}},postaction={decorate}]
      (S) to[out=77,in=-70] (X0) to[out=110,in=-80] (X1)
      to[out=100,in=-90] (X2) to[out=75,in=-80] (X3);

    \end{scope}

    % BOUNDARIES
    \draw[singularity] (-N) -- node[above] {future singularity $r=0$} (N);
    \draw[singularity] (S) -- node[below] {past singularity $r=0$} (-S);
    \path (S) -- (W) node[mydarkblue,pos=0.50,below=-2.5,rotate=-45,scale=0.85]
    {\contour{mylightpurple}{anti-horizon $r=2GM$}};
    \path (W) -- (N) node[mydarkblue,pos=0.32,above=-2.5,rotate=45,scale=0.85]
    {\contour{mylightpurple}{horizon $r=2GM$}};
    \draw[thick,mydarkblue] (-N) -- (-E) -- (-S) -- (-W) -- cycle;
    \draw[thick,mydarkblue] (N) -- (E) -- (S) -- (W) -- cycle;

    % REGIONS
    \node[fill=white,inner sep=2] at (-O) {III};
    \node[fill=white,inner sep=2] at (O) {I};
    \node[fill=white,inner sep=2] at (0,0.64) {II};
    \node[fill=white,inner sep=2] at (0,-0.64) {IV};

    % INFINITY LABELS
    \node[above=1,left=1,mydarkblue] at (-2,0) {$i^0$};
    \node[above=1,right=1,mydarkblue] at (2,0) {$i^0$};
    \node[right=1,below=1,mydarkpurple] at (-S) {$i^-$};
    \node[right=1,above=1,mydarkpurple] at (-N) {$i^+$};
    \node[right=1,below=1,mydarkpurple] at (S) {$i^-$};
    \node[right=1,above=1,mydarkpurple] at (N) {$i^+$};
    \node[mydarkblue,below left=-1] at (-1.5,-0.5) {$\calI^-$};
    \node[mydarkblue,above left=-1] at (-1.5,0.5) {$\calI^+$};
    \node[mydarkblue,above right=-1] at (1.5,0.5) {$\calI^+$};
    \node[mydarkblue,below right=-1] at (1.5,-0.5) {$\calI^-$};

    % LIGHT CONE FRONT
    \conefront{X0};
    \conefront{X1};
    \conefront{X2};

    % ESCAPING PHOTONS
    \draw[photon] (X0) ++ (45:0.1) --++ (45:0.3);
    \draw[photon] (X1) ++ (45:0.1) --++ (45:0.3);

  \end{tikzpicture}
  \caption{Diagrama de Penrose extendido para el agujero negro de Schwarzschild (Recuperado de \cite{neutelings-2024})}
\end{figure}




%%%%%%%%%%%%%%%%%%%%%%%%%%%%%%%%%%%%%%%%%%%%%%%%%%%%%%%%%%%%%%%%
%   % material extra no se si agregarlo
%   \item \textbf{Transformación conforme en diagramas de Penrose:} Para construir el diagrama de Penrose, utilizamos una transformación conforme que reescala la métrica para compactificar el espacio-tiempo. La métrica en coordenadas $(T,X,\theta,\varphi)$ se expresa como:
%   \begin{equation}
%       g = \Omega^{-2}(T,X,\ldots)\left(dT\otimes dT - dX\otimes dX - R(T,X)[\text{términos angulares}]\right).
%   \end{equation}
%   El factor conforme $\Omega^{-2}$ aparece multiplicando toda la métrica. Definimos una nueva métrica (no física) multiplicando por $\Omega^2$:
%   \begin{equation}
%       g_{\text{diagram}} = \Omega^2 g.
%   \end{equation}
%   Sustituyendo la expresión de $g$ en la ecuación anterior, obtenemos:
%   \begin{equation}
%       g_{\text{diagram}} = \Omega^2 \times \Omega^{-2} \left( dT\otimes dT - dX\otimes dX - R(T,X)[\text{términos angulares}] \right).
%   \end{equation}
%   Como $\Omega^2 \times \Omega^{-2} = 1$, el factor conforme se cancela, dejando:
%   \begin{equation}
%       g_{\text{diagram}} = dT\otimes dT - dX\otimes dX - R(T,X)[\text{términos angulares}].
%   \end{equation}
%   Nos interesa principalmente la estructura causal, la cual está contenida en la parte $(T,X)$ de la métrica. Por lo tanto, suprimiendo los términos angulares, obtenemos:
%   \begin{equation}
%       g_{\text{diagram}} = dT\otimes dT - dX\otimes dX.
%   \end{equation}
%   La elección del factor conforme $\Omega$ y de las nuevas coordenadas $(T,X)$ se hace de manera que el infinito de la métrica original se mapea a puntos finitos en el diagrama. Esto permite representar todo el espacio-tiempo en una región compacta, facilitando el análisis de su estructura causal.
%   

%   Remarks: (1) One chooses, in tep (ii), null coords, and compactifies these in step (iii), because these preserve the causal structure.
%   (2) Max owr, dropping the conformal factor $\Omega^{-2} \neq 0$ does afect the shape of timelike and spacelike geodesics but not of null geodesics

%   More precisley:
%   $\gamma$ is a null geodesic of a metric $g$ iff
%   $\gamma$ is a mull geodesic  of the metric $\Omega^2 g$, Whre $\Omega^2$ is a nowhere vanishing smoth function on the manifold. proof $\rightarrow$ Tutonals.


\printbibliography[keyword={BlackHoles},heading=subbibliography]

\begin{task}{Latex}{}
Cambiar el titulo de referencias a referencias del capitulo    
\end{task}
\nocite{*}