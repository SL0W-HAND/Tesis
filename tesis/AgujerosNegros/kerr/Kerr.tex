
\section{Algoritmo de Newman-Janis y agujeros negros rotantes}

El algoritmo de Newman-Janis es una técnica en relatividad general que permite generar soluciones exactas a las ecuaciones de Einstein. En 1965, Ezra T. Newman y Alfred I. Janis \cite{newman-1965} descubrieron que, mediante una transformación de coordenadas complejas aplicada a la métrica de Schwarzschild, podían obtener la métrica de Kerr, que describe un agujero negro en rotación.

Particularmente nos centraremos en usar el desarrollo hecho por \cite{drake-2000} donde da una versión de este algoritmo donde remueve algunas de las ambigüedades presentes en el trabajo original de \cite{newman-1965}.

El algoritmo se puede resumir en 5 pasos:

\begin{enumerate}[1.]
    \item Escribe un elemento de línea estático y esféricamente simétrico en coordenadas nulas avanzadas $\{u, r, \theta, \phi\}$:

          \begin{equation}
              ds^2 = e^{2 \Phi(r)} du^2 + e^{\Phi(r) + \lambda(r)} du dr - r^2 (d\theta^2 + \sin^2 \theta d\phi^2)
          \end{equation}

    \item Expresar la forma contravariante de la métrica en términos de un tetrad nulo:

          \begin{equation}
              g^{\mu \nu} = l^\mu n^\nu + l^\nu n^\mu - m^\mu \bar{m}^\nu - m^\nu \bar{m}^\mu
              \label{MetricaTetrada}
          \end{equation}

          donde

          \begin{equation}
              l_\mu l^\mu = m_\mu m^\mu = n_\mu n^\mu = 0, \quad l_\mu n^\mu = -m_\mu \bar{m}^\mu = 1, \quad l_\mu m^\mu = n_\mu m^\mu = 0
          \end{equation}

          Donde los tetrads nulos tendrán la forma inicial:

          \begin{equation}
              \begin{aligned}
                  l^\mu & = \delta_1^\mu                                                                          \\
                  n^\mu & = e^{-\lambda(r) - \Phi(r)} \delta_0^\mu - \frac{1}{2} e^{-2 \lambda(r)} \delta_1^\mu   \\
                  m^\mu & = \frac{1}{\sqrt{2} r} \left( \delta_2^\mu + \frac{i}{\sin \theta} \delta_3^\mu \right)
              \end{aligned}
          \end{equation}

          En este punto es conveniente usar la notación de tetradas:

          \begin{equation}
              Z_a^\mu = (l^\mu, n^\mu, m^\mu, \bar{m}^\mu), \quad a = 1, 2, 3, 4
          \end{equation}

    \item El siguiente paso es hacer una transformación compleja directa que extiende las coordenadas $x^\rho$ a un nuevo conjunto de coordenadas complejas $\tilde{x}^\rho$:

          \begin{equation}
              x^\rho \rightarrow \tilde{x}^\rho = x^\rho + i y^\rho(x^\sigma)
          \end{equation}

          donde $y^\rho(x^\sigma)$ son funciones analíticas de las coordenadas reales $x^\sigma$, y simultáneamente dejamos que los vectores tetrad nulos $Z_a^\mu$ sufran una transformación:

          \begin{equation}
              Z_a^\mu(x^\rho) \rightarrow \tilde{Z}_a^\mu(\tilde{x}^\rho, \overline{\tilde{x}}^\rho)
          \end{equation}

          Finalmente requerimos como condición que la transformación recupere el tetrad y la métrica antiguos cuando $\tilde{x}^\rho = \overline{\tilde{x}}^\rho$. En resumen, el efecto de esta "transformación tilde" es crear una nueva métrica cuyos componentes (reales) son funciones de variables complejas:

          \begin{equation}
              g_{\mu \nu} \rightarrow \tilde{g}_{\mu \nu}: \tilde{x} \times \tilde{x} \mapsto \mathbb{R}
          \end{equation}

          mientras que se debe de cumplir también:

          \begin{equation}
              \left. \tilde{Z}_a^\mu(\tilde{x}^\rho, \overline{\tilde{x}}^\rho) \right|_{\overline{x} = \tilde{x}} = Z_a^\mu(x^\rho)
          \end{equation}

          La transformación tilde claramente no es única, ya que existen muchas opciones diferentes para los coeficientes de los vectores tetrad nulos que satisfacen las condiciones. En el paper original \cite{newman-1965} se elige la transformación tilde como:

          \begin{equation}
              \tilde{x}^\rho = x^\rho + i a \cos x^2 (\delta_0^\rho - \delta_1^\rho) = x^\rho + i a \cos \theta (\delta_0^\rho - \delta_1^\rho)
          \end{equation}

          En este caso, la transformación tilde se escribe para cada coordenada $\rho$ como:

          \begin{itemize}
              \item Para $\rho = 0$ (coordenada $u$):
                    \begin{equation}
                        \tilde{u} = u + i a \cos \theta
                    \end{equation}
                    porque $\delta_0^0 = 1$ y $\delta_1^0 = 0$.

              \item Para $\rho = 1$ (coordenada $r$):
                    \begin{equation}
                        \tilde{r} = r - i a \cos \theta
                    \end{equation}
                    ya que $\delta_0^1 = 0$ y $\delta_1^1 = 1$.

              \item Para $\rho = 2, 3$ (coordenadas $\theta$ y $\phi$):
                    \begin{equation}
                        \tilde{\theta} = \theta, \quad \tilde{\phi} = \phi
                    \end{equation}
                    pues $\delta_0^\rho - \delta_1^\rho = 0$ para $\rho = 2, 3$.
          \end{itemize}

          Después de aplicar la transformación, las coordenadas $\tilde{x}^\rho$ son complejas. El radio $r$ se reemplaza por $\tilde{r} = r - i a \cos \theta$, lo que permitirá que las funciones métricas (por ejemplo, $e^{2 \Phi}$) pasen a tener una dependencia combinada en $r$ y $\theta$.

          \begin{itemize}
              \item $u \rightarrow \tilde{u} = u + ia \cos \theta$ (coordenada tipo luz)
              \item $r \rightarrow \tilde{r} = r - i a \cos \theta$
              \item $\theta \rightarrow \tilde{\theta} = \theta$
              \item $\phi \rightarrow \tilde{\phi} = \phi$
          \end{itemize}

          \begin{equation}
              \begin{aligned}
                  l^\mu & \rightarrow \tilde{l}^\mu = \delta_1^\mu                                                                                                                                                                  \\
                  n^\mu & \rightarrow \tilde{n}^\mu = e^{-\lambda(\tilde{r}, \overline{\tilde{r}}) - \phi(\tilde{r}, \overline{\tilde{r}})} \delta_0^\mu - \frac{1}{2} e^{-2 \lambda(\tilde{r}, \overline{\tilde{r}})} \delta_1^\mu \\
                  m^\mu & \rightarrow \tilde{m}^\mu = \frac{1}{\sqrt{2} \overline{\tilde{r}}} \left( \delta_2^\mu + \frac{i}{\sin \tilde{\theta}} \delta_3^\mu \right)
              \end{aligned}
          \end{equation}

          En este paso se extiende la variable radial $r$ a una variable compleja. La idea es reemplazar $r$ por $\tilde{r}$ (y análogamente para $\theta$ si fuera necesario) de modo que las funciones métricas pasen a depender de dos variables: $r$ y $\theta$. De esta forma, se escriben las funciones como $\phi(\tilde{r}, \overline{\tilde{r}})$ y $\lambda(\tilde{r}, \overline{\tilde{r}})$.

          \begin{itemize}
              \item $\tilde{r} \rightarrow r$,
              \item $\overline{\tilde{r}} \rightarrow r$,
              \item $\tilde{\theta} \rightarrow \theta$,
              \item $\tilde{r} + \overline{\tilde{r}} = 2r$,
              \item $\tilde{r} - \overline{\tilde{r}} = 2ia \cos \theta$.
          \end{itemize}
          \begin{note}[Problema de la Complejidad]
              Una métrica compleja no tiene sentido físico en relatividad general, ya que el espacio-tiempo es real. Por lo tanto, se requiere eliminar las partes imaginarias introducidas por la complejificación. Esto se logra mediante la condición de realidad.

              \textbf{Mecanismo de la Condición de Realidad}

              La condición de realidad opera en dos niveles:
              \begin{enumerate}[a.]
                  \item \textbf{Sustitución de $\tilde{r}$ y $\overline{\tilde{r}}$}

                        Se reemplazan las combinaciones de $\tilde{r}$ (complejo) y su conjugado $\overline{\tilde{r}} = r - ia \cos \theta$ por expresiones reales que dependen de $r$ y $\theta$. Por ejemplo:

                        $$
                            \tilde{r} + \overline{\tilde{r}} = 2r, \quad \tilde{r} \overline{\tilde{r}} = r^2 + a^2 \cos^2 \theta.
                        $$

                        Ejemplo:
                        Si una función compleja es $f(\tilde{r}) = \frac{1}{\tilde{r}}$, la condición de realidad la convierte en:

                        $$
                            f(r, \theta) = \frac{1}{r^2 + a^2 \cos^2 \theta} \cdot (r - ia \cos \theta)
                        $$

                        pero solo se conserva la parte real relevante para la métrica.
                  \item  \textbf{Simetrización de Funciones}

                        Las funciones que dependían originalmente de $r$ en la métrica estática se redefinen como funciones simétricas en $\tilde{r}$ y $\overline{\tilde{r}}$. Por ejemplo:
                        - En la métrica de Schwarzschild, $1 - \frac{2M}{r}$ se convierte en $1 - \frac{2M}{\tilde{r}} \rightarrow 1 - \frac{2Mr}{r^2 + a^2 \cos^2 \theta}$.

              \end{enumerate}



              \textbf{Aplicación en las Tetradas}

              Las tetradas nulas $l^\mu, n^\mu, m^\mu$ también se afectan por la complejificación. Para garantizar su realidad:

              \begin{enumerate}[a.]
                  \item \textbf{Tetrada $m^\mu$}

                        La parte angular $m^\mu$ adquiere un término adicional $ia \sin \theta (\delta_0^\mu - \delta_1^\mu)$, que cancela las contribuciones imaginarias del denominador complejo $\tilde{r} = r + ia \cos \theta$.
                        Ejemplo:

                        $$
                            m^\mu = \frac{1}{\sqrt{2}(r + ia \cos \theta)} \left( ia \sin \theta (\delta_0^\mu - \delta_1^\mu) + \delta_2^\mu + \frac{i}{\sin \theta} \delta_3^\mu \right)
                        $$

                  \item \textbf{Tetradas $l^\mu$ y $n^\mu$}

                        Se mantienen reales al reemplazar $\tilde{r} \rightarrow r$ en sus componentes radiales, pero incorporando $a$ en las funciones $\phi(r, \theta)$ y $\lambda(r, \theta)$.
              \end{enumerate}
          \end{note}




    \item Se obtiene una nueva métrica al realizar una transformación de coordenadas complejas en los vectores tetrad nulos $\tilde{Z}_a^\mu$. Los vectores tetrad nulos se transforman de la manera habitual:

          \begin{equation}
              Z_a^\mu = \tilde{Z}_a^\nu \frac{\partial x^\mu}{\partial \tilde{x}^\nu}
          \end{equation}

          mediante esta transformación, de forma genérica podemos efectuar la de las tetradas $m^\mu = \tilde{m}^\nu \frac{\partial x^\mu}{\partial \tilde{x}^\nu}$ y $l^\mu = \tilde{l}^\nu \frac{\partial x^\mu}{\partial \tilde{x}^\nu}$:
          \begin{note}
            Las derivadas parciales $\pd{x^\mu}{\tilde{x}^\nu}$ las tomamos a partir de las transformaciones inversas:
            $$\begin{aligned} u & =\tilde{u}-i a \cos \tilde{\theta} \\ r & =\tilde{r}+i a \cos \tilde{\theta} \\ \theta & =\tilde{\theta} \\ \phi & =\tilde{\phi}\end{aligned}$$
            y la matriz jacobiana de la transformación es:
            \begin{equation}
               \left(\pd{x^\mu}{\tilde{x}^\nu}\right) =\left(\begin{array}{cccc}
                1 & 0 & +i a \sin \theta & 0\\
                0 & 1 & -i a \sin \theta & 0\\
                0 & 0 & 1 & 0\\
                0 & 0 & 0& 1
                \end{array}\right)
                \end{equation}
                Lo cual nos da que:
                \begin{align*}
                \pd{x^\mu}{\tilde{x}^0}&= \delta^\mu_0\\
                \pd{x^\mu}{\tilde{x}^1}&= \delta^\mu_1\\
                \pd{x^\mu}{\tilde{x}^2}&= ia (\delta^\mu_0 - \delta^\mu_1)\sin \theta + \delta^\mu_2\\
                \pd{x^\mu}{\tilde{x}^3}&=\delta^\mu_3
                \end{align*}
          \end{note}
          \begin{equation}
              \begin{aligned}
                  l^\mu & = \tilde{l}^\nu \frac{\partial x^\mu}{\partial \tilde{x}^\nu}                                                                                                                                                                           \\
                        & = \tilde{l}^0 \frac{\partial x^\mu}{\partial \tilde{x}^0} + \tilde{l}^1 \frac{\partial x^\mu}{\partial \tilde{x}^1} + \tilde{l}^2 \frac{\partial x^\mu}{\partial \tilde{x}^2} + \tilde{l}^3 \frac{\partial x^\mu}{\partial \tilde{x}^3} \\
                        & = \delta_1^0 \frac{\partial x^\mu}{\partial \tilde{x}^0} + \delta_1^1 \frac{\partial x^\mu}{\partial \tilde{x}^1} + \delta_1^2 \frac{\partial x^\mu}{\partial \tilde{x}^2} + \delta_1^3 \frac{\partial x^\mu}{\partial \tilde{x}^3}     \\
                        & = \delta_1^1 \frac{\partial x^\mu}{\partial \tilde{x}^1} = \delta_1^\mu
              \end{aligned}
          \end{equation}

          el caso $m^\mu = \tilde{m}^\nu \frac{\partial x^\mu}{\partial \tilde{x}^\nu}$:

          \begin{equation}
              \begin{aligned}
                  m^\mu & = \tilde{m}^\nu \frac{\partial x^\mu}{\partial \tilde{x}^\nu}                                                                                                                                                                                                                           \\
                        & = \tilde{m}^0 \frac{\partial x^\mu}{\partial \tilde{x}^0} + \tilde{m}^1 \frac{\partial x^\mu}{\partial \tilde{x}^1} + \tilde{m}^2 \frac{\partial x^\mu}{\partial \tilde{x}^2} + \tilde{m}^3 \frac{\partial x^\mu}{\partial \tilde{x}^3}                                                 \\
                        & = 0 * \frac{\partial x^\mu}{\partial \tilde{x}^0} + 0 * \frac{\partial x^\mu}{\partial \tilde{x}^1} + \frac{1}{\sqrt{2} \tilde{r}} \frac{\partial x^\mu}{\partial \tilde{x}^2} + \frac{1}{\sqrt{2} \tilde{r}} \frac{i}{\sin \tilde{\theta}} \frac{\partial x^\mu}{\partial \tilde{x}^3} \\
                        & = \frac{1}{\sqrt{2}(r + ia \cos \theta)} \left( ia \sin \theta (\delta_0^\mu - \delta_1^\mu) + \delta_2^\mu + \frac{i}{\sin \theta} \delta_3^\mu \right)
              \end{aligned}
          \end{equation}

          dado que la tetrada $n^\mu$ es particular para cada métrica semilla que se use, el cálculo de esta debe de hacerse de forma particular. Al efectuar la transformación del tetrad, se obtiene:

          \begin{equation}
              \begin{aligned}
                  l^\mu & = \delta_1^\mu     \\
                  m^\mu & = \frac{1}{\sqrt{2}(r + ia \cos \theta)} \left[ ia \sin \theta (\delta_0^\mu - \delta_1^\mu) + \delta_2^\mu + \frac{i}{\sin \theta} \delta_3^\mu \right]
              \end{aligned}
          \end{equation}

    \item \begin{task}
        Convertir la métrica obtenida a coordenadas de Boyer-Lindquist, que son las coordenadas más comunes para describir agujeros negros rotantes. 
    \end{task}
\end{enumerate}

\subsection{Aplicación a la métrica de Schwarzschild}
La métrica de Schwarzschild en coordenadas avanzadas de Eddington-Finkelstein es:
\begin{equation}
    ds^2=\left(1-\frac{2 m}{r}\right) d u^2+2 d u d r-r^2\left(d \theta^2+\sin ^2 \theta d \phi^2\right) .
\end{equation}

El "seed metric" propuesto es

\begin{equation}
    d s^2=e^{2 \Phi(r)} d u^2+e^{\Phi(r)+\lambda(r)} d u d r-r^2\left(d \theta^2+\sin ^2 \theta d \phi^2\right)
\end{equation}

Comparando término a término con la métrica obtenida:
Tenemos

\begin{equation}
    e^{2 \Phi(r)}=1-\frac{2 m}{r}
\end{equation}

lo que implica

\begin{equation}
    \Phi(r)=\frac{1}{2} \ln \left(1-\frac{2 M}{r}\right) .
\end{equation}

En la métrica obtenida aparece el coeficiente +2 , por lo que se identifica

\begin{equation}
    e^{\Phi(r)+\lambda(r)}=2
\end{equation}

Despejando para $\lambda(r)$,

\begin{equation}
    \begin{aligned}
        \lambda(r)= & \ln 2-\Phi(r)=\ln 2-\frac{1}{2} \ln \left(1-\frac{2 M}{r}\right)   \\
        =           & \frac{1}{2}\ln 4 -\frac{1}{2} \ln \left(1-\frac{2 M}{r}\right)     \\
        =           & \frac{1}{2} \left(\ln 4 -  \ln \left(1-\frac{2 M}{r}\right)\right) \\
        =           & \frac{1}{2} \ln \left(\frac{4}{1-\frac{2 M}{r}}\right)             \\
        =           & \frac{1}{2} \ln \left(\frac{4 r}{r-2 M}\right)
    \end{aligned}
\end{equation}
como lo comentamos en el caso genérico solo hace falta preocuparse por el caso de la tetrada $n^\mu$ que es particular para cada métrica semilla, en este caso se obtiene
\begin{equation}
    n^\mu=e^{-\lambda(r)-\Phi(r)} \delta_0^\mu-\frac{1}{2} e^{-2 \lambda(r)} \delta_1^\mu
\end{equation}
Para el caso particular de la métrica de Schwarzschild tenemos
\begin{equation}
   \begin{aligned}
    n^\mu&=e^{- \frac{1}{2} \ln \left(\frac{4 r}{r-2 m}\right) -\frac{1}{2} \ln \left(1-\frac{2 m}{r}\right)  }\delta_0^\mu-\frac{1}{2} e^{-2 \frac{1}{2} \ln \left(\frac{4 r}{r-2 m}\right)} \delta_1^\mu \\
    &=e^{\ln \left(\frac{4 r}{r-2 m}\right)^{-1/2}}e^{\ln \left(1-\frac{2 m}{r}\right)^{-1/2}} \delta_0^\mu-\frac{1}{2} e^{\ln \left(\frac{4 r}{r-2 m}\right)^{-1}} \delta_1^\mu \\
    &=\left(\frac{4 r}{r-2 m}\right)^{-1/2}\left(1-\frac{2 m}{r}\right)^{-1/2} \delta_0^\mu-\frac{1}{2} \left(\frac{4 r}{r-2 m}\right)^{-1} \delta_1^\mu \\ 
    &=\left(\frac{4 r}{r-2 m}\frac{r - 2 m}{r}\right)^{-1/2}\delta_0^\mu-\frac{1}{2} \left(\frac{r-2 m}{4r}\right)\delta_1^\mu \\
    &= \frac{1}{2}\delta_0^\mu-\frac{1}{8} \left(1 - \frac{2 m}{r}\right)\delta_1^\mu \\
   \end{aligned}
\end{equation}
procederemos a la hacer la transformación compleja de la tetrada $n^\mu \to \tilde{n}^\mu$:
\begin{equation}
    \tilde{n}^\mu=\frac{1}{2}\delta_0^\mu-\frac{1}{8} \left(1 - \frac{2 m}{\tilde{\bar{r} }}\right)\delta_1^\mu 
\end{equation}
Para sacar la nueva tetrada $n^\mu$ se hace el siguiente procedimiento:
\begin{equation}
\begin{aligned}
    n^\mu&= \tilde{n}^\mu \pd{x^\mu}{\tilde{x}^\nu} \\
&= \tilde{n}^0 \pd{x^\mu}{\tilde{x}^0} + \tilde{n}^1 \pd{x^\mu}{\tilde{x}^1} + \tilde{n}^2 \pd{x^\mu}{\tilde{x}^2} + \tilde{n}^3 \pd{x^\mu}{\tilde{x}^3} \\
&= \tilde{n}^0 \pd{x^\mu}{\tilde{x}^0} + \tilde{n}^1 \pd{x^\mu}{\tilde{x}^1} +0 \cdot \pd{x^\mu}{\tilde{x}^2} + 0 \cdot \pd{x^\mu}{\tilde{x}^3} \\
& = \frac{1}{2}\delta^{\mu}_0 -\frac{1}{8} \left(1 - \frac{2 m\tilde{r}}{\tilde{\bar{r}}\tilde{r}}\right)\delta_1^\mu \\
& = \frac{1}{2}\delta^{\mu}_0 -\frac{1}{8} \left(1 - \frac{2 m(r-ia\cos\theta)}{r^2+a^2 \cos^2 \theta }\right)\delta_1^\mu \\
\end{aligned}    
\end{equation}
de forma que todas las tetradas quedan como:
\begin{equation}
    \begin{aligned}
        l^\mu & = \delta_1^\mu     \\
        n^\mu & = \frac{1}{2}\delta^{\mu}_0 -\frac{1}{8} \left(1 - \frac{2 m(r-ia\cos\theta)}{r^2+a^2 \cos^2 \theta }\right)\delta_1^\mu  \\
        m^\mu & = \frac{1}{\sqrt{2}(r + ia \cos \theta)} \left[ ia \sin \theta (\delta_0^\mu - \delta_1^\mu) + \delta_2^\mu + \frac{i}{\sin \theta} \delta_3^\mu \right]
    \end{aligned}
\end{equation}
Y usando la transformación (\ref{MetricaTetrada}) obtenemos una forma de la métrica de Kerr
\begin{equation}
   g^{\mu\nu} =  \left(\begin{matrix}
- \frac{a^{2} \sin^{2}{\left(\theta \right)}}{a^{2} \cos^{2}{\left(\theta \right)} + r^{2}} & \frac{0.5 \left(a^{2} \sin^{2}{\left(\theta \right)} + a^{2} + r^{2}\right)}{a^{2} \cos^{2}{\left(\theta \right)} + r^{2}} & 0 & - \frac{a}{a^{2} \cos^{2}{\left(\theta \right)} + r^{2}} \\
\frac{0.5 \left(a^{2} \sin^{2}{\left(\theta \right)} + a^{2} + r^{2}\right)}{a^{2} \cos^{2}{\left(\theta \right)} + r^{2}} & \frac{- 3 a^{2} \sin^{2}{\left(\theta \right)} - a^{2} - 2 i a m \cos{\left(\theta \right)} + 2 m r - r^{2}}{4 \left(a^{2} \cos^{2}{\left(\theta \right)} + r^{2}\right)} & 0 & \frac{a}{a^{2} \cos^{2}{\left(\theta \right)} + r^{2}} \\
0 & 0 & - \frac{1}{a^{2} \cos^{2}{\left(\theta \right)} + r^{2}} & 0 \\
- \frac{a}{a^{2} \cos^{2}{\left(\theta \right)} + r^{2}} & \frac{a}{a^{2} \cos^{2}{\left(\theta \right)} + r^{2}} & 0 & - \frac{1}{\left(a^{2} \cos^{2}{\left(\theta \right)} + r^{2}\right) \sin^{2}{\left(\theta \right)}} \\
\end{matrix}\right)
\end{equation}

\begin{task}
    Verificar que la métrica obtenida es equivalente a la métrica de Kerr en coordenadas de Boyer-Lindquist.

\end{task}
La mewtrica en coordenadas de Boyer-Lindquist es:

\begin{equation}
    g_{\mu \nu}=\left(\begin{array}{cccc}
-\left(1-\frac{2 M r}{\Sigma}\right) c^2 & 0 & 0 & -\frac{2 M a r \sin ^2 \theta}{\Sigma} c \\
0 & \frac{\Sigma}{\Delta} & 0 & 0 \\
0 & 0 & \Sigma & 0 \\
-\frac{2 M a r \sin ^2 \theta}{\Sigma} c & 0 & 0 & \left(r^2+a^2+\frac{2 M r a^2 \sin ^2 \theta}{\Sigma}\right) \sin ^2 \theta
\end{array}\right)
    \label{eq:MetricaKerrBL}
\end{equation}
Donde funciones $\Sigma$ y $\Delta$ son:
- $\Sigma=r^2+a^2 \cos ^2 \theta$
- $\Delta=r^2-2 M r+a^2$

\section{Análisis de la métrica de Kerr}
\subsection{Símbolos de Christoffel}
Se uso el algoritmo que aparece en los anexos para calcular los símbolos de Christoffel no cero de la métrica de Kerr
