\section{Solución de Kerr}
En el caso de un agujero negro en rotación (métrica de Kerr), el espacio-tiempo fuera de él es estacionario, y se puede elegir un marco de referencia independiente del tiempo que asintóticamente tiende al marco de Lorentz en el infinito. Las coordenadas propuestas por Boyer y Lindquist (1967) representan dicho marco de referencia. La métrica de Kerr en estas coordenadas es:

$$
    d s^2=-\left(1-\frac{2 M r}{\Sigma}\right) d t^2-\frac{4 M r a \sin ^2 \theta}{\Sigma} d t d \phi+\frac{\Sigma}{\Delta} d r^2+\Sigma d \theta^2+\frac{A \sin ^2 \theta}{\Sigma} d \phi^2,(3.2 .1)
$$

donde


\begin{align}
     & \Sigma \equiv r^2+a^2 \cos ^2 \theta, \quad \Delta \equiv r^2-2 M r+a^2 \\
     & A=\left(r^2+a^2\right)^2-a^2 \Delta \sin ^2 \theta
\end{align}

\subsection{deducion}
This video investigates some of the underlying conditions required to develop a suitable metric to describe the space-time surrounding a uniformly rotating mass. It particularly focusses on the slow speed and weak field case before looking at the frame dragging effect that results from a rotating mass.
This video is the first in a series on the Kerr metric.

- Given the axial symmetry of the situation we start with the coordinates,

$$
    x^\mu=\left(x^0, x^1, x^2, x^3\right)=(c t, r, \theta, \varphi)
$$

- A stationary space-time is one for which all the components of $g_{\mu v}$ are independent of $t$. This means the space-time outside the mass does not chiange over time.

$$
    \frac{\partial g_{\mu v}}{\partial x^0}=\frac{\partial g_{\mu v}}{\partial t}=0
$$

- An axially symmetric space-time is one for which all the components of $g_{\mu v}$ are independent of the angle involved which in this case is $\varphi$. This means the space-time outside the mass does not change when moving in the $\varphi$ direction.

$$
    \frac{\partial g_{\mu v}}{\partial x^3}=\frac{\partial g_{\mu v}}{\partial \varphi}=0
$$
- Together, this means we have two Killing vectors, one for each of the two directions,

$$
    \frac{\partial}{\partial t} \text { and } \frac{\partial}{\partial \varphi}
$$

- Our metric has the initial form,

$$
    g_{\mu v}(r, \theta)=\left[\begin{array}{llll}
            g_{00} & g_{01} & g_{02} & g_{03} \\
            g_{10} & g_{11} & g_{12} & g_{13} \\
            g_{20} & g_{21} & g_{22} & g_{23} \\
            g_{30} & g_{31} & g_{32} & g_{33}
        \end{array}\right]=g_{v \mu}
$$

- The line element or interval is $\mathrm{ds}^2=g_{\mu v} \mathrm{dx}^\mu \mathrm{dx} \mathrm{x}^v$
- This space-time is also invariant under the simultaneous transformation of $t \rightarrow-t$ and $\varphi \rightarrow-\varphi$. This is due entirely to the axial symmetry and the stationarity in the $\varphi$ and t directions.


\begin{align}
     & (-c d t)(-d \varphi)=(c d t)(d \varphi)=c d t d \varphi \\
     & \Rightarrow g_{03}=g_{30} \neq 0
\end{align}


- No such symmetries exist for combinations of pairs of coordinates other than these two. So that means we can rule out some other cross terms in the metric.


\begin{align}
     & (-c d t)(-c d t)=c^2 d t^2 \quad \text { but }-c d t d r \neq c d t d r                                           \\
     & d s^2=g_{00} c^2 d t^2+g_{01} c d t d r+\ldots \ldots \ldots                                                      \\
     & d s^2=g_{00} c^2 d t^2-g_{01} c d t d r+\ldots \ldots \ldots                                                      \\
     & -c d t d \theta \neq c d t d \theta,-d \theta d \varphi \neq d \theta d \varphi,-d r d \varphi \neq d r d \varphi
\end{align}


- So we have, $g_{01}=g_{02}=g_{23}=g_{13}=0$
- Our line element is now,


\begin{align}
    d s^2= & g_{00} c^2 d t^2+2 g_{03} d t d \varphi+g_{33} d \varphi^2         \\
           & +\left(g_{11} d r^2+2 g_{12} d r d \theta+g_{22} d \theta^2\right)
\end{align}


- But what about $g_{12}$ ?

- The metric coefficients $\mathrm{g}_{\mu \nu}$ are all functions of r and $\theta$ and the part of the line element in parentheses is the line element of a separate two dimensional submanifold.


\begin{align}
    d s^2= & g_{00} c^2 d t^2+2 g_{03} c d t d \varphi+g_{33} d \varphi^2                  \\
           & +\left(g_{11} d r^2+2 g_{12} d r d \theta+g_{22} d \theta^2\right)            \\
    =      & g_{00} c^2 d t^2+2 g_{03} c d t d \varphi+g_{33} d \varphi^2+g_{\alpha \beta}
\end{align}


- Now, any two dimensional pseudo-Riemannian manifold is conformally flat, which means there exists some $\Omega$, which is a function of the coordinates, such that $\mathrm{g}_{\alpha \beta}$ can be expressed as a flat space metric in the form,

$$
    g_{\alpha \beta}=\Omega^2\left(x^i\right) \eta_{\alpha \beta} \text { where } \quad \eta_{\alpha \beta}=\left[\begin{array}{ll}
            1 & 0 \\
            0 & 1
        \end{array}\right]
$$

- This has reduced $\mathrm{g}_{\alpha \beta}$ to a diagonal form in which $\mathrm{g}_{12}=0$.


- Our new line element is,

$$
    d s^2=g_{t t} c^2 d t^2+g_{t r} d r^2+g_{\theta \theta} d \theta^2+2 g_{t \varphi} c d t d \varphi+g_{\theta \varphi \varphi} d \varphi^2
$$

- Whatever the final form of this metric it must reduce to the Schwarzschild metric when the rotation disappears.
- The Schwarzschild line element is,

$$
    d s_{S c h}^2=-\left(1-\frac{2 G M}{c^2 r}\right) c^2 d t^2+\left(1-\frac{2 G M}{c^2 r}\right)^{-1} d r^2+r^2\left(d \theta^2+\sin ^2 \theta d \varphi^2\right)
$$

- In the case of a slowly rotating mass in the weak field approximation, we could imagine a correction to the Schwarzschild metric to give a new line element to describe this case, in the form,

$$
    d s^2=d s_{S c h}^2-\frac{4 G M a c}{c^3 r^2} \sin ^2 \theta r d \varphi c d t=d s_{S c h}^2-\frac{4 G J}{c^3 r^2} \sin ^2 \theta r d \varphi c d t
$$

- The last term is the Lense-Thirring one which describes the dragging of inertial frames and is valid to first order only in the angular momentum, J (= Mac).
- So our metric in this case is,


\begin{align}
    d s^2=- & \left(1-\frac{2 G M}{c^2 r}\right) c^2 d t^2+\left(1-\frac{2 G M}{c^2 r}\right)^{-1} d r^2+r^2\left(d \theta^2+\sin ^2 \theta d \varphi^2\right) \\
            & -\frac{4 G J}{c^3 r} \sin ^2 \theta d \varphi c d t
\end{align}


- From Newtonian theory a body of mass $M$ and moment of inertia I rotating with angular speed $\omega$ has angular momentum, $\mathrm{J} \sim \mathrm{I} \omega \sim$ $\operatorname{Mr}^2 \omega \sim \operatorname{Mr} v$, where $v$ is the speed of rotation.
- So, $\frac{G J}{c^3 r^2} \sim\left(\frac{G M}{c^2 r}\right)\left(\frac{v}{c}\right)$ where the last factor shows the dependence of the curvature of the space-time on the mass and rotation speed of the body.
- Note that when $\frac{-2}{c^2 r}$ Para salir de la pantalla completa, put ak field case so that,

$$
    \left(1-\frac{2 G M}{c^2 r}\right)^{-1}=\frac{1}{1-\frac{2 G M}{c^2 r}} \approx 1-\frac{2 G M}{c^2 r}
$$

- This can happen when $M$ is small or $r$ is large or both conditions are true.
- This leads to the following form of the weak field approximation for a slowly rotating mass,


\begin{align}
    d s^2 \approx- & \left(1-\frac{2 G M}{c^2 r}\right) c^2 d t^2+\left(1-\frac{2 G M}{c^2 r}\right) d r^2+r^2\left(d \theta^2+\sin ^2 \theta d \varphi^2\right) \\
                   & -\frac{4 G J}{c^3 r} \sin ^2 \theta d \varphi c d t
\end{align}

- We know this metric has two Killing vectors such that along geodesics followed by a given particle we have,

$$
    \frac{d}{d \lambda}(\underset{\sim}{\underset{\sim}{x}} \cdot \underset{\sim}{u})=0
$$

- One of the Killing vectors is,

$$
    \underset{\sim}{K}=K^\alpha{\underset{\sim}{\alpha}}_\alpha=(0,0,0,1) \frac{\partial}{\partial \varphi}
$$

- The four momentum of the particle based on its four velocity is,

$$
    \underset{\sim}{p}=m_0 \underset{\sim}{u}=\left(p^t, p^r, p^\theta, p^{\varphi}\right)
$$
- In this geometry the angular momentum component of the four momentum of the particle is conserved along its geodesics.


\begin{align}
     & \frac{d}{d \lambda}\left(\underset{\sim}{K} \cdot m_0 \underset{\sim}{u}\right)=\frac{d}{d \lambda}(\underset{\sim}{K} \cdot \underset{\sim}{p}) \\
     & =\frac{d}{d \lambda}\left((0,0,0,1) \frac{\partial}{\partial \varphi}-\left(p^t, p^r, p^\theta, p^{\varphi}\right)\right)                        \\
     & =\frac{d}{d \lambda}\left(g_{\varphi \varphi} 1 p^{\varphi}\right)                                                                               \\
     & =\frac{d}{d \lambda}\left(g_{\varphi \varphi} p^{\varphi}\right)                                                                                 \\
     & =\frac{d}{d \lambda}\left(p_{\varphi}\right)                                                                                                     \\
     & =0                                                                                                                                               \\
     & \Rightarrow p_{\varphi}=\text { constant }=\mathrm{L}
\end{align}

- For this geometry we have two Killing vectors and a metric of the form,

$$
    d s^2=g_{t t} c^2 d t^2+g_{t r} d r^2+g_{\theta \theta} d \theta^2+2 g_{t \varphi} c d t d \varphi+g_{\varphi \varphi} d \varphi^2
$$

- The contravariant time and azimuthal components of the four momentum are,


\begin{align}
     & \mathrm{p}^{\varphi}=g^{\varphi \alpha} p_\alpha=g^{\varphi t} p_t+g^{\varphi \varphi} p_{\varphi} \\
     & \mathrm{p}^t=g^{t \alpha} p_\alpha=g^{t t} p_t+g^{t \varphi} p_{\varphi}
\end{align}


- Now lets say we release our particle, with mass $m_0$, from rest and far from the source mass $M$ so that it initially has, $p_{\varphi}=0$ (along its geodesic). The components of its 4-momentum that are constant along the geodesics its moves along are,

$$
    \mathrm{p}^t=m_0 u^0=m_0 c \frac{d t}{d \tau} \quad \text { and } \quad \mathrm{p}^{\varphi}=m_0 u^{\varphi}=m_0 \frac{d \varphi}{d \tau}
$$

- These two quantities allow us to form the result,

$$
    \frac{d \varphi}{c d t}=\frac{d \varphi}{d(c t)}=m_0 \frac{d \varphi}{d \tau} \frac{d \tau}{m_0 c d t}=\frac{\mathrm{p}^{\varphi}}{\mathrm{p}^t}=\frac{g^{\varphi t}}{g^{t t}}=\omega(r, \theta)
$$

- Here, $\omega$ is the coordinate angular velocity that is acquired by the particle as it moves along its geodesic (which means it is freely falling), even though it started from rest far from the source mass $M$.
- The fact this particle is following a geodesic means that the space-time curvature around the source mass must be curved such that the particle acquires an angular velocity in the same direction as the source mass.
- This effect is called the dragging of inertial frames.
- In the case of photons we have in some inertial frame a 4-velocity given by,

$$
    \underset{\sim}{u}=\frac{d x^\mu}{d \lambda}{\underset{\sim}{\mu}}^e \quad \text { and } \quad \underset{\sim}{u} \cdot \underset{\sim}{u}=0
$$

- The 4-momentum of this photon in some arbitrary inertial frame is,

$$
    \underset{\sim}{p}=\alpha \underset{\sim}{u}=\alpha\left(\frac{c d t}{d \lambda}, \frac{d r}{d \lambda}, \frac{d \theta}{d \lambda}, \frac{d \varphi}{d \lambda}\right)=\left(p^t, p^r, p^\theta, p^{\varphi}\right)
$$

- When we again release a photon in the direction of the source mass with zero initial angular momentum component, we still get,

$$
    \frac{d \varphi}{c d t}=\frac{d \varphi}{d(c t)}=\alpha \frac{d \varphi}{d \lambda} \frac{d \lambda}{\alpha c d t}=\frac{\mathrm{p}^{\varphi}}{\mathrm{p}^t}=\frac{g^{\varphi t}}{g^{t t}}=\omega(r, \theta)
$$

- Since photons follow the curvature of space-time they are a way to probe that space-time and to determine the effect the given rotating object with mass M has on space-time.
- Note that the direction of the angular velocity acquired by the particle, whether it be with mass or a photon, is in the same direction as the rotation of the source mass $M$,

$$
    \frac{d \varphi}{d t}=c \omega(r, \theta)
$$
Kerr metric - 2

This video looks at one method for arriving at the form of the Kerr metric by first starting with the slow rotation in a weak case. It goes on to use oblate spheroidal coordinates and to make use of various approximations including the Schwarzschild case to develop a suitable metric. The final form produced, known as the Kerr metric, satisfies the empty space field equations, $R_{\alpha \beta}=0$.
- For a slowly rotating mass $M$ in the weak field case the space-time outside this mass can be described by the metric,


\begin{align}
    d s^2=- & \left(1-\frac{2 G M}{c^2 r}\right) c^2 d t^2+\left(1-\frac{2 G M}{c^2 r}\right)^{-1} d r^2+r^2\left(d \theta^2+\sin ^2 \theta d \varphi^2\right) \\
            & -\frac{4 G \dot{j}}{c^3 r} \sin ^2 \theta d \varphi c d t
\end{align}


- When $\frac{2 G M}{c^2 r} \ll 1$ we have the weak field case so that our metric can be further approximated by,


\begin{align}
    d s^2 \approx- & \left(1-\frac{2 G M}{c^2 r}\right) c^2 d t^2+\left(1-\frac{2 G M}{c^2 r}\right) d r^2+r^2\left(d \theta^2+\sin ^2 \theta d \varphi^2\right) \\
                   & -\frac{4 G M a c}{c^3 r} \sin ^2 \theta d \varphi c d t
\end{align}


- Lets continue this idea of approximation.
\begin{equation}
    \begin{aligned}
        d s^2 \approx & -\left(1-\frac{2 G M}{c^2 r}\right) c^2 d t^2+\left(1-\frac{2 G M}{c^2 r}\right) d r^2+r^2\left(d \theta^2+\sin ^2 \theta d \varphi^2\right) \\
                      & -\frac{4 G M a c}{c^3 r} \sin ^2 \theta d \varphi c d t                                                                                      \\
        \approx       & -c^2 d t^2+\left(1-\frac{2 G M}{c^2 r}\right) d r^2+r^2\left(d \theta^2+\sin ^2 \theta d \varphi^2\right)+\frac{2 G M}{c^2 r} c^2 d t^2      \\
                      & -\frac{4 G M a}{c^2 r} \sin ^2 \theta d \varphi c d t+\frac{2 G M}{c^2 r} a^2 \sin ^4 \theta d \varphi^2                                     \\
        =             & -c^2 d t^2+\left(1-\frac{2 G M}{c^2 r}\right) d r^2+r^2\left(d \theta^2+\sin ^2 \theta d \varphi^2\right)                                    \\
                      & +\frac{2 G M}{c^2 r}\left(c^2 d t^2-2 a \sin ^2 \theta d \varphi c d t+a^2 \sin ^4 \theta d \varphi^2\right)                                 \\
        =             & -c^2 d t^2+\left(1-\frac{2 G M}{c^2 r}\right) d r^2+r^2\left(d \theta^2+\sin ^2 \theta d \varphi^2\right)                                    \\
                      & +\frac{2 G M}{c^2 r}\left(c d t-a \sin ^2 \theta d \varphi\right)^2
    \end{aligned}
\end{equation}

- This equation is still valid to first order in $a$.
- We know that rotating bodies bulge in the equatorial plane away from the axis of rotation and become oblate spheroidal in shape.
- In the case of our Sun, its equatorial diameter is roughly 10 km greater that its polar diameter.
- This suggest that we can use spheroidal coordinates or ellipsoidal coordinates as a starting point in our search for the form of the Kerr metric.


\begin{align}
     & x=\sqrt{r^2+a^2} \sin \theta \cos \varphi \\
     & y=\sqrt{r^2+a^2} \sin \theta \sin \varphi \\
     & z=r \cos \theta
\end{align}
- We want to express the Minkowski metric in oblate spheroidal coordinates.

$$
    d s^2=-c^2 d t^2+d x^2+d y^2+d z_t^2
$$



\begin{align}
    d x & =\frac{\partial x}{\partial r} d r+\frac{\partial x}{\partial \theta} d \theta+\frac{\partial x}{\partial \varphi} d \varphi                              \\
        & =\frac{r}{\sqrt{r^2+a^2}} \sin \theta \cos \varphi d r+\sqrt{r^2+a^2} \cos \theta \cos \varphi d \theta-\sqrt{r^2+a^2} \sin \theta \sin \varphi d \varphi \\
    d y & =\frac{\partial y}{\partial r} d r+\frac{\partial y}{\partial \theta} d \theta+\frac{\partial y}{\partial \varphi} d \varphi                              \\
        & =\frac{r}{\sqrt{r^2+a^2}} \sin \theta \sin \varphi d r+\sqrt{r^2+a^2} \cos \theta \sin \varphi d \theta+\sqrt{r^2+a^2} \sin \theta \cos \varphi d \varphi \\
    d z & =\frac{\partial z}{\partial r} d r+\frac{\partial z}{\partial \theta} d \theta                                                                            \\
        & =\cos \theta d r-r \sin \theta d \theta
\end{align}

\begin{equation}
    \begin{aligned}
        = & \left(\frac{a^2}{2\left(a^2+r^2\right)}+\frac{r^2}{a^2+r^2}+\frac{a^2 \cos ^2 \theta}{2\left(a^2+r^2\right)}-\frac{a^2 \sin ^2 \theta}{2\left(a^2+r^2\right)}\right) d r^2+                                                       \\
          & \left(\frac{a^4}{2\left(a^2+r^2\right)}+\frac{3 a^2 r^2}{2\left(a^2+r^2\right)}+\frac{r^4}{a^2+r^2}+\frac{a^4 \cos ^2 \theta}{2\left(a^2+r^2\right)}+\right.                                                                      \\
          & \left.\frac{a^2 r^2 \cos ^2 \theta}{2\left(a^2+r^2\right)}-\frac{a^4 \sin ^2 \theta}{2\left(a^2+r^2\right)}-\frac{a^2 r^2 \sin ^2 \theta}{2\left(a^2+r^2\right)}\right) d \theta^2+\left(\frac{a^4}{2\left(a^2+r^2\right)}\right. \\
          & \frac{a^2 r^2}{a^2+r^2}+\frac{r^4}{2\left(a^2+r^2\right)}-\frac{a^4 \cos ^2 \theta}{2\left(a^2+r^2\right)}-\frac{a^2 r^2 \cos ^2 \theta}{a^2+r^2}-                                                                                \\
          & \left.\frac{r^4 \cos ^2 \theta}{2\left(a^2+r^2\right)}+\frac{a^4 \sin ^2 \theta}{2\left(a^2+r^2\right)}+\frac{a^2 r^2 \sin ^2 \theta}{a^2+r^2}+\frac{r^4 \sin ^2 \theta}{2\left(a^2+r^2\right)}\right) d \varphi^2
    \end{aligned}
\end{equation}

- For the coefficient of $d r^2$ we have,


\begin{align}
    \frac{a^2}{2\left(a^2+r^2\right)}+\frac{r^2}{a^2+r^2}+\frac{a^2 \cos ^2 \theta}{2\left(a^2+r^2\right)}-\frac{a^2 \sin ^2 \theta}{2\left(a^2+r^2\right)}    & = \\
    \frac{a^2}{2\left(a^2+r^2\right)}+\frac{2 r^2}{2\left(a^2+r^2\right)}+\frac{a^2\left(2 \cos ^2 \theta-1\right)}{2\left(a^2+r^2\right)}                     & = \\
    \frac{a^2}{2\left(a^2+r^2\right)}+\frac{2 r^2}{2\left(a^2+r^2\right)}+\frac{2 a^2 \cos ^2 \theta}{2\left(a^2+r^2\right)}-\frac{a^2}{2\left(a^2+r^2\right)} & = \\
    \frac{2 r^2}{2\left(a^2+r^2\right)}+\frac{2 a^2 \cos ^2 \theta}{2\left(a^2+r^2\right)}                                                                     & = \\
    \frac{r^2+a^2 \cos ^2 \theta}{a^2+r^2}                                                                                                                     & =
\end{align}
For$ d \theta^2$

\begin{align}
     & \frac{a^4}{2\left(a^2+r^2\right)}+\frac{3 a^2 r^2}{2\left(a^2+r^2\right)}+\frac{r^4}{a^2+r^2}+\frac{a^4 \cos ^2 \theta}{2\left(a^2+r^2\right)}+                                                    \\
     & \frac{a^2 r^2 \cos ^2 \theta}{2\left(a^2+r^2\right)}-\frac{a^4 \sin ^2 \theta}{2\left(a^2+r^2\right)}-\frac{a^2 r^2 \sin ^2 \theta}{2\left(a^2+r^2\right)}=\frac{a^4}{2\left(a^2+r^2\right)}+      \\
     & \frac{3 a^2 r^2}{2\left(a^2+r^2\right)}+\frac{r^4}{a^2+r^2}+\frac{a^4\left(2 \cos ^2 \theta-1\right)}{2\left(a^2+r^2\right)}+\frac{a^2 r^2\left(2 \cos ^2 \theta-1\right)}{2\left(a^2+r^2\right)}= \\
     & \frac{2 a^2 r^2}{2\left(a^2+r^2\right)}+\frac{r^4}{a^2+r^2}+\frac{2 a^4 \cos ^2 \theta}{2\left(a^2+r^2\right)}+\frac{2 a^2 r^2 \cos ^2 \theta}{2\left(a^2+r^2\right)}=                             \\
     & \frac{a^2 r^2}{\left(a^2+r^2\right)}+\frac{r^4}{a^2+r^2}+\frac{a^4 \cos ^2 \theta}{\left(a^2+r^2\right)}+\frac{a^2 r^2 \cos ^2 \theta}{\left(a^2+r^2\right)}=                                      \\
     & \frac{\left(r^2+a^2 \cos ^2 \theta\right)\left(r^2+a^2\right)}{a^2+r^2}=                                                                                                                           \\
     & r^2+a^2 \cos ^2 \theta=
\end{align}
For $ d \varphi^2$
\begin{equation}
    \begin{aligned}
         & \frac{a^4}{2\left(a^2+r^2\right)}+\frac{a^2 r^2}{a^2+r^2}+\frac{r^4}{2\left(a^2+r^2\right)}-\frac{a^4 \cos ^2 \theta}{2\left(a^2+r^2\right)}-\frac{a^2 r^2 \cos ^2 \theta}{a^2+r^2}        \\
         & -\frac{r^4 \cos ^2 \theta}{2\left(a^2+r^2\right)}+\frac{a^4 \sin ^2 \theta}{2\left(a^2+r^2\right)}+\frac{a^2 r^2 \sin ^2 \theta}{a^2+r^2}+\frac{r^4 \sin ^2 \theta}{2\left(a^2+r^2\right)} \\
         & =\frac{a^4}{2\left(a^2+r^2\right)}+\frac{a^2 r^2}{a^2+r^2}+\frac{r^4}{2\left(a^2+r^2\right)}-\frac{a^4 \cos 2 \theta}{2\left(a^2+r^2\right)}-                                              \\
         & \frac{2 a^2 r^2 \cos 2 \theta}{2\left(a^2+r^2\right)}-\frac{r^4 \cos 2 \theta}{2\left(a^2+r^2\right)}                                                                                      \\
         & =\frac{1}{2\left(a^2+r^2\right)}\left(\left(a^2+r^2\right)^2-\left(a^2+r^2\right)^2 \cos 2 \theta\right)                                                                                   \\
         & =\frac{1}{2}\left(\left(a^2+r^2\right)-\left(a^2+r^2\right) \cos 2 \theta\right)                                                                                                           \\
         & =\frac{1}{2}\left(\left(a^2+r^2\right)-\left(a^2+r^2\right)\left(1-2 \sin ^2 \theta\right)\right)                                                                                          \\
         & =\frac{1}{2}\left(2\left(a^2+r^2\right) \sin ^2 \theta\right)                                                                                                                              \\
         & =\left(a^2+r^2\right) \sin ^2 \theta
    \end{aligned}
\end{equation}

- Lets make the useful substitution, $\rho^2=r^2+a^2 \cos ^2 \theta$


\begin{align}
    d s^2= & -c^2 d t^2+d x^2+d y^2+d z^2                                                                                       \\
    =-     & c^2 d t^2+\frac{\left(r^2+a^2 \cos ^2 \theta\right)}{a^2+r^2} d r^2+\left(r^2+a^2 \cos ^2 \theta\right) d \theta^2 \\
           & +\left(a^2+r^2\right) \sin ^2 \theta d \varphi^2                                                                   \\
    =      & -c^2 d t^2+\frac{\rho^2}{a^2+r^2} d r^2+\rho^2 d \theta^2+\left(a^2+r^2\right) \sin ^2 \theta d \varphi^2
\end{align}


- Notice there is no cross term in this form of the metric which is contrary to what we expect of a rotating source.
- We can, by way of guess work, add two cross terms that cancel each other out so that we are effectively adding zero to the right hand side.


\begin{align}
    d s^2=  & -c^2 d t^2+\frac{\rho^2}{a^2} d r^2+\rho^2 d \theta^2+\left(a^2+r^2\right) \sin ^2 \theta d \varphi^2                                    \\
    \approx & ?\left(c^2 d t^2-2 a \sin ^2 \theta c d t d \varphi+a^2 \sin ^4 \theta d \varphi^2\right)+\frac{\rho^2}{a^2+r^2} d r^2+\rho^2 d \theta^2 \\
            & ? ?\left(\left(a^2+r^2\right)^2 d \varphi^2-2 a\left(a^2+r^2\right) c d t d \varphi+a^2 c^2 d t^2\right)
\end{align}


- This looks like a binomial expansion where the undetermined constants in front can be found such that they result in the cross terms cancelling each other.


\begin{align}
    d s^2 \approx & -\frac{\left(a^2+r^2\right)}{\rho^2}\left(c^2 d t^2-2 a \sin ^2 \theta c d t d \varphi+a^2 \sin ^4 \theta d \varphi^2\right)+\frac{\rho^2}{a^2+r^2} d r^2 \\
                  & +\rho^2 d \theta^2+\frac{\sin ^2 \theta}{\rho^2}\left(\left(a^2+r^2\right)^2 d \varphi^2-2 a\left(a^2+r^2\right) c d t d \varphi+a^2 c^2 d t^2\right)
\end{align}

\begin{align}
    d s^2 \approx & -\frac{\left(a^2+r^2\right)}{\rho^2}\left(c d t-a \sin ^2 \theta d \varphi\right)^2+\frac{\rho^2}{a^2+r^2} d r^2+\rho^2 d \theta^2 \\
                  & +\frac{\sin ^2 \theta}{\rho^2}\left(\left(a^2+r^2\right) d \varphi-a c d t\right)^2
\end{align}


- We also are missing the presence of the source mass $M$ in any of the terms.
- We know that in the limit as $a \rightarrow 0$ and the mass ceases to rotate then this metric reduces to the Schwarzschild metric.
- The Schwarzschild metric has the mass $M$ of the source appearing in the $d t^2, d r^2$ terms as well as the cross term.
- So that suggests that we need to alter the coefficients of the first and second terms.


\begin{align}
    \frac{-\left(a^2+r^2\right)}{\rho^2} & \rightarrow \frac{-\left(a^2+r^2\right)+k}{\rho^2} \\
    \frac{\rho^2}{a^2+r^2}               & \rightarrow \frac{\rho^2}{a^2+r^2+h}
\end{align}


- Substituting these in we now have,


\begin{align}
    d s^2 \approx & \frac{-\left(a^2+r^2\right)+k}{\rho^2}\left(c d t-a \sin ^2 \theta d \varphi\right)^2+\frac{\rho^2}{a^2+r^2+h} d r^2                                 \\
                  & +\rho^2 d \theta^2+\frac{\sin ^2 \theta}{\rho^2}\left(\left(a^2+r^2\right) d \varphi-a c d t\right)^2                                                \\
    =             & \frac{-\left(a^2+r^2\right)+k}{r^2+a^2 \cos ^2 \theta}\left(c d t-a \sin ^2 \theta d \varphi\right)^2+\frac{r^2+a^2 \cos ^2 \theta}{a^2+r^2+h} d r^2 \\
                  & +\left(r^2+a^2 \cos ^2 \theta\right) d \theta^2+\frac{\sin ^2 \theta}{r^2+a^2 \cos ^2 \theta}\left(\left(a^2+r^2\right) d \varphi-a c d t\right)^2
\end{align}

- For $a \rightarrow 0$ we get,

\begin{align*}
    \lim _{a \rightarrow 0} & \left[ \frac{-(a^2+r^2)+k}{r^2+a^2 \cos^2 \theta}(c\mathrm{d}t-a\sin^2\theta\mathrm{d}\phi)^2 + \frac{r^2+a^2\cos^2\theta}{a^2+r^2+h}\mathrm{d}r^2 \right.      \\
                            & +\left. (r^2+a^2\cos^2\theta)\mathrm{d}\theta^2 + \frac{\sin^2\theta}{r^2+a^2\cos^2\theta}\left((a^2+r^2)\mathrm{d}\phi - ac\mathrm{d}t\right)^2 \right]        \\
    ds^2                    & \approx \frac{-r^2+k}{r^2}(c\mathrm{d}t)^2 + \frac{r^2}{r^2+h}\mathrm{d}r^2 + r^2\mathrm{d}\theta^2 + \frac{\sin^2\theta}{r^2}(r^2\mathrm{d}\phi)^2             \\
                            & = \frac{-r^2+k}{r^2}c^2\mathrm{d}t^2 + \frac{r^2}{r^2+h}\mathrm{d}r^2 + r^2\mathrm{d}\theta^2 + r^2\sin^2\theta\mathrm{d}\phi^2                                 \\
                            & = -\left(1-\frac{k}{r^2}\right)c^2\mathrm{d}t^2 + \frac{r^2}{r^2+h}\mathrm{d}r^2 + r^2\mathrm{d}\theta^2 + r^2\sin^2\theta\mathrm{d}\phi^2                      \\
                            & = -\left(1-\frac{k}{r^2}\right)c^2\mathrm{d}t^2 + \left(\frac{1}{1+\frac{h}{r^2}}\right)\mathrm{d}r^2 + r^2\mathrm{d}\theta^2 + r^2\sin^2\theta\mathrm{d}\phi^2
\end{align*}

- This leads to the following identifications,


\begin{align}
     & 1-\frac{k}{r^2}=1-\frac{2 G M}{c^2 r} \Rightarrow k=\frac{2 G M r}{c^2}                                                              \\
     & \frac{1}{1+\frac{h}{r^2}}=\frac{1}{1-\frac{2 G M}{c^2 r}}=\left(1-\frac{2 G M}{c^2 r}\right)^{-1} \Rightarrow h=-\frac{2 G M r}{c^2}
\end{align}


- Which gives,


\begin{align}
    d s^2 \approx & \frac{-\left(a^2+r^2\right)+k}{r^2+a^2 \cos ^2 \theta}\left(c d t-a \sin ^2 \theta d \varphi\right)^2+\frac{r^2+a^2 \cos ^2 \theta}{a^2+r^2+h} d r^2                                     \\
                  & +\left(r^2+a^2 \cos ^2 \theta\right) d \theta^2+\frac{\sin ^2 \theta}{r^2+a^2 \cos ^2 \theta}\left(\left(a^2+r^2\right) d \varphi-a c d t\right)^2                                       \\
    =             & \frac{-\left(a^2+r^2\right)+\frac{2 G M r}{c^2}}{r^2+a^2 \cos ^2 \theta}\left(c d t-a \sin ^2 \theta d \varphi\right)^2+\frac{r^2+a^2 \cos ^2 \theta}{a^2+r^2-\frac{2 G M r}{c^2}} d r^2 \\
                  & +\left(r^2+a^2 \cos ^2 \theta\right) d \theta^2+\frac{\sin ^2 \theta}{r^2+a^2 \cos ^2 \theta}\left(\left(a^2+r^2\right) d \varphi-a c d t\right)^2
\end{align}


- Now make the substitutions,

$$
    \rho^2=r^2+a^2 \cos ^2 \theta \quad \text { and } \quad \Delta=a^2+r^2-\frac{2 G M r}{c^2}
$$

- Which leads to,


\begin{align}
    d s^2 \approx & \frac{-\left(a^2+r^2\right)+\frac{2 G M r}{c^2}}{r^2+a^2 \cos ^2 \theta}\left(c d t-a \sin ^2 \theta d \varphi\right)^2+\frac{r^2+a^2 \cos ^2 \theta}{a^2+r^2-\frac{2 G M r}{c^2}} d r^2 \\
                  & +\left(r^2+a^2 \cos ^2 \theta\right) d \theta^2+\frac{\sin ^2 \theta}{r^2+a^2 \cos ^2 \theta}\left(\left(a^2+r^2\right) d \varphi-a c d t\right)^2                                       \\
    =-            & \frac{\Delta}{\rho^2}\left(c d t-a \sin ^2 \theta d \varphi\right)^2+\frac{\rho^2}{\Delta} d r^2+\rho^2 d \theta^2                                                                       \\
                  & +\frac{\sin ^2 \theta}{\rho^2}\left(\left(a^2+r^2\right) d \varphi-a c d t\right)^2
\end{align}


- We can expand this and rewrite it.

\begin{align}
    d s^2 \approx & -\frac{\Delta}{\rho^2} c^2 d t^2+2 \frac{\Delta}{\rho^2} a \sin ^2 \theta c d t d \varphi-\frac{\Delta}{\rho^2} a^2 \sin ^4 \theta d \varphi^2               \\
                  & +\frac{\rho^2}{\Delta} d r^2+\rho^2 d \theta^2+\frac{\sin ^2 \theta}{\rho^2}\left(a^2+r^2\right)^2 d \varphi^2                                               \\
                  & -2 \frac{\sin ^2 \theta}{\rho^2}\left(a^2+r^2\right) a c d t d \varphi+\frac{\sin ^2 \theta}{\rho^2} a^2 c^2 d t^2                                           \\
    =             & -\left(\frac{\Delta}{\rho^2}-\frac{\sin ^2 \theta}{\rho^2} a^2\right) c^2 d t^2-2 \frac{a \sin ^2 \theta}{\rho^2}\left(a^2+r^2-\Delta\right) c d t d \varphi \\
                  & +\frac{\rho^2}{\Delta} d r^2+\rho^2 d \theta^2+\frac{\sin ^2 \theta}{\rho^2}\left(\left(a^2+r^2\right)^2-\Delta a^2 \sin ^2 \theta\right) d \varphi^2
\end{align}
- Now we make the substitutions,
\begin{align}
     & a^2+r^2-\Delta=a^2+r^2-\left(a^2+r^2-\frac{2 G M r}{c^2}\right)=\frac{2 G M r}{c^2} \\
     & \Sigma^2=\left(a^2+r^2\right)^2-\Delta a^2 \sin ^2 \theta
\end{align}

\begin{align}
    d s^2 \approx & -\left(\frac{\Delta-a^2 \sin ^2 \theta}{\rho^2}\right) c^2 d t^2-2 \frac{a \sin ^2 \theta}{\rho^2} \frac{2 G M r}{c^2} c d t d \varphi \\
                  & +\frac{\rho^2}{\Delta} d r^2+\rho^2 d \theta^2+\frac{\Sigma^2 \sin ^2 \theta}{\rho^2} d \varphi^2                                      \\
    =             & -\frac{\Delta-a^2 \sin ^2 \theta}{\rho^2} c^2 d t^2-\frac{4 G M r a \sin ^2 \theta}{c^2 \rho^2} c d t d \varphi                        \\
                  & +\frac{\rho^2}{\Delta} d r^2+\rho^2 d \theta^2+\frac{\Sigma^2 \sin ^2 \theta}{\rho^2} d \varphi^2
\end{align}


- This is one form of the Kerr metric we seek in Boyer-Lindquist coordinates $(t, r, \theta, \varphi)$.
- To verify that it actually works we need to check that it satisfies the Einstein equation in the vacuum or empty space case.
- One form of the Kerr metric is,

$$
    g_{\mu \nu}=\left[\begin{array}{cccc}
            -\frac{\Delta-a^2 \sin ^2 \theta}{\rho^2}    & 0                     & 0      & -\frac{2 G M r a \sin ^2 \theta}{c^2 \rho^2} \\
            0                                            & \frac{\rho^2}{\Delta} & 0      & 0                                            \\
            0                                            & 0                     & \rho^2 & 0                                            \\
            -\frac{2 G M r a \sin ^2 \theta}{c^2 \rho^2} & 0                     & 0      & \frac{\Sigma^2 \sin ^2 \theta}{\rho^2}
        \end{array}\right]
$$

- One form of the line element is,

$$
    d s^2 \approx-\frac{\Delta-a^2 \sin ^2 \theta}{\rho^2} c^2 d t^2-\frac{4 G M r a \sin ^2 \theta}{c^2 \rho^2} c d t d \varphi+\frac{\rho^2}{\Delta} d r^2+\rho^2 d \theta^2+\frac{\Sigma^2 \sin ^2 \theta}{\rho^2} d \varphi^2
$$

- We can make some further approximations to give us the familiar form of the Kerr metric.
\begin{align}
     & \frac{\Delta-a^2 \sin ^2 \theta}{\rho^2}=\frac{a^2+r^2-\frac{2 G M r}{c^2}-a^2 \sin ^2 \theta}{r^2+a^2 \cos ^2 \theta}                       \\
     & =\frac{a^2+r^2}{r^2+a^2 \cos ^2 \theta}-\frac{\frac{2 G M r}{c^2}}{r^2+a^2 \cos ^2 \theta}-\frac{a^2 \sin ^2 \theta}{r^2+a^2 \cos ^2 \theta} \\
     & \approx 1-\frac{2 G M r}{c^2 \rho^2}                                                                                                         \\
\end{align}
\begin{align}
    \frac{\Sigma^2 \sin ^2 \theta}{\rho^2}= & \frac{1}{\rho^2}\left[\left(a^2+r^2\right)^2-\Delta a^2 \sin ^2 \theta\right] \sin ^2 \theta                                                                                                \\
    =                                       & \frac{1}{\rho^2}\left[\left(a^2+r^2\right)^2-\left(a^2+r^2-\frac{2 G M r}{c^2}\right) a^2 \sin ^2 \theta\right] \sin ^2 \theta                                                              \\
    =                                       & {\left[\frac{\left(a^2+r^2\right)^2}{r^2+a^2 \cos ^2 \theta}-\frac{a^2+r^2}{r^2+a^2 \cos ^2 \theta}+\frac{2 G M r / c^2}{r^2+a^2 \cos ^2 \theta} a^2 \sin ^2 \theta\right] \sin ^2 \theta } \\
    \approx                                 & {\left[a^2+r^2-1+\frac{2 G M r a^2 \sin ^2 \theta}{c^2 \rho^2}\right] \sin ^2 \theta }                                                                                                      \\
    \approx                                 & \left(a^2+r^2+\frac{2 G M r a^2 \sin ^2 \theta}{c^2 \rho^2}\right) \sin ^2 \theta
\end{align}
- Finally, this leads to the Kerr line elelment,


\begin{align}
    d s^2 \approx- & \left(1-\frac{2 G M r}{c^2 \rho^2}\right) c^2 d t^2-\frac{4 G M r a \sin ^2 \theta}{c^2 \rho^2} c d t d \varphi+\frac{\rho^2}{\Delta} d r^2+\rho^2 d \theta^2 \\
                   & +\left(a^2+r^2+\frac{2 G M r a^2 \sin ^2 \theta}{c^2 \rho^2}\right)\sin^2(\theta) d \varphi^2
\end{align}


- Finally, the Kerr metric is,

$$
    g_{\mu \nu}=\left[\begin{array}{cccc}
            -\left(1-\frac{2 G M r}{c^2 \rho^2}\right)   & 0                     & 0      & -\frac{2 G M r a \sin ^2 \theta}{c^2 \rho^2}                          \\
            0                                            & \frac{\rho^2}{\Delta} & 0      & 0                                                                     \\
            0                                            & 0                     & \rho^2 & 0                                                                     \\
            -\frac{2 G M r a \sin ^2 \theta}{c^2 \rho^2} & 0                     & 0      & (a^2+r^2+\frac{2 G M r a^2 \sin ^2 \theta}{c^2 \rho^2})\sin^2(\theta)
        \end{array}\right]
$$

- Once again, the line element is,


\begin{align}
    d s^2 \approx- & \left(1-\frac{2 G M r}{c^2 \rho^2}\right) c^2 d t^2-\frac{4 G M r a \sin ^2 \theta}{c^2 \rho^2} c d t d \varphi+\frac{\rho^2}{\Delta} d r^2+\rho^2 d \theta^2 \\
                   & +\left(a^2+r^2+\frac{2 G M r a^2 \sin ^2 \theta}{c^2 \rho^2}\right) \sin^2(\theta) d \varphi^2
\end{align}


- It is this version that satisfies the empty space field equations, $R_{\alpha \beta}=0$.

\subsection{update kerr}
Kerr Metric - 2 (update 2)

This video is the second update to the video titled "Kerr Metric - 2". It contains an alternative derivation for the coefficient of the $\mathrm{d} \varphi^2$ term of the Kerr line element. Many thanks to the viewer named Errenium for pointing out this approach.

- In the previous video the Kerr line element was found to be,

$$
    d s^2 \approx-\frac{\Delta-a^2 \sin ^2 \theta}{\rho^2} c^2 d t^2-\frac{4 G M r a \sin ^2 \theta}{c^2 \rho^2} c d t d \varphi+\frac{\rho^2}{\Delta} d r^2+\rho^2 d \theta^2+\frac{\Sigma^2 \sin ^2 \theta}{\rho^2} d \varphi^2
$$

- Where, $\rho^2=r^2+a^2 \cos ^2 \theta$ and $\Delta=a^2+r^2-\frac{2 G M r}{c^2}$
- And, $\Sigma^2=\left(a^2+r^2\right)^2-\Delta a^2 \sin ^2 \theta$


\begin{align}
    \frac{\Delta-a^2 \sin ^2 \theta}{\rho^2} & =\frac{a^2+r^2-\frac{2 G M r}{c^2}-a^2 \sin ^2 \theta}{r^2+a^2 \cos ^2 \theta}                                                               \\
                                             & =\frac{a^2+r^2}{r^2+a^2 \cos ^2 \theta}-\frac{\frac{2 G M r}{c^2}}{r^2+a^2 \cos ^2 \theta}-\frac{a^2 \sin ^2 \theta}{r^2+a^2 \cos ^2 \theta} \\
                                             & \approx 1-\frac{2 G M r}{c^2 \rho^2}
\end{align}

\begin{equation}
    \begin{aligned}
        \frac{\Sigma^2 \sin ^2 \theta}{\rho^2} & =\frac{1}{\rho^2}\left[\left(a^2+r^2\right)^2-\Delta a^2 \sin ^2 \theta\right] \sin ^2 \theta                                                                         \\
                                               & =\frac{1}{\rho^2}\left[\left(a^2+r^2\right)^2-\left(a^2+r^2-\frac{2 G M r}{c^2}\right) a^2 \sin ^2 \theta\right] \sin ^2 \theta                                       \\
                                               & =\frac{1}{\rho^2}\left[a^4+2 a^2 r^2+r^4-a^4 \sin ^2 \theta-r^2 a^2 \sin ^2 \theta+\frac{2 G M r}{c^2} a^2 \sin ^2 \theta\right] \sin ^2 \theta                       \\
                                               & =\frac{1}{\rho^2}\left[a^4-a^4 \sin ^2 \theta+r^4+a^2 r^2+a^2 r^2-a^2 r^2 \sin ^2 \theta+\frac{2 G M r}{c^2} a^2 \sin ^2 \theta\right] \sin ^2 \theta                 \\
                                               & =\frac{1}{\rho^2}\left[a^4\left(1-\sin ^2 \theta\right)+r^4+a^2 r^2+a^2 r^2\left(1-\sin ^2 \theta\right)+\frac{2 G M r}{c^2} a^2 \sin ^2 \theta\right] \sin ^2 \theta \\
                                               & =\frac{1}{\rho^2}\left[a^4 \cos ^2 \theta+r^4+a^2 r^2+a^2 r^2 \cos ^2 \theta+\frac{2 G M r}{c^2} a^2 \sin ^2 \theta\right] \sin ^2 \theta                             \\
                                               & =\frac{1}{\rho^2}\left[r^2\left(r^2+a^2\right)+a^2 \cos ^2 \theta\left(r^2+a^2\right)+\frac{2 G M r}{c^2} a^2 \sin ^2 \theta\right] \sin ^2 \theta                    \\
                                               & =\frac{1}{\rho^2}\left[\left(r^2+a^2\right)\left(r^2+a^2 \cos ^2 \theta\right)+\frac{2 G M r}{c^2} a^2 \sin ^2 \theta\right] \sin ^2 \theta                           \\
                                               & =\frac{1}{\rho^2}\left[\left(r^2+a^2\right) \rho^2+\frac{2 G M r}{c^2} a^2 \sin ^2 \theta\right] \sin ^2 \theta                                                       \\
                                               & =\left[\left(r^2+a^2\right)+\frac{2 G M r}{c^2 \rho^2} a^2 \sin ^2 \theta\right] \sin ^2 \theta                                                                       \\
                                               & =\left(r^2+a^2+\frac{2 G M r}{c^2 \rho^2} a^2 \sin ^2 \theta\right) \sin ^2 \theta
    \end{aligned}
\end{equation}

- The line element we started with was,


\begin{align}
    d s^2 \approx- & \frac{\Delta-\sin ^2 \theta\left(a^2 / c^2\right)}{\rho^2} c^2 d t^2-\frac{4 G M r a \sin ^2 \theta}{c^3 \rho^2} c d t d \varphi+\frac{\rho^2}{\Delta} d r^2 \\
                   & +\rho^2 d \theta^2+\frac{\Sigma^2 \sin ^2 \theta}{\rho^2} d \varphi^2
\end{align}


- And we found,

$$
    \begin{array}{l}
        \frac{\Delta-a^2 \sin ^2 \theta}{\rho^2} \approx 1-\frac{2 G M r}{c^2 \rho^2} \\
        \frac{\Sigma^2 \sin ^2 \theta}{\rho^2}=\left(r^2+a^2+\frac{2 G M r}{c^2 \rho^2} a^2 \sin ^2 \theta\right) \sin ^2 \theta
    \end{array}
$$

- Finally, this leads to the Kerr line element,

$$
    \begin{array}{l}
        d s^2 \approx-\left(1-\frac{2 G M r}{c^2 \rho^2}\right) c^2 d t^2-\frac{4 G M r a \sin ^2 \theta}{c^2 \rho^2} c d t d \varphi+\frac{\rho^2}{\Delta} d r^2+\rho^2 d \theta^2 \\
        +\left(a^2+r^2+\frac{2 G M r a^2 \sin ^2 \theta}{c^2 \rho^2}\right) \sin ^2 \theta d \varphi^2
    \end{array}
$$

- Finally, the Kerr metric is,

$$
    g_{\mu \nu}=\left[\begin{array}{cccc}
            -\left(1-\frac{2 G M r}{c^2 \rho^2}\right)   & 0                     & 0      & -\frac{2 G M r a \sin ^2 \theta}{c^2 \rho^2}                                                       \\
            0                                            & \frac{\rho^2}{\Delta} & 0      & 0                                                                                                  \\
            0                                            & 0                     & \rho^2 & 0                                                                                                  \\
            -\frac{2 G M r a \sin ^2 \theta}{c^2 \rho^2} & 0                     & 0      & \left(a^2+r^2+\frac{2 G M r a^2 \sin ^2 \theta}{c^2 \rho^2} \frac{\Delta}{2}\right) \sin ^2 \theta
        \end{array}\right]
$$

- Once again, the line element is,

$$
    \begin{array}{l}
        d s^2 \approx-\left(1-\frac{2 G M r}{c^2 \rho^2}\right) c^2 d t^2-\frac{4 G M r a \sin ^2 \theta}{c^2 \rho^2} c d t d \varphi+\frac{\rho^2}{\Delta} d r^2+\rho^2 d \theta^2 \\
        +\left(a^2+r^2+\frac{2 G M r a^2 \sin ^2 \theta}{c^2 \rho^2}\right) \sin ^2 \theta d \varphi^2
    \end{array}
$$

- It is this version that satisfies the empty space field equations, $R_{\alpha \beta}=0$.


\section{Algoritmo de Newman-Janis y agujeros negros rotantes}