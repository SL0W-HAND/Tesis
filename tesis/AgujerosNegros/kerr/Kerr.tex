\section{Solución de Kerr}
\noindent En el caso de un agujero negro en rotación (métrica de Kerr), el espacio-tiempo fuera de él es estacionario, y se puede elegir un marco de referencia independiente del tiempo que asintóticamente tiende al marco de Lorentz en el infinito.
Primero analizamos las condiciones subyacentes requeridas para desarrollar una métrica que describa el espacio-tiempo alrededor de una masa en rotación uniforme. Nos enfocamos primero en el caso de velocidad lenta y campo débil antes de considerar el efecto de arrastre de marco (frame dragging).

Dada la simetría axial, comenzamos con las coordenadas:
\begin{equation}
    x^\mu = \left(x^0, x^1, x^2, x^3\right) = (c t, r, \theta, \varphi)
\end{equation}
Un espacio-tiempo estacionario es aquel donde los componentes de $g_{\mu\nu}$ son independientes de $t$:
\begin{equation}
    \frac{\partial g_{\mu\nu}}{\partial x^0} = \frac{\partial g_{\mu\nu}}{\partial t} = 0
\end{equation}
Un espacio-tiempo axialmente simétrico tiene componentes independientes de $\varphi$:
\begin{equation}
    \frac{\partial g_{\mu\nu}}{\partial x^3} = \frac{\partial g_{\mu\nu}}{\partial \varphi} = 0
\end{equation}

%Esto implica la existencia de dos vectores de Killing:
%\begin{equation}
%    \mathbf{e}_t \quad \text{y} \quad \mathbf{e}_\varphi
%\end{equation}
La métrica tiene la forma genérica inicial:
\begin{equation}
    g_{\mu\nu}(r, \theta) = \begin{bmatrix}
        g_{00} & g_{01} & g_{02} & g_{03} \\
        g_{10} & g_{11} & g_{12} & g_{13} \\
        g_{20} & g_{21} & g_{22} & g_{23} \\
        g_{30} & g_{31} & g_{32} & g_{33}
    \end{bmatrix} = g_{\nu\mu}
\end{equation}

El elemento de línea es:
\begin{equation}
    \mathrm{d}s^2 = g_{\mu\nu} \mathrm{d}x^\mu \mathrm{d}x^\nu
\end{equation}

\begin{itemize}
    \item La invariancia bajo $t \to -t$ y $\varphi \to -\varphi$ implica:
          \begin{align}
              (-c \mathrm{d}t)(-\mathrm{d}\varphi) & = c \mathrm{d}t \mathrm{d}\varphi \\
              \Rightarrow g_{03}                   & = g_{30} \neq 0
          \end{align}

    \item No existen simetrías para otros pares de coordenadas, eliminando términos cruzados adicionales, como por ejemplo $(-c d t)(-c d t)=c^2 d t^2$ pero $-c d t d r \neq c d t d r$.             .
\end{itemize}
Para ilustrar esto tómese el caso donde el elemento de linea lo consideramos bajo diferentes cambios de signos:
\begin{align*}
     & d s^2=g_{00} c^2 d t^2+g_{01} c d t d r+\ldots \ldots \ldots  \\
     & d s^2=g_{00} c^2 d t^2-g_{01} c d t d r+\ldots \ldots \ldots. \\
\end{align*}
Donde es fácil observar que bajo los cambios de signo, los términos cruzados no son iguales,
\[-c d t d \theta \neq c d t d \theta,-d \theta d \varphi \neq d \theta d \varphi,-d r d \varphi \neq d r d \varphi,\]
Lo que implica directamente que , $g_{01}=g_{02}=g_{23}=g_{13}=0$ y podamos escribir el elemento de linea  en bloques como:
\begin{align}
    \mathrm{d}s^2 & = g_{00} c^2 \mathrm{d}t^2 + 2 g_{03} \mathrm{d}t \mathrm{d}\varphi + g_{33} \mathrm{d}\varphi^2 \nonumber \\
                  & + \left(g_{11} \mathrm{d}r^2 + 2 g_{12} \mathrm{d}r \mathrm{d}\theta + g_{22} \mathrm{d}\theta^2\right).
\end{align}
Los coeficientes $g_{\mu\nu}$ dependen solo de $r$ y $\theta$. El término entre paréntesis corresponde a una subvariedad bidimensional:
\begin{align}
    \mathrm{d}s^2 & = g_{00} c^2 \mathrm{d}t^2 + 2 g_{03} c \mathrm{d}t \mathrm{d}\varphi + g_{33} \mathrm{d}\varphi^2 \nonumber \\
                  & + g_{\alpha\beta} \mathrm{d}x^\alpha \mathrm{d}x^\beta \quad (\alpha,\beta = 1,2).
\end{align}
Toda variedad pseudo-Riemanniana bidimensional es \textbf{conformalmente plana}. Esto significa que existe un factor conforme $\Omega(r,\theta)$ tal que:
\begin{equation}
    g_{\alpha\beta} = \Omega^2(r,\theta) \eta_{\alpha\beta}, \quad \text{con} \quad \eta_{\alpha\beta} = \begin{bmatrix}
        1 & 0 \\
        0 & 1
    \end{bmatrix}.
\end{equation}
Esta propiedad implica que:
\begin{itemize}
    \item El término cruzado $g_{12}$ se anula ($\eta_{12} = 0 \Rightarrow g_{12} = 0$)
    \item La métrica 2D puede diagonalizarse mediante transformaciones conformes
    \item $\Omega(r,\theta)$ codifica la curvatura mediante un factor escalar
\end{itemize}
La métrica simplificada queda entonces:
\begin{align}
    \mathrm{d}s^2 & = g_{tt} c^2 \mathrm{d}t^2 + g_{rr} \mathrm{d}r^2 + g_{\theta\theta} \mathrm{d}\theta^2 \nonumber \\
                  & + 2 g_{t\varphi} c \mathrm{d}t \mathrm{d}\varphi + g_{\varphi\varphi} \mathrm{d}\varphi^2
\end{align}
como condición final, imponemos que la forma final de esta métrica debe reducirse a la métrica de Schwarzschild cuando la rotación desaparece.
Recordemos que el elemento de línea de Schwarzschild es,
$$
    d s_{Sch}^2=-\left(1-\frac{2 G M}{c^2 r}\right) c^2 d t^2+\left(1-\frac{2 G M}{c^2 r}\right)^{-1} d r^2+r^2\left(d \theta^2+\sin ^2 \theta d \varphi^2\right).
$$
En el caso de una masa en rotación lenta en la aproximación de campo débil, podríamos imaginar una corrección a la métrica de Schwarzschild para obtener un nuevo elemento de línea que describa este caso, en la forma,

$$
    d s^2=d s_{S c h}^2-\frac{4 G M a c}{c^3 r^2} \sin ^2 \theta r d \varphi c d t=d s_{S c h}^2-\frac{4 G J}{c^3 r^2} \sin ^2 \theta r d \varphi c d t
$$
El último término es el de Lense-Thirring, que describe el arrastre de los marcos inerciales y es válido solo hasta primer orden en el momento angular, $J = Mac$. (No voy a discutir mucho sobre el efecto Lense-Thirring, porque eso va más allá del alcance de este trabajo. Si quiere ver más, consulte ).
\begin{multline}
    d s^2 = - \left(1-\frac{2 G M}{c^2 r}\right) c^2 d t^2
    + \left(1-\frac{2 G M}{c^2 r}\right)^{-1} d r^2 \\
    + r^2\left(d \theta^2+\sin^2 \theta d \varphi^2\right)
    - \frac{4 G J}{c^3 r} \sin^2 \theta d \varphi c d t.
\end{multline}
bajo la aproximación de campo débil cuando $\frac{2GM}{c^2r} \ll 1$ (y $v \ll c$):
\begin{equation}
    \left(1-\frac{2GM}{c^2r}\right)^{-1} \approx 1 + \frac{2GM}{c^2r} \quad (\text{Expansión de Taylor a primer orden})
\end{equation}
Esto puede ocurrir cuando $M$ es pequeño o $r$ es grande o ambas condiciones son verdaderas.
Esto lleva a la siguiente forma de la aproximación de campo débil para una masa en rotación lenta, la métrica se simplifica a:
\begin{multline}
    ds^2 \approx -\left(1-\frac{2GM}{c^2r}\right)c^2\mathrm{d}t^2
    + \left(1+\frac{2GM}{c^2r}\right)\mathrm{d}r^2 \\
    + r^2\left(\mathrm{d}\theta^2 + \sin^2\theta\mathrm{d}\varphi^2\right)
    - \frac{4GJ}{c^3r}\sin^2\theta\, \mathrm{d}\varphi\, \mathrm{d}t
\end{multline}
En términos del parámetro de Kerr $a = J/Mc$:
\begin{align}
    ds^2 \approx & -\left(1-\frac{2GM}{c^2r}\right)c^2\mathrm{d}t^2
    + \left(1+\frac{2GM}{c^2r}\right)\mathrm{d}r^2 \nonumber                              \\
                 & + r^2\left(\mathrm{d}\theta^2 + \sin^2\theta\mathrm{d}\varphi^2\right)
    - \frac{4GMa}{c^2r}\sin^2\theta\, \mathrm{d}\varphi\, \mathrm{d}t
\end{align}

Continuando con la aproximación.
\begin{equation}
    \begin{aligned}
        d s^2 \approx & -\left(1-\frac{2 G M}{c^2 r}\right) c^2 d t^2+\left(1+\frac{2 G M}{c^2 r}\right) d r^2+r^2\left(d \theta^2+\sin ^2 \theta d \varphi^2\right) \\
                      & -\frac{4 G M a c}{c^3 r} \sin ^2 \theta d \varphi c d t                                                                                      \\
        \approx       & -c^2 d t^2+\left(1+\frac{2 G M}{c^2 r}\right) d r^2+r^2\left(d \theta^2+\sin ^2 \theta d \varphi^2\right)+\frac{2 G M}{c^2 r} c^2 d t^2      \\
                      & -\frac{4 G M a}{c^2 r} \sin ^2 \theta d \varphi c d t+\frac{2 G M}{c^2 r} a^2 \sin ^4 \theta d \varphi^2                                     \\
        =             & -c^2 d t^2+\left(1+\frac{2 G M}{c^2 r}\right) d r^2+r^2\left(d \theta^2+\sin ^2 \theta d \varphi^2\right)                                    \\
                      & +\frac{2 G M}{c^2 r}\left(c^2 d t^2-2 a \sin ^2 \theta d \varphi c d t+a^2 \sin ^4 \theta d \varphi^2\right)                                 \\
        =             & -c^2 d t^2+\left(1+\frac{2 G M}{c^2 r}\right) d r^2+r^2\left(d \theta^2+\sin ^2 \theta d \varphi^2\right)                                    \\
                      & +\frac{2 G M}{c^2 r}\left(c d t-a \sin ^2 \theta d \varphi\right)^2
    \end{aligned}
\end{equation}
Al expandir el cuadrado perfecto en la última línea:
\begin{equation}
    \left(c\mathrm{d}t - a\sin^2\theta\mathrm{d}\varphi\right)^2 = c^2\mathrm{d}t^2 - 2ac\sin^2\theta\mathrm{d}t\mathrm{d}\varphi + a^2\sin^4\theta\mathrm{d}\varphi^2
\end{equation}

El término $a^2\sin^4\theta\mathrm{d}\varphi^2$ aparece naturalmente en la expansión algebraica. Sin embargo, en nuestra aproximación a \textbf{primer orden en $a$}, este término sería de orden $\mathcal{O}(a^2)$ y en principio debería despreciarse. Su presencia aquí se justifica por dos razones:

\begin{itemize}
    \item \textbf{Consistencia matemática}: Al mantenerlo, preservamos la estructura exacta del cuadrado perfecto, lo que facilita la interpretación geométrica.

    \item \textbf{Orden de aproximación}: Aunque $a^2$ es técnicamente de segundo orden, el factor $\frac{2GM}{c^2r}$ (que es $\ll1$ en el régimen de campo débil) lo convierte en un término de orden $\mathcal{O}\left(\frac{GM}{c^2r}a^2\right)$, que puede considerarse despreciable frente a los términos lineales en $a$.
\end{itemize}
Esta ecuación sigue siendo válida hasta el primer orden en $a$.
Sabemos que los cuerpos en rotación se abultan en el plano ecuatorial alejándose del eje de rotación y se vuelven esferoidales oblatos en forma. Esto sugiere que podemos hacer una conjetura educada y usar coordenadas esferoidales o coordenadas elipsoidales como punto de partida en nuestra búsqueda de la forma de la métrica de Kerr.

\begin{align}
     & x=\sqrt{r^2+a^2} \sin \theta \cos \varphi \\
     & y=\sqrt{r^2+a^2} \sin \theta \sin \varphi \\
     & z=r \cos \theta
\end{align}
La utilización de coordenadas esferoidales se motiva por el hecho de que la métrica de Kerr es axialmente simétrica y la fuente de la métrica está rotando. La métrica debería ser simétrica bajo rotaciones alrededor del eje de rotación. Además, se observa que, al principio, el parámetro $a$ de las coordenadas es simplemente una constante; pero lo que procederemos a hacer es la conjetura fundamentada de escribir la métrica en términos de las coordenadas esferoidales y, luego, proponemos que el parámetro $a $estará relacionado con el momento angular de la fuente. Finalmente, comparamos los términos con nuestra aproximación de la métrica.

Primero queremos expresar la métrica de Minkowski en coordenadas esferoidales oblatas.

$$
    d s^2=-c^2 d t^2+d x^2+d y^2+d z^2
$$
Ahora el cambio de coordenadas está dado por las siguientes ecuaciones,
\begin{align}
    d x & = \frac{\partial x}{\partial r} d r + \frac{\partial x}{\partial \theta} d \theta + \frac{\partial x}{\partial \varphi} d \varphi \nonumber                    \\
        & = \frac{r}{\sqrt{r^2+a^2}} \sin \theta \cos \varphi d r + \sqrt{r^2+a^2} \cos \theta \cos \varphi d \theta - \sqrt{r^2+a^2} \sin \theta \sin \varphi d \varphi
\end{align}
\begin{align}
    d y & = \frac{\partial y}{\partial r} d r + \frac{\partial y}{\partial \theta} d \theta + \frac{\partial y}{\partial \varphi} d \varphi \nonumber                    \\
        & = \frac{r}{\sqrt{r^2+a^2}} \sin \theta \sin \varphi d r + \sqrt{r^2+a^2} \cos \theta \sin \varphi d \theta + \sqrt{r^2+a^2} \sin \theta \cos \varphi d \varphi
\end{align}
\begin{align}
    d z & = \frac{\partial z}{\partial r} d r + \frac{\partial z}{\partial \theta} d \theta \nonumber \\
        & = \cos \theta d r - r \sin \theta d \theta
\end{align}

\begin{equation}
    \begin{aligned}
        d x^2+d y^2+d z^2 & = \left(\frac{a^2}{2\left(a^2+r^2\right)}+\frac{r^2}{a^2+r^2}+\frac{a^2 \cos ^2 \theta}{2\left(a^2+r^2\right)}-\frac{a^2 \sin ^2 \theta}{2\left(a^2+r^2\right)}\right) d r^2+                                                     \\
                          & \left(\frac{a^4}{2\left(a^2+r^2\right)}+\frac{3 a^2 r^2}{2\left(a^2+r^2\right)}+\frac{r^4}{a^2+r^2}+\frac{a^4 \cos ^2 \theta}{2\left(a^2+r^2\right)}+\right.                                                                      \\
                          & \left.\frac{a^2 r^2 \cos ^2 \theta}{2\left(a^2+r^2\right)}-\frac{a^4 \sin ^2 \theta}{2\left(a^2+r^2\right)}-\frac{a^2 r^2 \sin ^2 \theta}{2\left(a^2+r^2\right)}\right) d \theta^2+\left(\frac{a^4}{2\left(a^2+r^2\right)}\right. \\
                          & \frac{a^2 r^2}{a^2+r^2}+\frac{r^4}{2\left(a^2+r^2\right)}-\frac{a^4 \cos ^2 \theta}{2\left(a^2+r^2\right)}-\frac{a^2 r^2 \cos ^2 \theta}{a^2+r^2}-                                                                                \\
                          & \left.\frac{r^4 \cos ^2 \theta}{2\left(a^2+r^2\right)}+\frac{a^4 \sin ^2 \theta}{2\left(a^2+r^2\right)}+\frac{a^2 r^2 \sin ^2 \theta}{a^2+r^2}+\frac{r^4 \sin ^2 \theta}{2\left(a^2+r^2\right)}\right) d \varphi^2
    \end{aligned}
\end{equation}
Ahora hagamos algunas simplificaciones para los coeficientes.
Primero para el coeficiente de $d r^2$ tenemos,
\begin{align}
    \begin{aligned}
        \frac{a^2}{2\left(a^2+r^2\right)}+\frac{r^2}{a^2+r^2}+\frac{a^2 \cos ^2 \theta}{2\left(a^2+r^2\right)}-\frac{a^2 \sin ^2 \theta}{2\left(a^2+r^2\right)}    & =                                        \\
        \frac{a^2}{2\left(a^2+r^2\right)}+\frac{2 r^2}{2\left(a^2+r^2\right)}+\frac{a^2\left(2 \cos ^2 \theta-1\right)}{2\left(a^2+r^2\right)}                     & =                                        \\
        \frac{a^2}{2\left(a^2+r^2\right)}+\frac{2 r^2}{2\left(a^2+r^2\right)}+\frac{2 a^2 \cos ^2 \theta}{2\left(a^2+r^2\right)}-\frac{a^2}{2\left(a^2+r^2\right)} & =                                        \\
        \frac{2 r^2}{2\left(a^2+r^2\right)}+\frac{2 a^2 \cos ^2 \theta}{2\left(a^2+r^2\right)}                                                                     & = \frac{r^2+a^2 \cos ^2 \theta}{a^2+r^2}
    \end{aligned}
\end{align}
para   $ d \theta^2$
\begin{align}
    \begin{aligned}
         & \frac{a^4}{2\left(a^2+r^2\right)}+\frac{3 a^2 r^2}{2\left(a^2+r^2\right)}+\frac{r^4}{a^2+r^2}+\frac{a^4 \cos ^2 \theta}{2\left(a^2+r^2\right)}+                                                    \\
         & \frac{a^2 r^2 \cos ^2 \theta}{2\left(a^2+r^2\right)}-\frac{a^4 \sin ^2 \theta}{2\left(a^2+r^2\right)}-\frac{a^2 r^2 \sin ^2 \theta}{2\left(a^2+r^2\right)}=\frac{a^4}{2\left(a^2+r^2\right)}+      \\
         & \frac{3 a^2 r^2}{2\left(a^2+r^2\right)}+\frac{r^4}{a^2+r^2}+\frac{a^4\left(2 \cos ^2 \theta-1\right)}{2\left(a^2+r^2\right)}+\frac{a^2 r^2\left(2 \cos ^2 \theta-1\right)}{2\left(a^2+r^2\right)}= \\
         & \frac{2 a^2 r^2}{2\left(a^2+r^2\right)}+\frac{r^4}{a^2+r^2}+\frac{2 a^4 \cos ^2 \theta}{2\left(a^2+r^2\right)}+\frac{2 a^2 r^2 \cos ^2 \theta}{2\left(a^2+r^2\right)}=                             \\
         & \frac{a^2 r^2}{\left(a^2+r^2\right)}+\frac{r^4}{a^2+r^2}+\frac{a^4 \cos ^2 \theta}{\left(a^2+r^2\right)}+\frac{a^2 r^2 \cos ^2 \theta}{\left(a^2+r^2\right)}=                                      \\
         & \frac{\left(r^2+a^2 \cos ^2 \theta\right)\left(r^2+a^2\right)}{a^2+r^2}=   r^2+a^2 \cos ^2 \theta
    \end{aligned}
\end{align}
para $ d \varphi^2$
\begin{equation}
    \begin{aligned}
         & \frac{a^4}{2\left(a^2+r^2\right)}+\frac{a^2 r^2}{a^2+r^2}+\frac{r^4}{2\left(a^2+r^2\right)}-\frac{a^4 \cos ^2 \theta}{2\left(a^2+r^2\right)}-\frac{a^2 r^2 \cos ^2 \theta}{a^2+r^2}        \\
         & -\frac{r^4 \cos ^2 \theta}{2\left(a^2+r^2\right)}+\frac{a^4 \sin ^2 \theta}{2\left(a^2+r^2\right)}+\frac{a^2 r^2 \sin ^2 \theta}{a^2+r^2}+\frac{r^4 \sin ^2 \theta}{2\left(a^2+r^2\right)} \\
         & =\frac{a^4}{2\left(a^2+r^2\right)}+\frac{a^2 r^2}{a^2+r^2}+\frac{r^4}{2\left(a^2+r^2\right)}-\frac{a^4 \cos 2 \theta}{2\left(a^2+r^2\right)}-                                              \\
         & \frac{2 a^2 r^2 \cos 2 \theta}{2\left(a^2+r^2\right)}-\frac{r^4 \cos 2 \theta}{2\left(a^2+r^2\right)}                                                                                      \\
         & =\frac{1}{2\left(a^2+r^2\right)}\left(\left(a^2+r^2\right)^2-\left(a^2+r^2\right)^2 \cos 2 \theta\right)                                                                                   \\
         & =\frac{1}{2}\left(\left(a^2+r^2\right)-\left(a^2+r^2\right) \cos 2 \theta\right)                                                                                                           \\
         & =\frac{1}{2}\left(\left(a^2+r^2\right)-\left(a^2+r^2\right)\left(1-2 \sin ^2 \theta\right)\right)                                                                                          \\
         & =\frac{1}{2}\left(2\left(a^2+r^2\right) \sin ^2 \theta\right)                                                                                                                              \\
         & =\left(a^2+r^2\right) \sin ^2 \theta
    \end{aligned}
\end{equation}
Hagamos la sustitución, $\rho^2=r^2+a^2 \cos ^2 \theta$ para que podamos escribir la métrica en una forma más compacta,
\begin{align}
    d s^2= & -c^2 d t^2+d x^2+d y^2+d z^2                                                                                       \\
    =-     & c^2 d t^2+\frac{\left(r^2+a^2 \cos ^2 \theta\right)}{a^2+r^2} d r^2+\left(r^2+a^2 \cos ^2 \theta\right) d \theta^2 \\
           & +\left(a^2+r^2\right) \sin ^2 \theta d \varphi^2                                                                   \\
    =      & -c^2 d t^2+\frac{\rho^2}{a^2+r^2} d r^2+\rho^2 d \theta^2+\left(a^2+r^2\right) \sin ^2 \theta d \varphi^2
\end{align}
Esta es la métrica del espacio de Minkowski en coordenadas esferoidales, ahora queremos modificar la métrica para que se parezca a la que aproximamos antes, incluyendo la condición de que la métrica se reduzca a la métrica de Schwarzschild cuando la masa deja de rotar ($a = 0$).

Nótese que no hay un término cruzado en esta forma de la métrica, lo cual es contrario a lo que esperamos de una fuente rotante.
Podemos, a modo de conjetura, agregar dos términos cruzados que se cancelen entre sí para que efectivamente estemos agregando cero al lado derecho.

\begin{align}
    d s^2=  & -c^2 d t^2+\frac{\rho^2}{a^2+r^2} d r^2+\rho^2 d \theta^2+\left(a^2+r^2\right) \sin ^2 \theta d \varphi^2                                      \\
    \approx & \alpha \left(c^2 d t^2-2 a \sin ^2 \theta c d t d \varphi+a^2 \sin ^4 \theta d \varphi^2\right)+\frac{\rho^2}{a^2+r^2} d r^2+\rho^2 d \theta^2 \\
            & +\beta \left(\left(a^2+r^2\right)^2 d \varphi^2-2 a\left(a^2+r^2\right) c d t d \varphi+a^2 c^2 d t^2\right)
\end{align}
Esto parece una expansión binomial donde las constantes indeterminadas al frente pueden encontrarse de tal manera que resulten en la cancelación de los términos cruzados.
\begin{align}
    \begin{aligned}
        d s^2 \approx & -\frac{\left(a^2+r^2\right)}{\rho^2}\left(c^2 d t^2-2 a \sin ^2 \theta c d t d \varphi+a^2 \sin ^4 \theta d \varphi^2\right)+\frac{\rho^2}{a^2+r^2} d r^2 \\
                      & +\rho^2 d \theta^2+\frac{\sin ^2 \theta}{\rho^2}\left(\left(a^2+r^2\right)^2 d \varphi^2-2 a\left(a^2+r^2\right) c d t d \varphi+a^2 c^2 d t^2\right)     \\
                      & \approx  -\frac{\left(a^2+r^2\right)}{\rho^2}\left(c d t-a \sin ^2 \theta d \varphi\right)^2+\frac{\rho^2}{a^2+r^2} d r^2+\rho^2 d \theta^2               \\
                      & +\frac{\sin ^2 \theta}{\rho^2}\left(\left(a^2+r^2\right) d \varphi-a c d t\right)^2
    \end{aligned}
\end{align}
También nos falta la presencia de la masa fuente $M$ en cualquiera de los términos. Sabemos que en el límite cuando $a \rightarrow 0$ y la masa deja de rotar, esta métrica se reduce a la métrica de Schwarzschild.
La métrica de Schwarzschild tiene la masa $M$ de la fuente apareciendo en los términos $d t^2, d r^2$ así como en el término cruzado. Esto sugiere que necesitamos alterar los coeficientes de los primeros y segundos términos de la siguiente manera.

\begin{align}
    \frac{-\left(a^2+r^2\right)}{\rho^2} & \rightarrow \frac{-\left(a^2+r^2\right)+k}{\rho^2} \\
    \frac{\rho^2}{a^2+r^2}               & \rightarrow \frac{\rho^2}{a^2+r^2+h}
\end{align}
Sustituyendo estos términos, ahora tenemos,
\begin{align}
    d s^2 \approx & \frac{-\left(a^2+r^2\right)+k}{\rho^2}\left(c d t-a \sin ^2 \theta d \varphi\right)^2+\frac{\rho^2}{a^2+r^2+h} d r^2                                 \\
                  & +\rho^2 d \theta^2+\frac{\sin ^2 \theta}{\rho^2}\left(\left(a^2+r^2\right) d \varphi-a c d t\right)^2                                                \\
    =             & \frac{-\left(a^2+r^2\right)+k}{r^2+a^2 \cos ^2 \theta}\left(c d t-a \sin ^2 \theta d \varphi\right)^2+\frac{r^2+a^2 \cos ^2 \theta}{a^2+r^2+h} d r^2 \\
                  & +\left(r^2+a^2 \cos ^2 \theta\right) d \theta^2+\frac{\sin ^2 \theta}{r^2+a^2 \cos ^2 \theta}\left(\left(a^2+r^2\right) d \varphi-a c d t\right)^2
\end{align}

Cuando $a \rightarrow 0$ se obtiene,
\begin{align*}
    \lim _{a \rightarrow 0} & \left[ \frac{-(a^2+r^2)+k}{r^2+a^2 \cos^2 \theta}(c\mathrm{d}t-a\sin^2\theta\mathrm{d}\phi)^2 + \frac{r^2+a^2\cos^2\theta}{a^2+r^2+h}\mathrm{d}r^2 \right.      \\
                            & +\left. (r^2+a^2\cos^2\theta)\mathrm{d}\theta^2 + \frac{\sin^2\theta}{r^2+a^2\cos^2\theta}\left((a^2+r^2)\mathrm{d}\phi - ac\mathrm{d}t\right)^2 \right]        \\
    ds^2                    & \approx \frac{-r^2+k}{r^2}(c\mathrm{d}t)^2 + \frac{r^2}{r^2+h}\mathrm{d}r^2 + r^2\mathrm{d}\theta^2 + \frac{\sin^2\theta}{r^2}(r^2\mathrm{d}\phi)^2             \\
                            & = \frac{-r^2+k}{r^2}c^2\mathrm{d}t^2 + \frac{r^2}{r^2+h}\mathrm{d}r^2 + r^2\mathrm{d}\theta^2 + r^2\sin^2\theta\mathrm{d}\phi^2                                 \\
                            & = -\left(1-\frac{k}{r^2}\right)c^2\mathrm{d}t^2 + \frac{r^2}{r^2+h}\mathrm{d}r^2 + r^2\mathrm{d}\theta^2 + r^2\sin^2\theta\mathrm{d}\phi^2                      \\
                            & = -\left(1-\frac{k}{r^2}\right)c^2\mathrm{d}t^2 + \left(\frac{1}{1+\frac{h}{r^2}}\right)\mathrm{d}r^2 + r^2\mathrm{d}\theta^2 + r^2\sin^2\theta\mathrm{d}\phi^2
\end{align*}

Esto lleva a las siguientes identificaciones,
\begin{align}
     & 1-\frac{k}{r^2}=1-\frac{2 G M}{c^2 r} \Rightarrow k=\frac{2 G M r}{c^2}                                                              \\
     & \frac{1}{1+\frac{h}{r^2}}=\frac{1}{1-\frac{2 G M}{c^2 r}}=\left(1-\frac{2 G M}{c^2 r}\right)^{-1} \Rightarrow h=-\frac{2 G M r}{c^2}
\end{align}
lo que nos lleva a
\begin{align}
    d s^2 \approx & \frac{-\left(a^2+r^2\right)+k}{r^2+a^2 \cos ^2 \theta}\left(c d t-a \sin ^2 \theta d \varphi\right)^2+\frac{r^2+a^2 \cos ^2 \theta}{a^2+r^2+h} d r^2                                     \\
                  & +\left(r^2+a^2 \cos ^2 \theta\right) d \theta^2+\frac{\sin ^2 \theta}{r^2+a^2 \cos ^2 \theta}\left(\left(a^2+r^2\right) d \varphi-a c d t\right)^2                                       \\
    =             & \frac{-\left(a^2+r^2\right)+\frac{2 G M r}{c^2}}{r^2+a^2 \cos ^2 \theta}\left(c d t-a \sin ^2 \theta d \varphi\right)^2+\frac{r^2+a^2 \cos ^2 \theta}{a^2+r^2-\frac{2 G M r}{c^2}} d r^2 \\
                  & +\left(r^2+a^2 \cos ^2 \theta\right) d \theta^2+\frac{\sin ^2 \theta}{r^2+a^2 \cos ^2 \theta}\left(\left(a^2+r^2\right) d \varphi-a c d t\right)^2
\end{align}
Ahora hagamos las sustituciones,
$$
    \rho^2=r^2+a^2 \cos ^2 \theta \quad \text { and } \quad \Delta=a^2+r^2-\frac{2 G M r}{c^2}
$$
Lo que lleva a,
\begin{align}
    d s^2 \approx & \frac{-\left(a^2+r^2\right)+\frac{2 G M r}{c^2}}{r^2+a^2 \cos ^2 \theta}\left(c d t-a \sin ^2 \theta d \varphi\right)^2+\frac{r^2+a^2 \cos ^2 \theta}{a^2+r^2-\frac{2 G M r}{c^2}} d r^2 \\
                  & +\left(r^2+a^2 \cos ^2 \theta\right) d \theta^2+\frac{\sin ^2 \theta}{r^2+a^2 \cos ^2 \theta}\left(\left(a^2+r^2\right) d \varphi-a c d t\right)^2                                       \\
    =-            & \frac{\Delta}{\rho^2}\left(c d t-a \sin ^2 \theta d \varphi\right)^2+\frac{\rho^2}{\Delta} d r^2+\rho^2 d \theta^2                                                                       \\
                  & +\frac{\sin ^2 \theta}{\rho^2}\left(\left(a^2+r^2\right) d \varphi-a c d t\right)^2
\end{align}
Podemos expandir esto y reescribirlo.
\begin{align}
    d s^2 \approx & -\frac{\Delta}{\rho^2} c^2 d t^2+2 \frac{\Delta}{\rho^2} a \sin ^2 \theta c d t d \varphi-\frac{\Delta}{\rho^2} a^2 \sin ^4 \theta d \varphi^2               \\
                  & +\frac{\rho^2}{\Delta} d r^2+\rho^2 d \theta^2+\frac{\sin ^2 \theta}{\rho^2}\left(a^2+r^2\right)^2 d \varphi^2                                               \\
                  & -2 \frac{\sin ^2 \theta}{\rho^2}\left(a^2+r^2\right) a c d t d \varphi+\frac{\sin ^2 \theta}{\rho^2} a^2 c^2 d t^2                                           \\
    =             & -\left(\frac{\Delta}{\rho^2}-\frac{\sin ^2 \theta}{\rho^2} a^2\right) c^2 d t^2-2 \frac{a \sin ^2 \theta}{\rho^2}\left(a^2+r^2-\Delta\right) c d t d \varphi \\
                  & +\frac{\rho^2}{\Delta} d r^2+\rho^2 d \theta^2+\frac{\sin ^2 \theta}{\rho^2}\left(\left(a^2+r^2\right)^2-\Delta a^2 \sin ^2 \theta\right) d \varphi^2
\end{align}
Ahora hacemos las sustituciones,
\begin{align}
     & a^2+r^2-\Delta=a^2+r^2-\left(a^2+r^2-\frac{2 G M r}{c^2}\right)=\frac{2 G M r}{c^2} \\
     & \Sigma^2=\left(a^2+r^2\right)^2-\Delta a^2 \sin ^2 \theta
\end{align}
tal que,
\begin{align}
    d s^2 \approx & -\left(\frac{\Delta-a^2 \sin ^2 \theta}{\rho^2}\right) c^2 d t^2-2 \frac{a \sin ^2 \theta}{\rho^2} \frac{2 G M r}{c^2} c d t d \varphi \\
                  & +\frac{\rho^2}{\Delta} d r^2+\rho^2 d \theta^2+\frac{\Sigma^2 \sin ^2 \theta}{\rho^2} d \varphi^2                                      \\
    =             & -\frac{\Delta-a^2 \sin ^2 \theta}{\rho^2} c^2 d t^2-\frac{4 G M r a \sin ^2 \theta}{c^2 \rho^2} c d t d \varphi                        \\
                  & +\frac{\rho^2}{\Delta} d r^2+\rho^2 d \theta^2+\frac{\Sigma^2 \sin ^2 \theta}{\rho^2} d \varphi^2
\end{align}
Esta es una forma de la métrica de Kerr que buscamos en las coordenadas de Boyer-Lindquist $(t, r, \theta, \varphi)$.

Para verificar que realmente funciona, necesitamos comprobar que satisface la ecuación de Einstein en el caso de vacío o espacio vacío.
Una forma de la métrica de Kerr es

$$
    g_{\mu \nu}=\left[\begin{array}{cccc}
            -\frac{\Delta-a^2 \sin ^2 \theta}{\rho^2}    & 0                     & 0      & -\frac{2 G M r a \sin ^2 \theta}{c^2 \rho^2} \\
            0                                            & \frac{\rho^2}{\Delta} & 0      & 0                                            \\
            0                                            & 0                     & \rho^2 & 0                                            \\
            -\frac{2 G M r a \sin ^2 \theta}{c^2 \rho^2} & 0                     & 0      & \frac{\Sigma^2 \sin ^2 \theta}{\rho^2}
        \end{array}\right]
$$

Una forma del elemento de línea es,
$$
    d s^2 =-\frac{\Delta-a^2 \sin ^2 \theta}{\rho^2} c^2 d t^2-\frac{4 G M r a \sin ^2 \theta}{c^2 \rho^2} c d t d \varphi+\frac{\rho^2}{\Delta} d r^2+\rho^2 d \theta^2+\frac{\Sigma^2 \sin ^2 \theta}{\rho^2} d \varphi^2
$$
como nota podemos expresar el término $\frac{\Sigma^2 \sin ^2 \theta}{\rho^2}$ de la siguiente manera,
\begin{equation}
    \begin{aligned}
        \frac{\Sigma^2 \sin ^2 \theta}{\rho^2} & =\frac{1}{\rho^2}\left[\left(a^2+r^2\right)^2-\Delta a^2 \sin ^2 \theta\right] \sin ^2 \theta                                                                         \\
                                               & =\frac{1}{\rho^2}\left[\left(a^2+r^2\right)^2-\left(a^2+r^2-\frac{2 G M r}{c^2}\right) a^2 \sin ^2 \theta\right] \sin ^2 \theta                                       \\
                                               & =\frac{1}{\rho^2}\left[a^4+2 a^2 r^2+r^4-a^4 \sin ^2 \theta-r^2 a^2 \sin ^2 \theta+\frac{2 G M r}{c^2} a^2 \sin ^2 \theta\right] \sin ^2 \theta                       \\
                                               & =\frac{1}{\rho^2}\left[a^4-a^4 \sin ^2 \theta+r^4+a^2 r^2+a^2 r^2-a^2 r^2 \sin ^2 \theta+\frac{2 G M r}{c^2} a^2 \sin ^2 \theta\right] \sin ^2 \theta                 \\
                                               & =\frac{1}{\rho^2}\left[a^4\left(1-\sin ^2 \theta\right)+r^4+a^2 r^2+a^2 r^2\left(1-\sin ^2 \theta\right)+\frac{2 G M r}{c^2} a^2 \sin ^2 \theta\right] \sin ^2 \theta \\
                                               & =\frac{1}{\rho^2}\left[a^4 \cos ^2 \theta+r^4+a^2 r^2+a^2 r^2 \cos ^2 \theta+\frac{2 G M r}{c^2} a^2 \sin ^2 \theta\right] \sin ^2 \theta                             \\
                                               & =\frac{1}{\rho^2}\left[r^2\left(r^2+a^2\right)+a^2 \cos ^2 \theta\left(r^2+a^2\right)+\frac{2 G M r}{c^2} a^2 \sin ^2 \theta\right] \sin ^2 \theta                    \\
                                               & =\frac{1}{\rho^2}\left[\left(r^2+a^2\right)\left(r^2+a^2 \cos ^2 \theta\right)+\frac{2 G M r}{c^2} a^2 \sin ^2 \theta\right] \sin ^2 \theta                           \\
                                               & =\frac{1}{\rho^2}\left[\left(r^2+a^2\right) \rho^2+\frac{2 G M r}{c^2} a^2 \sin ^2 \theta\right] \sin ^2 \theta                                                       \\
                                               & =\left[\left(r^2+a^2\right)+\frac{2 G M r}{c^2 \rho^2} a^2 \sin ^2 \theta\right] \sin ^2 \theta                                                                       \\
                                               & =\left(r^2+a^2+\frac{2 G M r}{c^2 \rho^2} a^2 \sin ^2 \theta\right) \sin ^2 \theta
    \end{aligned}
\end{equation}

\subsection{Arrastre de marcos inerciales}

%%%%%%%%%%%%%%%%%%%%%%%%%%%%%%%%%%%%
Sabemos que esta métrica tiene dos vectores de Killing, de modo que a lo largo de las geodésicas seguidas por una partícula dada tenemos,

$$
    \frac{d}{d \lambda}({\underset{\sim}{x}} \cdot \underset{\sim}{u})=0
$$

Uno de los vectores de Killing es,

$$
    \underset{\sim}{K}=K^\alpha{\underset{\sim}{\alpha}}_\alpha=(0,0,0,1) \mathbf{e}_\varphi
$$

El cuatro-momento de la partícula basado en su cuatro-velocidad es,

$$
    \underset{\sim}{p}=m_0 \underset{\sim}{u}=\left(p^t, p^r, p^\theta, p^{\varphi}\right)
$$
En esta geometría, el componente del momento angular del cuatro-momento de la partícula se conserva a lo largo de sus geodésicas.


\begin{align}
     & \frac{d}{d \lambda}\left(\underset{\sim}{K} \cdot m_0 \underset{\sim}{u}\right)=\frac{d}{d \lambda}(\underset{\sim}{K} \cdot \underset{\sim}{p}) \\
     & =\frac{d}{d \lambda}\left((0,0,0,1) \frac{\partial}{\partial \varphi}-\left(p^t, p^r, p^\theta, p^{\varphi}\right)\right)                        \\
     & =\frac{d}{d \lambda}\left(g_{\varphi \varphi} 1 p^{\varphi}\right)                                                                               \\
     & =\frac{d}{d \lambda}\left(g_{\varphi \varphi} p^{\varphi}\right)                                                                                 \\
     & =\frac{d}{d \lambda}\left(p_{\varphi}\right)                                                                                                     \\
     & =0                                                                                                                                               \\
     & \Rightarrow p_{\varphi}=\text { constante }=\mathrm{L}
\end{align}

Para esta geometría tenemos dos vectores de Killing y una métrica de la forma,
$$
    d s^2=g_{t t} c^2 d t^2+g_{t r} d r^2+g_{\theta \theta} d \theta^2+2 g_{t \varphi} c d t d \varphi+g_{\varphi \varphi} d \varphi^2
$$
Los componentes contravariantes temporal y azimutal del cuatro-momento son:
\begin{align}
     & \mathrm{p}^{\varphi}=g^{\varphi \alpha} p_\alpha=g^{\varphi t} p_t+g^{\varphi \varphi} p_{\varphi} \\
     & \mathrm{p}^t=g^{t \alpha} p_\alpha=g^{t t} p_t+g^{t \varphi} p_{\varphi}
\end{align}

Ahora digamos que liberamos nuestra partícula, con masa $m_0$, desde el reposo y lejos de la masa fuente $M$ de modo que inicialmente tiene, $p_{\varphi}=0$ (a lo largo de su geodésica). Los componentes de su cuatro-momento que son constantes a lo largo de las geodésicas por las que se mueve son,

$$
    \mathrm{p}^t=m_0 u^0=m_0 c \frac{d t}{d \tau} \quad \text { y } \quad \mathrm{p}^{\varphi}=m_0 u^{\varphi}=m_0 \frac{d \varphi}{d \tau}
$$

Estas dos cantidades nos permiten formar el resultado,

$$
    \frac{d \varphi}{c d t}=\frac{d \varphi}{d(c t)}=m_0 \frac{d \varphi}{d \tau} \frac{d \tau}{m_0 c d t}=\frac{\mathrm{p}^{\varphi}}{\mathrm{p}^t}=\frac{g^{\varphi t}}{g^{t t}}=\omega(r, \theta)
$$

Aquí, $\omega$ es la velocidad angular de coordenadas que adquiere la partícula a medida que se mueve a lo largo de su geodésica (lo que significa que está cayendo libremente), aunque comenzó desde el reposo lejos de la masa fuente $M$.
El hecho de que esta partícula siga una geodésica significa que la curvatura del espacio-tiempo alrededor de la masa fuente debe estar curvada de tal manera que la partícula adquiera una velocidad angular en la misma dirección que la masa fuente.
Este efecto se llama arrastre de marcos inerciales.
En el caso de los fotones tenemos en algún marco inercial una cuatro-velocidad dada por,

$$
    \underset{\sim}{u}=\frac{d x^\mu}{d \lambda}{\underset{\sim}{\mu}}^e \quad \text { y } \quad \underset{\sim}{u} \cdot \underset{\sim}{u}=0
$$

- El cuatro-momento de este fotón en algún marco inercial arbitrario es,

$$
    \underset{\sim}{p}=\alpha \underset{\sim}{u}=\alpha\left(\frac{c d t}{d \lambda}, \frac{d r}{d \lambda}, \frac{d \theta}{d \lambda}, \frac{d \varphi}{d \lambda}\right)=\left(p^t, p^r, p^\theta, p^{\varphi}\right)
$$

- Cuando nuevamente liberamos un fotón en la dirección de la masa fuente con componente de momento angular inicial cero, aún obtenemos,

$$
    \frac{d \varphi}{c d t}=\frac{d \varphi}{d(c t)}=\alpha \frac{d \varphi}{d \lambda} \frac{d \lambda}{\alpha c d t}=\frac{\mathrm{p}^{\varphi}}{\mathrm{p}^t}=\frac{g^{\varphi t}}{g^{t t}}=\omega(r, \theta)
$$

- Dado que los fotones siguen la curvatura del espacio-tiempo, son una forma de sondear ese espacio-tiempo y determinar el efecto que tiene el objeto rotante con masa M en el espacio-tiempo.
- Nótese que la dirección de la velocidad angular adquirida por la partícula, ya sea con masa o un fotón, está en la misma dirección que la rotación de la masa fuente $M$,

$$
    \frac{d \varphi}{d t}=c \omega(r, \theta)
$$
%%%%%%%%%%%%%%%%%%%%%%%%%%%%%%%%%%%%%%%%%%%%%%%%%%%%%%%%%%

\section{Algoritmo de Newman-Janis y agujeros negros rotantes}