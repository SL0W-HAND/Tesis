
\section[Algoritmo de Newman-Janis]{Algoritmo de Newman-Janis}

El algoritmo de Newman-Janis es una técnica en relatividad general que permite generar métricas típicamente en rotación (lo cual no es siempre el caso) a partir de métricas semilla  axialmente simétricas. En 1965, Ezra T. Newman y Alfred I. Janis \cite{newman-1965}  mediante una transformación de coordenadas complejas aplicada a la métrica de Schwarzschild, obtuvieron la métrica de Kerr, que describe un agujero negro en rotación, además este fue el procedimiento usado para encontrar la métrica de Kerr-Newman\cite{newman-1965b} .

En resumen, el procedimiento del algoritmo que se toman las transformaciones complejas de las coordenadas de Edington-Finkelstein 

\begin{equation}
r \rightarrow r+\mathrm{i} a \cos \theta, \quad u \rightarrow u-\mathrm{i} a \cos \theta,
\end{equation}
Y mediante el uso de tetradas nulas después del proceso de complexificación se reconstruye una nueva métrica. Sin embargo, esto no implica que la métrica resultante sea solución a las ecuaciones de Einstein, por lo que este procedimiento solo genera candidatos a soluciones que posteriormente debe de corroborar.
A pesar de que el algoritmo se ha intentado justificar y formalizar en trabajos posteriores como lo señala \cite{drake-2000}, no se ha llegado a un consenso sobre su validez general.

Este algoritmo es un tanto polémico debido a la introducción de coordenadas complejas, como menciona la review hecha por \cite{drake-2000} mucho lo consideran como un procedimiento ad-hoc y ambiguo ya que existen diferentes maneras de realizar la complexificación como lo discute \cite{azreg-ainou-2014} .

\begin{align}
r^2 & \rightarrow(r+\mathrm{i} a \cos \theta)(r-\mathrm{i} a \cos \theta)=r^2+a^2 \cos ^2 \theta, \\
\frac{1}{r}& \rightarrow \frac{1}{2}\left(\frac{1}{r+\mathrm{i} a \cos \theta}+\frac{1}{r-\mathrm{i} a \cos \theta}\right)=\frac{r}{r^2+a^2 \cos ^2 \theta}, \\
r^2 & \rightarrow r \sqrt{(r+\mathrm{i} a \cos \theta)(r-\mathrm{i} a \cos \theta)}=r \sqrt{r^2+a^2 \cos ^2 \theta} .
\end{align}
Incluso con lo ya mencionado esta es una herramienta extremadamente util en el marco de la relatividad, por lo que para este trabajo se usara el desarrollo hecho por \cite{azreg-ainou-2014} donde se propone un una refinación del método para crear una métrica que satisface las ecuaciones de Einstein, además de no requerir la complexificación tal que incluso puede (en algunos casos) dar la métrica en la forma de Boyer-Lindquist.

La forma de proceder que nos da el articulo es la siguiente:
Se parte de la métrica estática de la forma

\begin{equation}
\footnotemark \mathrm{d} s_{\text {stat }}^2 = - G(r) \mathrm{d} t^2+\frac{\mathrm{d} r^2}{F(r)}+H(r)\left(\mathrm{d} \theta^2+\sin ^2 \theta \mathrm{~d} \varphi^2\right).
\end{equation}
\footnotetext{En este trabajo se uso la signatura $(-,+,+,+)$ para la métrica, por lo que se reemplazó la signatura de la métrica del articulo }

En el caso ideal $G(r)=F(r)$ y $H(r)=r^2$, que por suerte es el caso de la métrica de Schwarzschild y el de algunas soluciones regulares  el  algoritmo refinado de \cite{azreg-ainou-2014} produce la métrica
\begin{align}
\mathrm{d}s^2 = & 
- \left(1 - \frac{2f}{\rho^2}\right) \mathrm{d}t^2 
+ \frac{\rho^2}{\Delta} \mathrm{d}r^2 
- \frac{4af \sin^2\theta}{\rho^2} \mathrm{d}t\, \mathrm{d}\phi 
+ \rho^2 \mathrm{d}\theta^2 
+ \frac{\Sigma \sin^2\theta}{\rho^2} \mathrm{d}\phi^2,
\label{eq:metricGeneratedByNewmanJanis}
\end{align}
con las siguientes definiciones

\begin{align}
\rho^2 & =r^2+a^2 \cos ^2 \theta, \\
 2 f&=r^2(1-F), \\
\Delta & =r^2 F+a^2=r^2-2 f+a^2, \\
\Sigma & =\left(r^2+a^2\right)^2-a^2 \Delta \sin ^2 \theta .
\end{align}
Lo importante de este procedimiento es que como lo menciona el articulo, la métrica resultante es una solución a las ecuaciones de Einstein (ya sea que la métrica semilla sea regular o no) $G_{\mu \nu}=T_{\mu \nu}$, donde $T^{\mu \nu}$ es de la forma
\begin{equation}
T^{\mu \nu}=\epsilon e_t^\mu e_t^\nu+p_r e_r^\mu e_r^\nu+p_\theta e_\theta^\mu e_\theta^\nu+p_\phi e_\phi^\mu e_\phi^\nu
\end{equation}
Donde $e_t^\mu$ es el vector de cuatro-velocidad del fluido, $\epsilon$ es la densidad, $\left(p_r, p_\theta, p_\phi\right)$ son los componentes de la presión, y la base $\left(e_t, e_r, e_\theta, e_\phi\right)$ es dual a la métrica anterior.

\subsection{Aplicación del algoritmo de Newman-Janis sobre la métrica de Schwarzschild}
Dada la métrica de Schwarzschild
\begin{equation}
\mathrm{d} s^2=-\left(1-\frac{2 m}{r}\right) \mathrm{d} t^2+\left(1-\frac{2m}{r}\right)^{-1} \mathrm{d} r^2+r^2\mathrm{d} \Omega,
\end{equation}
podemos hacer las identificaciones directas

\begin{equation}
G(r) = F(r) = 1-\frac{2m}{r}, \quad H(r) = r^2.    
\end{equation}
y proceder a aplicar el algoritmo de Newman-Janis sin complexificación

\begin{equation}
\begin{aligned}
    2f &=  2mr,\\
    \Delta &= r^2 - 2mr + a^2,\\
    \Sigma &= \left(r^2+a^2\right)^2-a^2 \Delta \sin ^2 \theta \\
    & =\left(r^2+a^2\right)^2-a^2 (r^2 - 2mr + a^2) \sin ^2 \theta . 
\end{aligned}    
\end{equation}

Y la métrica de Kerr resultante aplicando directamente (\ref{eq:metricGeneratedByNewmanJanis})
    \begin{align}
        \mathrm{d}s^2 = & 
- \left(1 - \frac{2mr}{\rho^2}\right) \mathrm{d}t^2 
+ \frac{\rho^2}{\Delta} \mathrm{d}r^2 
- \frac{4amr \sin^2\theta}{\rho^2} \mathrm{d}t\, \mathrm{d}\phi 
+ \rho^2 \mathrm{d}\theta^2 
+ \frac{\Sigma \sin^2\theta}{\rho^2} \mathrm{d}\phi^2,
    \end{align}

Dando la métrica de Kerr en coordenadas de Boyer-Lindquist

\begin{definition}{Métrica de Kerr en coordenadas de Boyer-Lindquist}{}
Donde los términos son definidos como
\begin{equation}
        g_{\mu \nu}=\left( 
            \begin{array}{cccc}
               -\left(1-\frac{2 mr}{\rho^2}\right) &0&0& - \frac{2amr \sin^2\theta}{\rho^2} \\
               0& \frac{\rho^2}{\Delta} & 0 & 0 \\
                0& 0 & \rho^2 & 0 \\
                -\frac{2amr \sin^2\theta}{\rho^2} & 0 & 0 & \frac{\Sigma \sin^2\theta}{\rho^2}
            \end{array}
        \right).
        \label{eq:covariantKerrMetric}
    \end{equation}
Con las siguientes definiciones

\begin{equation}
    \begin{aligned}
        \Delta &= r^2 - 2mr + a^2\\
    \Sigma &= \left(r^2+a^2\right)^2-a^2 \Delta \sin ^2 \theta  \\
    &=\left(r^2+a^2\right)^2-a^2 (r^2 - 2mr + a^2) \sin ^2 \theta ,\\
    \rho & = r^2+a^2 \cos ^2 \theta.
\end{aligned}
\end{equation}
\end{definition}

\section{Regiones de la métrica de Kerr}



\subsection{Geodésicas en Kerr}
Se uso el algoritmo que aparece en los anexos para calcular los símbolos de Christoffel no cero de la métrica de Kerr

