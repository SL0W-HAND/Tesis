\section{Solución de Kerr}
En el caso de un agujero negro en rotación (métrica de Kerr), el espacio-tiempo fuera de él es estacionario, y se puede elegir un marco de referencia independiente del tiempo que asintóticamente tiende al marco de Lorentz en el infinito. Las coordenadas propuestas por Boyer y Lindquist (1967) representan dicho marco de referencia. La métrica de Kerr en estas coordenadas es:

$$
d s^2=-\left(1-\frac{2 M r}{\Sigma}\right) d t^2-\frac{4 M r a \sin ^2 \theta}{\Sigma} d t d \phi+\frac{\Sigma}{\Delta} d r^2+\Sigma d \theta^2+\frac{A \sin ^2 \theta}{\Sigma} d \phi^2,(3.2 .1)
$$

donde

$$
\begin{aligned}
& \Sigma \equiv r^2+a^2 \cos ^2 \theta, \quad \Delta \equiv r^2-2 M r+a^2 \\
& A=\left(r^2+a^2\right)^2-a^2 \Delta \sin ^2 \theta
\end{aligned}
$$

\section{Algoritmo de Newman-Janis y agujeros ne-
gros rotantes}