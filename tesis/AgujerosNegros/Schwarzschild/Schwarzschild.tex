\section{Solución de Schwarzschild}
\label{sec:solucionSchwarzschild}
\noindent La métrica de Schwarzschild fue la primera solución analítica a las ecuaciones de campo de Einstein. Esta solución comprende el caso mas sencillo posible, el de un objeto esféricamente simétrico y no rotante (Se puede ver la una traducción de la propuesta original en \cite{schwarzschild1999gravitationalfieldmasspoint}, en este texto haremos una derivación inspirada en \cite{eigenchris-2021}).
\begin{itemize}
    \item Tomamos un universo estático y esféricamente simétrico, es decir, un universo que no cambia con el tiempo y que tiene la misma forma en todas las direcciones.
    \item Se usan coordenadas esféricas $(t,r,\theta,\phi)$.
    \item Se Toma una masa puntual M en el origen de coordenadas.
    \item Se asume que no hay materia en el espacio-tiempo, es decir, que el tensor de energía-momento es cero.
    \item lejos de la masa puntual, el espacio-tiempo debe ser plano, es decir, la métrica debe ser la métrica de Minkowski.
\end{itemize}
Al saber que usaremos una simetría esférica, la métrica de Minkowski pasa de las componentes cartesianas a las componentes esféricas, es decir, la métrica de Minkowski en coordenadas esféricas es:
\begin{equation}
    \eta_{\mu \nu}=\left(\begin{array}{cccc}
            -1 & 0 & 0 & 0 \\
            0  & 1 & 0 & 0 \\
            0  & 0 & 1 & 0 \\
            0  & 0 & 0 & 1
        \end{array}\right)=\left(\begin{array}{cccc}
            -1 & 0 & 0     & 0                      \\
            0  & 1 & 0     & 0                      \\
            0  & 0 & r^{2} & 0                      \\
            0  & 0 & 0     & r^{2} \sin ^{2} \theta
        \end{array}\right)
\end{equation}
y
\begin{equation}
    \eta^{\mu \nu}=\left(\begin{array}{cccc}
            -1 & 0 & 0 & 0 \\
            0  & 1 & 0 & 0 \\
            0  & 0 & 1 & 0 \\
            0  & 0 & 0 & 1
        \end{array}\right)=\left(\begin{array}{cccc}
            -1 & 0 & 0               & 0                                \\
            0  & 1 & 0               & 0                                \\
            0  & 0 & \frac{1}{r^{2}} & 0                                \\
            0  & 0 & 0               & \frac{1}{r^{2} \sin ^{2} \theta}
        \end{array}\right),
\end{equation}
Bajo las condiciones de simetría esférica y estática, se imponen los  vectores de Killing $\v{\xi_1 }= \v{e}_t$ ,$\v{\xi_2 }= \v{e}_\theta$ y $\v{\xi_3} = \v{e}_\phi$ .
La implicación de estos Killings 
\begin{align}
\pdi{t}{g_{\mu \nu}} =\pdi{\theta}{g_{\mu \nu}}= \pdi{\phi}{g_{\mu \nu}} = 0 
\end{align}
nos dice por mera eliminación que la métrica solo dependerá de la variable $r$, recordando la situación para una masa puntual en un universo vacío e isotrópico, así que es de esperarse que la métrica no dependa de las coordenadas $\theta$ y $\phi$.


Formalmente, una solución es estática en el tiempo si cumple dos condiciones,
\begin{enumerate}
    \item  Es \textbf{estacionaria}. Implica la invarianza bajo traslaciones en el tiempo, es decir la reversión temporal \( t \to -t \) deja invariante la métrica:
          \[ g_{\mu\nu}(t) = g_{\mu\nu}(-t). \]
Es fácil ver que para términos cruzados mediante las uno formas del elemento de linea
\begin{equation}
\mathbf{dt}\mathbf{dx^i} \neq \mathbf{-dt}\mathbf{dx^i}     
\end{equation}
pero
\begin{equation}
\mathbf{dt}\mathbf{dt} = (\mathbf{-dt})(\mathbf{-dt}) .    
\end{equation}
No hay términos cruzados entre las coordenadas espaciales y temporales, $ g_{ti} = 0 \quad \forall i \in \{r, \theta, \phi\}$ , la métrica en coordenadas de tipo \((t, r, \theta, \phi)\) adopta la forma general. 
\begin{equation}
    g_{\mu \nu} =
    \begin{bmatrix}
        g_{tt} & 0            & 0                & 0              \\
        0      & g_{rr}       & g_{r\theta}      & g_{r\phi}      \\
        0      & g_{\theta r} & g_{\theta\theta} & g_{\theta\phi} \\
        0      & g_{\phi r}   & g_{\phi\theta}   & g_{\phi\phi}
    \end{bmatrix}.
\end{equation}
    \item Es \textbf{irrotacional}.Dadas las simetrías angulares angulares implican que la métrica debe de ser invariante bajo cambios de coordenadas angulares ($\phi \rightarrow  - \phi, \theta \rightarrow  - \theta$).De manera similar con las uno formas se puede ver que

\begin{equation}
\mathbf{dx^i}\mathbf{dx^j} \neq -\mathbf{dx^i}\mathbf{dx^j}, \quad (i \neq j ) . 
\end{equation}

Y los componentes cruzados de la métrica deben ser cero, haciendo la métrica diagonal. 
\begin{equation}
    g_{\mu \nu} =
    \begin{bmatrix}
        g_{tt} & 0            & 0                & 0              \\
        0      & g_{rr}       &   0    &  0  \\
        0      & 0 & g_{\theta\theta} &0 \\
        0      &  0  & 0 & g_{\phi\phi}
    \end{bmatrix}.
\end{equation}
\end{enumerate}



Debido a que estamos usando una simetría esférica y que la métrica es diagonal,la métrica de la sub variedad debe ser asintótica a la descrita por una esfera de radio $r$ \footnote{La aparición de un termino $(\sin \theta)^2$, no contradice que la métrica dependa solo de $r$, ya que esta función es una parte fija de la estructura de la 2-esfera}, y le podemos multiplicar una función  escalar radial $C(r)$.
\begin{equation}
    \begin{array}{ll}
        g_{\theta \theta} & g_{\theta \phi} \\
        g_{\phi \theta}   & g_{\phi \phi}
    \end{array} = \left[\begin{array}{cc}
            C(r) r^2 & 0                       \\
            0        & C(r) r^2(\sin \theta)^2
        \end{array}\right].
\end{equation}


Los componentes desconocidos serán entonces 3 funciones dependientes de $r$
\begin{equation}
    \left[\begin{array}{cccc}
            -A(r) & 0    & 0        & 0                       \\
            0     & B(r) & 0        & 0                       \\
            0     & 0    & C(r) r^2 & 0                       \\
            0     & 0    & 0        & C(r) r^2(\sin \theta)^2
        \end{array}\right].
        \label{eq:genericMetric}
\end{equation}
Mediante cambio de variable 

\begin{equation}
    \tilde{r}=\sqrt{C(r)}  r,
\end{equation}
que permite escribir la métrica \ref{eq:genericMetric} como 
\begin{equation}
    \left[\begin{array}{cccc}
            -A(\tilde{r}) & 0    & 0        & 0                       \\
            0     & B(\tilde{r}) & 0        & 0                       \\
            0     & 0    & \tilde{r}^2 & 0                       \\
            0     & 0    & 0        & \tilde{r}^2(\sin \theta)^2
        \end{array}\right].
        \label{eq:genericmetric2}
\end{equation}
Aprovechando el carácter tensorial que hace la métrica sea valida independiente de la elección de coordenadas  $\tilde{r}$ la escribiré como $r$, esto se hace sin perdida de la interpretación de las coordenadas esféricas puesto que $\tilde{r}$ es una función monótona creciente de $r$.


La forma de proceder va a ser  calcular los Christoffel para después llegar al tensor de Ricci que debe de cumplir las ecuaciones de vacío (\ref{vacuumFieldEquations})
\begin{equation*}
    g_{\mu \nu} \rightarrow \Gamma_{\mu \nu}^\sigma \rightarrow R_{\mu \nu}.
\end{equation*}
Los símbolos de Christoffel los obtenemos a partir de 

\begin{equation}
    \Gamma_{\mu \nu}^\sigma=\frac{1}{2} g^{\sigma \rho}\left(\partial_\nu g_{\mu \rho}+\partial_\mu g_{\nu \rho}-\partial_\rho g_{\mu \nu}\right).
\end{equation}
El calculo se realizo con el programa del anexo \ref{chap:programa_christoffel} y se encontraron los Christoffel no cero 
\begin{equation}
\begin{aligned}
    &\Gamma_{01}^0=\Gamma_{10}^0=\frac{1}{2} \frac{1}{A}\left(\partial_r A\right),\\ 
    &\Gamma_{00}^1=\frac{1}{2} \frac{1}{B}\left(\partial_r A\right),\\
    &\Gamma_{11}^1=\frac{1}{B}\left(\partial_r B\right),\\
    &\Gamma_{22}^1=-\frac{r}{B},\\
    &\Gamma_{33}^1=-\frac{r(\sin \theta)^2}{B},\\
    &\Gamma_{12}^2=\Gamma_{21}^2=\Gamma_{13}^3=\Gamma_{31}^3=\frac{1}{r},\\
    &\Gamma_{33}^2=-\sin \theta \cos \theta,\\
    &\Gamma_{23}^3=\Gamma_{32}^3=\cot \theta
\end{aligned}
\end{equation}

Para calcular el tensor de Ricci $R_{\mu \nu}$ se obtiene contrayendo el tensor de Riemann,
\begin{equation}
    R_{\mu \nu}=R_{\mu \alpha \nu}^\alpha=\partial_\alpha \Gamma_{\mu \nu}^\alpha-\partial_\nu \Gamma_{\mu \alpha}^\alpha+\Gamma_{\alpha \lambda}^\alpha \Gamma_{\mu \nu}^\lambda-\Gamma_{\nu \lambda}^\alpha \Gamma_{\mu \alpha}^\lambda .
    \label{eq:ricci_tensor}
\end{equation}
El calculo se hizo con el programa del anexo \ref{chap:programa_ricci}, con un poco de refinamiento manual, se obtuvieron los componentes del tensor de Ricci 

\begin{equation}
    \begin{array}{l}
    R_{00}=2 r A B A^{\prime \prime}-r A A^{\prime} B^{\prime}+4 A B A^{\prime}-r B\left(A^{\prime}\right)^2=0 ,\\
    R_{11}=-2 r A B A^{\prime \prime}+r B\left(A^{\prime}\right)^2+r A A^{\prime} B^{\prime}+4 A^2 B^{\prime}=0, \\
    R_{22}=-2 A B+2 A B^2-r A^{\prime} B+r A B^{\prime}=0.
    \end{array}
    \end{equation}
Si sumamos los componentes $R_{00} + R_{11} = 0 $

\begin{equation}
    \begin{aligned}
        4 A B A^{\prime} + 4 A^2 B^{\prime} = 0 \\
        B A^{\prime}+A B^{\prime}=0 \\
        \partial_r(A B)=0 \\
        \Rightarrow A B=K
    \end{aligned}
\end{equation}
donde \( K \) es una constante de integración. Para determinar \( K \), consideramos el límite cuando \( r \to \infty \):
\begin{equation}
    \begin{aligned}
         & \lim_{ r \to \infty} A(r) = 1, \\
         & \lim_{ r \to \infty} B(r) =   1.
    \end{aligned}
\end{equation}
$A(r)$ y $B(r)$ son funciones que tienden a 1 cuando \( r \to \infty \), ya que la métrica en este caso debe de ser asintótica  a Minkowski . Por lo tanto, \( K = 1 \) y

\begin{equation}
    \Rightarrow B(r)=\frac{1}{A(r)} \text { para todo } r, \quad  B^{\prime}=\partial_r\left(A^{-1}\right)=-\frac{A^{\prime}}{A^2}.
\end{equation}
Con esta información es posible sustituir \( B(r) \) en la ecuación \( R_{22} = 0 \)
\begin{equation}
    \begin{aligned}
         & R_{22}=-2 A B+2 A B^2-r A^{\prime} B+r A B^{\prime}   = 0                                                           \\
         &-2 A \frac{1}{A}+2 A\left(\frac{1}{A}\right)^2-r A^{\prime} \frac{1}{A}+r A\left(-\frac{A^{\prime}}{A^2}\right)=0\\
         & -2 + 2 \frac{1}{A}-\frac{r A^{\prime}}{A}- \frac{r A^{\prime}}{A}=0 \\
            & -2 + \frac{2}{A} - \frac{2r A^{\prime}}{A}=0 \\
            &\Rightarrow  A^{\prime} + \frac{A}{r}=\frac{1}{r}.
    \end{aligned}
    \label{eq:ricci22}
\end{equation}
Resolviendo la ecuación anterior por factor integrante,
\begin{equation}
    \begin{aligned}
        A &= e^{- \int 1/r d r} \left[\int \left(\frac{1}{r}e^{\int 1/r dr } \right)dr - k\right]\\
        &= e^{- \ln r} \left[\int \frac{1}{r} e^{\ln r} dr   - k\right] \\
        &=\frac{1}{r}\left[\int dr - k\right]\\
        &= 1 - \frac{k}{r}. \\
    \end{aligned}
\end{equation}
Inmediatamente se desvela que $B=\frac{1}{1 - \frac{k}{r}}  $,por lo que la métrica queda como:
\begin{equation}
    g_{\mu \nu} =\left[\begin{array}{cccc}
        -\left(1-\frac{k}{r}\right) & 0                                & 0    & 0                   \\
            0             & \left(1-\frac{k}{r}\right)^{-1} & 0    & 0                   \\
            0             & 0                                & r^2 & 0                   \\
            0             & 0                                & 0    & r^2(\sin \theta)^2
        \end{array}\right]
\end{equation}
Por ultimo debemos de tomar la aproximación de campo débil en el limite newtoniano para obtener la constante de integración \( k \) (véase \cite[148-149]{ryder-2009} )
\begin{equation}
    g_{\mu \nu} =  \eta_{\mu \nu}+ h_{\mu \nu} .
\end{equation}
Centrémonos en el componente \( g_{tt} \) donde \( h_{tt} = -\frac{-2 \mathbf{\Phi}}{c^2}  = \frac{2MG}{rc^2}\) 
\begin{equation}
    g_{tt} = -1 + \frac{k}{r} = \eta_{tt} + h_{tt} = -1 + \frac{2GM}{rc^2}
\end{equation}
Donde $k = \frac{2GM}{c^2}$.
\begin{definition}{Metrica de Schwarzschild}{}
La métrica de Schwarzschild es una solución exacta a las ecuaciones de campo de Einstein en el vacío, que describe el campo gravitacional de un objeto esféricamente simétrico y no rotante. La métrica se expresa como:
\begin{equation}
    g_{\mu \nu}=\left(\begin{array}{cccc}
                -1+\frac{2 G M}{r c^2} & 0                                       & 0   & 0                  \\
                0                      & \left(1-\frac{2 G M}{r c^2}\right)^{-1} & 0   & 0                  \\
                0                      & 0                                       & r^2 & 0                  \\
                0                      & 0                                       & 0   & r^2 \sin ^2 \theta
            \end{array}\right)
            \label{eq:schwarzschild_metric}
\end{equation}
\end{definition}
Observe que la métrica  \ref{eq:schwarzschild_metric} diverge ``aparentemente'' cuando \( r = 2GM/c^2 \equiv r_s\), este punto se conoce como \textbf{radio de Schwarzschild},  y se denomina entonces que cuando el radio del cuerpo es menor a  $r_s $,la region que corresponde a $r<r_s $ la métrica describe la geometría de un agujero negro, y $r_s$ es la  frontera del horizonte de eventos del agujero negro. También es usual encontrase con este punto definido como $r_s = 2GM/c^2 =2m $, la ultima notación ($2m$) se encuentra cuando se trabaja con unidades naturales pero aquí  nos referiremos a  $m = GM/c^2$ como un simple cambio de variable.  

\subsection{Geodésicas en Schwarzschild tipo  nulas}

\noindent El movimiento de los cuerpos y de la luz en el espacio-tiempo de Schwarzschild está dado por la ecuación geodésica
\begin{equation}
    \frac{\mathrm{d}^2 x^\mu}{\mathrm{d} \lambda^2}+\Gamma^\mu{ }_{\nu \rho} \frac{\mathrm{~d} x^\nu}{\mathrm{d} \lambda} \frac{\mathrm{~d} x^\rho}{\mathrm{d} \lambda}=0.
    \label{eq:geodesic}
\end{equation}
Con $x^0 = ct$, $x^1 = r$, $x^2 = \theta$, $x^3 = \phi$ y $\lambda$ es un parámetro afín.

Los símbolos de Christoffel no cero son
\begin{equation}
        \begin{array}{l}
        \Gamma^0{ }_{10}=\Gamma^0{ }_{01}=\dfrac{G M}{c^2 r\left(r - \dfrac{2 G M}{c^2}\right)},                                                                                                                       \\
        \Gamma^1{ }_{11}=-\dfrac{G M}{c^2 r\left(r - \dfrac{2 G M}{c^2}\right)}, \quad \Gamma^1{ }_{22}=-\left(r - \dfrac{2 G M}{c^2}\right), \quad \Gamma^1{ }_{33}=-\left(r - \dfrac{2 G M}{c^2}\right) \sin \theta, \\
        \Gamma^2{ }_{12}=\Gamma^2{ }_{21}=\Gamma^3{ }_{13}=\Gamma^3{ }_{31}=\dfrac{1}{r}, \quad \Gamma^2{ }_{33}=-\sin \theta \cos \theta,                                                                             \\
        \Gamma^3{ }_{23}=\Gamma^3{ }_{32}=\cot \theta.
    \end{array}
\end{equation}

Particularmente nos interesa el caso de geodésicas nulas, es decir, aquellas que describen la trayectoria de la luz en el espacio-tiempo. Además se considera $\theta=\pi/2$ y $\phi=0$, para una simplificación de los cálculos (Si se desea ver las demás trayectorias solo hay que remplazar los valores en la ecuación \ref{eq:geodesic}).

Nos centraremos únicamente en las que nos interesan para describir la luz. Considere  la ecuación geodésica con $\mu=0$ es

\begin{equation}    
    \frac{\mathrm{d}^2 t}{\mathrm{d} \lambda^2}+\dfrac{2 G M}{c^2 r\left(r - \dfrac{2 G M}{c^2}\right)} \frac{\mathrm{d} t}{\mathrm{d} \lambda} \frac{\mathrm{d} r}{\mathrm{d} \lambda}=0
\end{equation}

ó
\begin{equation}
    \frac{\mathrm{d}}{\mathrm{~d} \lambda}\left[\left(1-\dfrac{2 G M}{c^2 r}\right) \frac{\mathrm{d} t}{\mathrm{~d} \lambda}\right]=0,
\end{equation}
lo cual se integra para dar
\begin{equation}
    \left(1-\dfrac{2 G M}{c^2 r}\right) \frac{\mathrm{d} t}{\mathrm{~d} \lambda}=b=\text { const. }
\end{equation}
La ecuación de geodésica tipo luz es una relación siempre valida, pero esta anclada a un parámetro $\lambda$, si queremos hacer comparaciones con otras variables, es util usar el elemento de linea ``light-like'' $ds = 0$ que describe precisamente a la luz  en particular aquí
\begin{equation}
    \begin{aligned}
        \mathrm{d} s^2 & =g_{\mu \nu} \mathrm{d} x^\mu \mathrm{d} x^\nu                                                                      \\
                       & =-\left(1-\dfrac{2 G M}{r c^2}\right)c^2 \mathrm{d} t^2+\left(1-\dfrac{2 G M}{r c^2}\right)^{-1} \mathrm{d} r^2 = 0
    \end{aligned}
\end{equation}
usando la relación de la geodésica tipo luz 
\begin{equation}
    \left(1-\frac{2 G M}{r c^2}\right)^{-1} \frac{dt}{d\lambda}  = b \rightarrow  \left(\frac{dt}{d\lambda}\right)^2 = b^2 \left(1-\frac{2 G M}{r c^2}\right)^{-2},
\end{equation}
que implica  entonces
\begin{equation}
    \frac{d r }{d \lambda}= \pm cb.
\end{equation}
Podemos relacionar $r$ con $t$  de la siguiente manera
\begin{equation}
    \frac{\frac{dr}{d\lambda}}{\frac{dt}{d\lambda}} =   \frac{dr}{dt} =  \frac{\pm cb }{ b \left(1-\frac{2 G M}{r c^2}\right)^{-1}} = \pm c \left(1-\frac{2 G M}{r c^2}\right).
\end{equation}
Resolviendo esta ultima ecuación
\begin{equation}
    ct = \pm \left(r + \frac{2 G M}{c^2} \ln \abs{r -\frac{2 G M}{c^2} } + k \right)
\end{equation}
o en términos de el radio de Schwarzschild $r_s = \frac{2 G M}{c^2}$
\begin{equation}
    ct = \pm \left(r + r_s \ln\abs{r - r_s} + k \right)
    \label{eq:lightRaysSchwarzschild}
\end{equation}
donde el signo $+$ es para geodésicas salientes y el signo $-$ es para geodésicas entrantes.
\begin{figure}[H]
    \begin{small}
        \begin{center}
            \includegraphics[width=0.99\textwidth]{AgujerosNegros/Schwarzschild/media/images/rayos_Luz_Schwarzschild_ManimCE_v0.19.0.png}
        \end{center}
        \caption{Gráfica $ct-r $ representando las trayectorias tipo luz, las salientes, son representadas de color azul y las entrantes en color rojo }
        \label{fig:lightraysSchwarzschild}
    \end{small}
\end{figure}

\subsection{Coordenadas Eddington-Finkelstein}
Las coordenadas de Eddington-Finkelstein son un sistema de coordenadas para la solución de Schwarzschild cuya idea principal es usar las trayectorias de la luz (geodésicas nulas radiales) para definir las coordenadas temporal y radial. Esto es posible debido a la simetría radial, donde \( \theta, \varphi = \text{cte} \implies d\Omega^2 = 0 \).
Partimos de la trayectoria de luz \ref{eq:lightRaysSchwarzschild} que esta en términos de la coordenada radial \( r \) y el tiempo \( ct \) e introducimos la nueva coordenada \( c\tilde{t} \), normalmente llamada coordenada temporal avanzada
\begin{equation}
        c\tilde{t}  = ct + r_s \ln \left| \frac{r}{r_s} - 1 \right|,      \Rightarrow 
        ct         = c\tilde{t} - r_s \ln \left| \frac{r}{r_s} - 1 \right|.
    \label{eq:ctVariables}
\end{equation}
Combinando la nueva coordenada con la ecuación de las geodésicas entrantes (\( - \)) en (\ref{eq:lightRaysSchwarzschild})
\begin{equation}
    c\tilde{t} - r_s \ln \left| \frac{r}{r_s} - 1 \right| = -r - r_s \ln \left| \frac{r}{r_s} - 1 \right| + k,
\end{equation}
simplificando 
\begin{equation}
    c\tilde{t} = -r + k  .
\label{eq:ctVariablesIn}
\end{equation}
Si tomamos el signo \( + \) de la ecuación de luz saliente (\( + \)) en (\ref{eq:lightRaysSchwarzschild}), y hacemos el mismo procedimiento de introducir el cambio de variable (\ref{eq:ctVariables}) para las geodésicas salientes
\begin{equation}
    c\tilde{t} = r + 2 r_s \ln \left| \frac{r}{r_s} - 1 \right| + k.
\end{equation}

\begin{figure}[H]
    \begin{small}
        \begin{center}
            \includegraphics[width=0.95\textwidth]{AgujerosNegros/Schwarzschild/media/images/EddingtonFinkelsteinIngoingLight_ManimCE_v0.19.0.png}
        \end{center}
        \caption{Geodésicas nulas entrantes y salientes en coordenadas avanzadas}
        \label{fig:EddingtonFinkelsteinInLight}
    \end{small}
\end{figure}
La idea base de las coordenadas de Eddington-Finkelstein es que las geodésicas nulas nos permiten definir un nuevo sistema de coordenadas que no diverja en el horizonte de eventos(aquí hay que notar que aunque la coordenada avanzada $c\tilde{t}$ nos dejo hacer que las geodésicas tipo luz pasen sin problema su definición sigue presentando una divergencia en $r_s$), ya que estas pueden cruzarle sin problemas como se ve en la figura \ref{fig:EddingtonFinkelsteinInLight}.Al montar la nueva coordenada \( v \) sobre las geodésicas entrantes tipo luz con valores constantes \ref{eq:ctVariablesIn} haciendo la asignación a la nueva coordenada como  $v = k$, se define

\begin{equation}
    \begin{aligned}
        v &= c\tilde{t} + r ,\\
        v  & = ct + r + r_s \ln \left| \frac{r}{r_s} - 1 \right|, \\
        ct & = v - r - r_s \ln \left| \frac{r}{r_s} - 1 \right|.
    \end{aligned}
\end{equation}
Una vez que hemos hecho la asignación para los rayos entrantes, se calcula cual es la trayectoria de los rayos salientes usando la ecuación anterior y la ecuación(\ref {eq:lightRaysSchwarzschild})(+),
\begin{equation}
    \begin{aligned}
        ct = v - r - r_s \ln \left| \frac{r}{r_s} - 1 \right|= r + r_s \ln \left| \frac{r}{r_s} - 1 \right| + k, \\
        v = 2 \left(r + r_s \ln \left| \frac{r}{r_s} - 1 \right|\right) +k.
    \end{aligned}
\end{equation}

\begin{figure}[H]
    \begin{small}
        \begin{center}
            \includegraphics[width=0.95\textwidth]{AgujerosNegros/Schwarzschild/media/images/EddingtonFinkelsteinVR_ManimCE_v0.19.0.png}
        \end{center}
        \caption{Diagrama de los rayos de luz en coordenadas de Eddington-Finkelstein. Las líneas azules representan los rayos entrantes y las rojas los salientes. Las nuevas coordenadas no ortogonales \(v\) y \(r_{\text{in}}\) están definidas en función de las originales \(v = ct + r + r_s \log \left| \frac{r}{r_s} - 1 \right|\) y \(r_{\text{in}} = r\) .}
    \end{small}
\end{figure}

\subsubsection{Métrica de Schwarzschild en coordenadas de Eddington-Finkelstein}
La métrica se obtiene expresando los vectores base \( e_v \) y \( e_{r_{\text{in}}} \) en términos de las bases originales \( e_{ct} \) y \( e_r \). 
Los vectores base transforman como sigue
\begin{equation}
    \begin{aligned}
        \v{e_v}               & = \frac{\partial ct}{\partial v} \v{e_{ct}} + \frac{\partial r}{\partial v} \v{e_r} = \v{e_{ct}},                                                            \\
        \v{e_{r_{\text{in}}} }& = \frac{\partial ct}{\partial r_{\text{in}}} \v{e_{ct}} + \frac{\partial r}{\partial r_{\text{in}}} \v{e_r} = -\frac{1}{1 - \frac{r_s}{r}} \v{e_{ct}} + \v{e_r}.
    \end{aligned}
\end{equation}

Los componentes de la métrica \( g_{\mu\nu} = \v{e_\mu} \cdot \v{e_\nu} \) son:

\begin{equation}
    \begin{aligned}
        g_{vv}          & = \v{e_v} \cdot \v{e_v} = -\left(1 - \frac{r_s}{r}\right), \\
        g_{vr} = g_{rv} & = \v{e_v} \cdot \v{e_{r_{\text{in}}}} = 1,                 \\
        g_{rr}          & = \v{e_{r_{\text{in}}}} \cdot \v{e_{r_{\text{in}}}} = 0.
    \end{aligned}
\end{equation}
Por lo tanto, la métrica en coordenadas de Eddington-Finkelstein es 
\begin{equation}
    ds^2 = -\left(1 - \frac{r_s}{r}\right) dv^2 + 2 dv dr + r^2 d\Omega^2,
\end{equation}
nótese que bajo este nuevo esquema de coordenadas la métrica no se indetermina en el horizonte de eventos \( r = r_s \).  
\subsubsection{Coordenadas salientes de Eddington-Finkelstein}
De forma análoga a como hicimos el cambio de coordenadas para tener los rayos de luz entrantes con valores constantes, para geodésicas nulas salientes (\(+\)) de la ecuación (\ref{eq:lightRaysSchwarzschild}) introducimos la coordenada \( u \):
\begin{equation}
        u  = ct - r - r_s \log \left| \frac{r}{r_s} - 1 \right|.
\end{equation}
Para rayos salientes (\(+\)) de forma completamente análoga al procedimiento anterior \(u = k\).
Y para los entrantes (\(-\)):
\begin{equation}
    \begin{aligned}
        u + r + r_s \log \left| \frac{r}{r_s} - 1 \right| = -r - r_s \log \left| \frac{r}{r_s} - 1 \right| + k, \\
        u = -2\left(r + r_s \log \left| \frac{r}{r_s} - 1 \right|  \right) +k.
    \end{aligned}
\end{equation}
\begin{figure}[H] 
    \centering 
    \begin{subfigure}{0.48\textwidth} 
        \includegraphics[width=\linewidth]{AgujerosNegros/Schwarzschild/media/images/EddingtonFinkelsteinOutgoingLight_ManimCE_v0.19.0.png}
        \caption{Rayos de luz con coordenada temporal atrasada $ c \tilde{t} = ct - r_s \ln \left| \frac{r}{r_s} - 1 \right|$.}
    \end{subfigure}
    \hfill 
    \begin{subfigure}{0.48\textwidth} 
        \includegraphics[width=\linewidth]{AgujerosNegros/Schwarzschild/media/images/EddingtonFinkelsteinUR_ManimCE_v0.19.0.png} 
        \caption{Rayos de luz en coordenadas de Eddington-Finkelstein salientes.}
    \end{subfigure}
    \caption{Se muestra la gráfica de los rayos de luz en coordenadas avanzadas y las nuevas coordenadas no ortogonales que están definidas como $u = ct - r - r_s \log \left| \frac{r}{r_s} - 1 \right|, \quad r_{\text{out}} = r$.}
\end{figure}

\subsubsection{Métrica en coordenadas salientes}
El calculo en coordenadas salientes se realiza de manera análoga al caso entrante.
\begin{equation}
    \begin{aligned}
        \v{e_u   }             & = \frac{\partial ct}{\partial u} \v{e_{ct}} + \frac{\partial r}{\partial u} \v{e_r} = \v{e_{ct}},                                                             \\
        \v{e_{r_{\text{out}}}} & = \frac{\partial ct}{\partial r_{\text{out}}} \v{e_{ct}} + \frac{\partial r}{\partial r_{\text{out}}} \v{e_r} = \frac{1}{1 - \frac{r_s}{r}} \v{e_{ct}} + \v{e_r}.
    \end{aligned}
\end{equation}

Componentes de la métrica \( g_{\mu\nu} = \v{e_\mu} \cdot \v{e_\nu} \):
\begin{equation}
    \begin{aligned}
        g_{uu}          & = \v{e_u} \cdot \v{e_u} = -\left(1 - \frac{r_s}{r}\right), \\
        g_{ur} = g_{ru} & = \v{e_u} \cdot \v{e_{r_{\text{out}}}} = -1,               \\
        g_{rr}          & = \v{e_{r_{\text{out}}}} \cdot \v{e_{r_{\text{out}}}} = 0.
    \end{aligned}
\end{equation}

La métrica en coordenadas salientes es:
\begin{equation}
    ds^2 = -\left(1 - \frac{r_s}{r}\right) du^2 - 2 du dr + r^2 d\Omega^2.
\end{equation}

\subsubsection{Propiedades de las coordenadas de Eddington-Finkelstein}
El cambio de coordenadas permite mostrar que las partículas de luz/masivas puedan pasar a través del horizonte de eventos, pero...
\begin{itemize}
    \item En coordenadas de EF entrantes: los haces de luz salientes no funcionan en $r_s$.
    \item En coordenadas de EF salientes: los haces de luz entrantes no funcionan en $r_s$.
\end{itemize}


\subsection{Coordenadas Kruskal-Szekeres}
La idea detrás de las coordenadas Kruskal-Szekeres es extender las coordenadas de Eddington-Finkelstein manteniendo los rayos de luz salientes y entrantes con pendiente $\pm 1$ obteniendo las ventajas de los diagramas de coordenadas avanzadas y retrasadas.
\subsubsection{Coordenadas nulas originales}
Con las coordenadas entrante (\(v\)) y saliente (\(u\)) de Eddington-Finkelstein:
\begin{equation}
    \begin{aligned}
        v   & = ct + r^*                                      \\
        u   & = ct - r^*                                      \\
        r^* & \equiv r + r_s \log\left|\frac{r}{r_s}-1\right|
    \end{aligned}
\end{equation}
y las coordenadas \( (r, t) \) como 
\begin{equation}
\frac{v+ u}{2} = ct, \quad \frac{v- u}{2} = r^*.
\end{equation}


\subsubsection{Vectores base en coordenadas \( (u, v) \)}
Expresamos los vectores base en términos de \( e_{ct} \) y \( e_r \):
\begin{equation}
    \begin{aligned}
        \v{e_v} & = \frac{\partial ct}{\partial v}\v{e_{ct}} + \frac{\partial r}{\partial v}\v{e_r} = \frac{1}{2}\v{e_{ct}} + \frac{1}{2}\left(1-\frac{r_s}{r}\right)^{-1}\v{e_r} \\
        \v{e_u} & = \frac{\partial ct}{\partial u}\v{e_{ct}} + \frac{\partial r}{\partial u}\v{e_r} = \frac{1}{2}\v{e_{ct}} - \frac{1}{2}\left(1-\frac{r_s}{r}\right)^{-1}\v{e_r}
    \end{aligned}
\end{equation}

\subsubsection{Productos punto fundamentales}
Usando la métrica de Schwarzschild \( g(e_{ct},e_{ct}) = -\left(1-\frac{r_s}{r}\right) \) y \( g(e_r,e_r) = \left(1-\frac{r_s}{r}\right)^{-1} \):
\begin{equation}
    \begin{aligned}
        e_v \cdot e_v & = \left(\frac{1}{2}\right)^2 g(e_{ct},e_{ct}) + \left(\frac{1}{2}\left(1-\frac{r_s}{r}\right)^{-1}\right)^2 g(e_r,e_r) = 0                                        \\
        e_u \cdot e_u & = \left(\frac{1}{2}\right)^2 g(e_{ct},e_{ct}) + \left(\frac{1}{2}\left(1-\frac{r_s}{r}\right)^{-1}\right)^2 g(e_r,e_r) = 0                                        \\
        e_v \cdot e_u & = \left(\frac{1}{2}\right)^2 g(e_{ct},e_{ct}) - \left(\frac{1}{2}\left(1-\frac{r_s}{r}\right)^{-1}\right)^2 g(e_r,e_r) = -\frac{1}{2}\left(1-\frac{r_s}{r}\right)
    \end{aligned}
\end{equation}

\subsubsection{Métrica en coordenadas nulas}
La métrica se expresa como:
\begin{equation}
    ds^2 = 2(e_v \cdot e_u) du\, dv + r^2 d\Omega^2 = -\left(1-\frac{r_s}{r}\right) du\, dv + r^2(d\theta^2 + \sin^2\theta d\varphi^2)
\end{equation}

\subsubsection{Transformación a Kruskal-Szekeres}

\begin{equation}
    \begin{aligned}
        \frac{v-u}{2}=r+r_s \log \left|\frac{r}{r_s}-1\right|                 \\
        \frac{v-u}{2 r_s}=\frac{r}{r_s}+\log \left|\frac{r}{r_s}-1\right|     \\
        e^{\frac{v-u}{2 r_s}}=e^{\frac{r}{r_s}+\log \abs{ \frac{r}{r_s} 1} }, \\
        -e^{\frac{v}{2 r_s}}\left(-e^{-\frac{u}{2 r_s}}\right)=e^{\frac{r}{r_s}}\left|\frac{r}{r_s}-1\right|
    \end{aligned}
\end{equation}
Definimos las nuevas coordenadas:
\begin{equation}
    \mathcal{U} = -e^{-\frac{u}{4r_s}}, \quad \mathcal{V} = e^{\frac{v}{4r_s}}
\end{equation}

\subsubsection{Vectores base en Kruskal-Szekeres}
Relación diferencial usando la regla de la cadena:
\begin{equation}
    \begin{aligned}
        e_\mathcal{U} & = \frac{\partial u}{\partial \mathcal{U}} e_u = 4r_s e^{\frac{u}{4r_s}} e_u = -4r_s \mathcal{U}^{-1} e_u \\
        e_\mathcal{V} & = \frac{\partial v}{\partial \mathcal{V}} e_v = 4r_s e^{-\frac{v}{4r_s}} e_v = 4r_s \mathcal{V}^{-1} e_v
    \end{aligned}
\end{equation}

\subsubsection{Producto punto clave}
Calculamos el producto punto usando \( e_u \cdot e_v = -\frac{1}{2}\left(1-\frac{r_s}{r}\right) \):
\begin{equation}
    e_\mathcal{U} \cdot e_\mathcal{V} = (-4r_s \mathcal{U}^{-1})(4r_s \mathcal{V}^{-1})(e_u \cdot e_v) = \frac{8r_s^2}{\mathcal{U}\mathcal{V}}\left(1-\frac{r_s}{r}\right)
\end{equation}

\subsubsection{Relación geométrica fundamental}
De la definición de \( \mathcal{U} \) y \( \mathcal{V} \):
\begin{equation}
    \mathcal{U}\mathcal{V} = -e^{\frac{r^*}{2r_s}} = -\left|\frac{r}{r_s}-1\right|^{1/2}e^{\frac{r}{2r_s}}
\end{equation}

Sustituyendo en el producto punto:
\begin{equation}
    e_\mathcal{U} \cdot e_\mathcal{V} = \frac{32r_s^3}{r}e^{-\frac{r}{r_s}}
\end{equation}

\subsubsection{Métrica final de Kruskal-Szekeres}
La métrica toma su forma canónica:
\begin{equation}
    \boxed{ds^2 = \frac{32r_s^3}{r}e^{-\frac{r}{r_s}} d\mathcal{U}d\mathcal{V} + r^2(d\theta^2 + \sin^2\theta d\varphi^2)}
\end{equation}

\begin{equation}
    \text{con } \mathcal{U}\mathcal{V} = \left(1-\frac{r}{r_s}\right)e^{\frac{r}{r_s}} \quad \text{y} \quad \frac{\mathcal{V}}{\mathcal{U}} = -e^{\frac{ct}{2r_s}}
\end{equation}
De momento $\mathcal{V}$ y $\mathcal{U}$ son coordenadas nulas, que describen trayectorias de luz, pero podemos hacerlas espaciales y temporales definiendo:
\begin{equation}
    \begin{aligned}
        T \equiv \frac{\mathcal{V}+\mathcal{U}}{2} \\
        X \equiv \frac{\mathcal{V}-\mathcal{U}}{2} \\
    \end{aligned}
\end{equation}
y viceversa
\begin{equation}
    \begin{aligned}
        \mathcal{V}=T+X \\
        \mathcal{U}=T-X
    \end{aligned}
\end{equation}
De modo que la métrica en términos de $T$ y $X$ es:
\begin{equation}
        g_{\mu \nu} =\left[\begin{array}{cccc}
                                             +\frac{4 r_s^3}{r} e^{-\frac{r}{r_s}} & 0                                     & 0    & 0                   \\
                                             0                                     & -\frac{4 r_s^3}{r} e^{-\frac{r}{r_s}} & 0    & 0                   \\
                                             0                                     & 0                                     & -r^2 & 0                   \\
                                             0                                     & 0                                     & 0    & -r^2(\sin \theta)^2
                                         \end{array}\right]
\end{equation}
Usando el mismo truco que usamos anteriormente podemos encontar las geodésicas nulas en estas coordenadas.
\begin{equation}
    \begin{array}{l}
        0=\left(\frac{d T}{d \lambda}\right)^2\left(\frac{\partial}{\partial T} \cdot \frac{\partial}{\partial T}\right)+\left(\frac{d X}{d \lambda}\right)^2\left(\frac{\partial}{\partial X} \cdot \frac{\partial}{\partial X}\right) \\
        0=\left(\frac{d T}{d \lambda}\right)^2 g_{T T}+\left(\frac{d X}{d \lambda}\right)^2 g_{X X}                                                                                                                                     \\
        0=\left(\frac{d T}{d \lambda}\right)^2\left(+2 g_{U V}\right)+\left(\frac{d X}{d \lambda}\right)^2\left(-2 g_{U V}\right)
    \end{array}
\end{equation}

\begin{equation}
    \d{X}{\lambda} =\pm \d{T}{\lambda} \to \d{X}{T} = \pm 1 \to T = \pm Y + k
\end{equation}

Eso quiere decir que las geodésicas nulas en estas coordenadas son rectas en el espacio-tiempo. 
ahora volver a poner las coordenadas en términos de $r$ y $t$ solo es cuestión de una composición de funciones.

\begin{equation}
    \begin{aligned}
        T&=\frac{V+U}{2} \\
        X&=\frac{V-U}{2}
    \end{aligned}
    \quad ; 
    \begin{aligned}
        V&=e^{\frac{v}{2 r_s}} \\
        U&=-e^{-\frac{u}{2 r_s}}
    \end{aligned}
    \quad ; 
    \begin{aligned}
        v&=c t+r+r_s \log \left|\frac{r}{r_s}-1\right| \\
        u&=c t-r-r_s \log \left|\frac{r}{r_s}-1\right|
    \end{aligned}
\end{equation}
Simplificando
\begin{equation}
    \begin{aligned}
        T & =  \frac{1}{2}\left(V + U\right)                                                                                                                                                      \\
          & = \frac{1}{2} e^{\frac{v}{2 r_s}}-\frac{1}{2} e^{-\frac{u}{2 r_S}}                                                                                                                    \\
          & = \frac{1}{2} e^{\frac{1}{2 r_s}\left(c t+r+r_s \log \left|\frac{r}{r_s}-1\right|\right)}-\frac{1}{2} e^{-\frac{1}{2 r_s}\left(c t-r-r_s \log \left|\frac{r}{r_s}-1\right|\right)}    \\
          & =\frac{1}{2} e^{\frac{c t}{2 r_s}} e^{\frac{r}{2 r_s}}\left(\left|\frac{r}{r_s}-1\right|\right)^{1 / 2}-\frac{1}{2} e^{\frac{-c t}{2 r_s}} e^{\frac{r}{2 r_s}}\left(\left|\frac{r}{r_s}-1\right|\right)^{1 / 2} \\
          & = e^{\frac{r}{2 r_s}} \sqrt{\left|\frac{r}{r_s}-1\right|} \frac{1}{2}\left(e^{\frac{c t}{2 r_s}}-e^{\frac{-c t}{2 r_s}}\right)                                                                     \\
          & = e^{\frac{r}{2 r_s}} \sqrt{\left|\frac{r}{r_s}-1\right|} \sinh\left(\frac{ct}{2r_s}\right)
          \label{eq:Kruskal-SzekeresT(r,t)}
    \end{aligned}
\end{equation}
y para $X$ el calculo es prácticamente el mismo a excepción de un signo ($+$) en la diferencia de exponentes.
\begin{equation}
        X  = e^{\frac{r}{2 r_s}} \sqrt{\left|\frac{r}{r_s}-1\right|}  \cosh\left(\frac{ct}{2r_s}\right)
        \label{eq:Kruskal-SzekeresX(r,t)}
\end{equation}
Donde podemos desarrollar la siguiente identidad
\begin{equation}
    \begin{aligned}
         X^2 -T^2  &=  \left( e^{\frac{r}{2r_s}} \sqrt{\left|\frac{r}{r_s}-1\right|}\, \cosh\left(\frac{ct}{2r_s}\right) \right)^2- \left( e^{\frac{r}{2r_s}} \sqrt{\left|\frac{r}{r_s}-1\right|}\, \sinh\left(\frac{ct}{2r_s}\right) \right)^2 \\[1mm]
        &= e^{\frac{r}{r_s}} \left|\frac{r}{r_s}-1\right| \left[ \cosh^2\left(\frac{ct}{2r_s}\right) -\sinh^2\left(\frac{ct}{2r_s}\right) \right]\\[1mm]
        &= \, e^{\frac{r}{r_s}} \left|\frac{r}{r_s}-1\right|\\[2mm]
        &=
        \begin{cases}
            \, e^{\frac{r}{r_s}} \left(\dfrac{r}{r_s}-1\right), & \text{si } r > r_s,\\[2mm]
            \, e^{\frac{r}{r_s}} \left(1-\dfrac{r}{r_s}\right), & \text{si } r < r_s.
        \end{cases}
    \end{aligned}
    \label{eq:Kruskal-SzekeresT^2-X^2}
\end{equation}

\noindent
Esta identidad nos ayuda a determinar las líneas de \(r\) constante en el espacio-tiempo de Schwarzschild. Lo interesante ocurre cuando observamos el caso \(r < r_s\): el valor absoluto \(\left|\frac{r}{r_s}-1\right|\) en la ecuación (\ref{eq:Kruskal-SzekeresT^2-X^2}) se puede expresar como \(1-\frac{r}{r_s}\). ¿Por qué es importante notar esto? Si nos fijamos en la segunda línea de la ecuación, podemos identificar los coeficientes de la siguiente forma:
\begin{equation}
    \begin{aligned}
       X^2 -  T^2  &= e^{\frac{r}{r_s}} \left(1-\frac{r}{r_s}\right) \left[ \cosh^2\left(\frac{ct}{2r_s}\right) - \sinh^2\left(\frac{ct}{2r_s}\right) \right] \\
        &= -\, e^{\frac{r}{r_s}} \left(\frac{r}{r_s}-1\right) \left[ \cosh^2\left(\frac{ct}{2r_s}\right) - \sinh^2\left(\frac{ct}{2r_s}\right) \right] \\
        &= e^{\frac{r}{r_s}} \left(\frac{r}{r_s}-1\right) \left[ \sinh^2\left(\frac{ct}{2r_s}\right) - \cosh^2\left(\frac{ct}{2r_s}\right) \right].
    \end{aligned}
\end{equation}
De ello se deduce que, dentro del horizonte de eventos (\(r < r_s\)), se tienen las siguientes expresiones:
\begin{equation}
    X = e^{\frac{r}{2r_s}} \sqrt{\frac{r}{r_s}-1} \, \frac{1}{2} \sinh\left(\frac{ct}{2r_s}\right), \quad 
    T = e^{\frac{r}{2r_s}} \sqrt{\frac{r}{r_s}-1} \, \frac{1}{2} \cosh\left(\frac{ct}{2r_s}\right).
\end{equation}
Esto implica que, dentro del horizonte de eventos, las coordenadas \(T\) y \(X\) se invierten en comparación con el caso fuera del horizonte. Dado este intercambio de coordenadas cuando se pasa el horizonte de eventos, normalmente por conveniencia matemática la ecuación \ref{eq:Kruskal-SzekeresT^2-X^2} se expresa sin el valor absoluto, es decir:
\begin{equation}
    X^2 - T^2 = e^{\frac{r}{r_s}} \left(\frac{r}{r_s}-1\right) \left[ \cosh^2\left(\frac{ct}{2r_s}\right) - \sinh^2\left(\frac{ct}{2r_s}\right) \right],
\end{equation} 
para $r \epsilon [0,\infty ) $.

Dado que el negativo de estas coordenadas $-X$ y $-T$ cumplen la identidad (\ref{eq:Kruskal-SzekeresT^2-X^2}) podemos escribir lo que se llama la extensión máxima de las coordenadas de Kruskal-Szekeres
\begin{table}[H]
    \centering
    \caption{Extensión máxima de las coordenadas de Kruskal-Szekeres}
    \begin{tabular}{|c|c|c|}
        \hline Region & $T$                                                                                         & $X$                                                                                \\
        \hline$I$        & $+\sinh \left(\frac{c t}{2 r_s}\right)\frac{r}{2 r_s} \sqrt{\frac{r}{r_s}-1}$ & $+\cosh \left(\frac{c t}{2 r_s}\right) e^{\frac{r}{2 r_s}} \sqrt{\frac{r}{r_s}-1}$ \\
        \hline$I I$       & $+\cosh \left(\frac{c t}{2 r_s}\right) e^{\frac{r}{2 r_s}} \sqrt{1-\frac{r}{r_s}}$          & $+\sinh \left(\frac{c t}{2 r_s}\right) e^{\frac{r}{2 r_s}} \sqrt{1-\frac{r}{r_s}}$ \\
        \hline$I I I$   & $-\sinh \left(\frac{c t}{2 r_s}\right) e^{\frac{r}{2 r_s}} \sqrt{\frac{r}{r_s}-1}$          & $-\cosh \left(\frac{c t}{2 r_s}\right) e^{\frac{r}{2 r_s}} \sqrt{\frac{r}{r_s}-1}$ \\
        \hline$I V$      & $-\cosh \left(\frac{c t}{2 r_s}\right) e^{\frac{r}{2 r_s}} \sqrt{1-\frac{r}{r_s}}$          & $-\sinh \left(\frac{c t}{2 r_s}\right) e^{\frac{r}{2 r_s}} \sqrt{1-\frac{r}{r_s}}$ \\
        \hline
    \end{tabular}
\end{table}

\noindent Ahora para la lineas de tiempo constante es fácil ver la siguiente relación
\begin{equation}
    \tanh \left(\frac{ct}{2 r_s}\right)=\left\{\begin{array}{ll}
        T / X & \text { (en region I y III) } \\
        X / T & \text { (en region II y IV) }
        \end{array}\right.
\end{equation}
Con esto tenemos todo lo necesario para construir el diagrama de Kruskal-Szekeres.  
\begin{figure}[H]
    \begin{small}
        \begin{center}
            \includegraphics[width=0.95\textwidth]{AgujerosNegros/Schwarzschild/media/images/Kruskal_Szekeres_diagram_ManimCE_v0.19.0.png}
        \end{center}
        \caption{}
        \label{fig:}
    \end{small}
\end{figure}
