\appendix
\chapter{Programa par ael calculo de símbolos de  Christoffel}

%%\lstinputlisting[language=Python]{christofel.py}
\begin{lstlisting}[language=Python, caption=Python example]
from sympy import symbols, Function, Matrix, Rational, diff, simplify, latex, init_printing, sin, cos

# Definición de Símbolos
# Se mantienen los símbolos originales que definiste
t, r, theta, phi, Gamma_sym = symbols('t r theta phi Gamma') # Renombré 'Gamma' a 'Gamma_sym' para evitar confusión con el nombre de la función Christoffel, aunque en tu código original no causaba error.
# CORRECCIÓN PRINCIPAL: 'indices' ahora es una lista de los 4 símbolos de coordenadas
indices = [t, r, theta, phi] 
A = Function('A')(r) # Para la métrica 3D comentada, se mantiene

# Parámetros y expresiones para la métrica de Kerr
# Se mantienen tus definiciones originales para 'a, M, c, delta, sigma' como símbolos.
# 'delta' y 'sigma' (minúsculas) se redefinen inmediatamente como expresiones.
a, M, c, delta_s, sigma_s = symbols('a M c Delta Sigma', real=True) # Delta y Sigma (mayúsculas) son símbolos, delta y sigma (minúsculas) se definen abajo

# Expresiones para la métrica (usando _expr para distinguirlas de los símbolos si fuera necesario)
delta = r**2 - 2*M*r + a**2
sigma = r**2 + a**2*cos(theta)**2

# Función para convertir índices a su forma de LaTeX
def latex_index(index_symbol): # El argumento es un símbolo (ej: t, r, theta, phi)
    # CORRECCIÓN: Añadido 't' al diccionario
    index_dict = {'t': 't', 'r': 'r', 'theta': '\\theta', 'phi': '\\phi'}
    return index_dict.get(str(index_symbol), str(index_symbol))

# christoffelSymbols = [] # Esta lista no se usaba, la comento o puedes eliminarla.
init_printing(use_unicode=True)

## Métrica de Kerr (g_ij)
# Usando las variables 'sigma' y 'delta' que son las expresiones definidas arriba
g_ij = [
    [-(1 - 2*M*r/sigma)*c**2, 0, 0, -2*M*a*r*sin(theta)**2*c/sigma],
    [0, sigma/delta, 0, 0],
    [0, 0, sigma, 0],
    [-2*M*a*r*sin(theta)**2*c/sigma, 0, 0, (r**2 + a**2 + 2*M*a*r*sin(theta)**2/sigma)*sin(theta)**2]
]

# Métrica inversa g^ij (tu variable 'gij')
# El cálculo de la inversa es correcto. len(g_ij) será 4.
gij = [[Matrix(g_ij).inv()[i-1, j-1] for j in range(1, len(g_ij)+1)] for i in range(1, len(g_ij)+1)]

# Función para calcular Símbolos de Christoffel
# upperIndex, lowerIndex1, lowerIndex2 son enteros 1-basados (1, 2, 3, 4)
def Christoffel(upperIndex, lowerIndex1, lowerIndex2):
    # Obtener los símbolos de coordenadas correspondientes a los índices
    # Ahora 'indices' es la lista [t, r, theta, phi]
    upperIndexSymbol = indices[upperIndex - 1]
    lowerIndexSymbol1 = indices[lowerIndex1 - 1]
    lowerIndexSymbol2 = indices[lowerIndex2 - 1]

    Christoffel_sum = 0 # Usé Christoffel_sum para la variable de suma, pero tu 'Christoffel = 0' original también funciona debido al alcance.
    
    # La suma ahora itera correctamente de 0 a 3 (len(indices) es 4)
    for i in range(0, len(indices), 1): # 'i' es el índice de suma (0-basado)
        
    
        term_inv_metric = gij[upperIndex-1][i]
        
        diff1 = diff(g_ij[i][lowerIndex2-1], lowerIndexSymbol1)
        diff2 = diff(g_ij[i][lowerIndex1-1], lowerIndexSymbol2)
        diff3 = diff(g_ij[lowerIndex1-1][lowerIndex2-1], indices[i])
        
        Christoffel_sum = Christoffel_sum + Rational(1,2) * term_inv_metric * (diff1 + diff2 - diff3)
    
    Christoffel_simplified = simplify(Christoffel_sum)

    latex_str = f"\\Gamma^{{{latex_index(upperIndexSymbol)}}}{{ }}_{{{latex_index(lowerIndexSymbol1)} {latex_index(lowerIndexSymbol2)}}} &= {latex(Christoffel_simplified)} \\\\"
    
    print(latex_str)
    return latex_str # o Christoffel_simplified si prefieres devolver el valor SymPy

# Calcular todos los símbolos de Christoffel
# Los bucles de 1 a 4 (range(1,5,1)) son correctos para una métrica 4x4
print("Calculando todos los símbolos de Christoffel para la métrica de Kerr:")
print("(Esto puede tomar un tiempo considerable...)")
for i in range(1, 5, 1):      # Índice superior k
    for j in range(1, 5, 1):  # Primer índice inferior m
        for k_loop in range(1, 5, 1): # Segundo índice inferior n (usé k_loop para evitar sobreescribir la k del bucle externo si la usaras)
            # Imprimir qué símbolo se está calculando puede ayudar a rastrear el progreso
            print(f"% Calculando Gamma^{latex_index(indices[i-1])}_({latex_index(indices[j-1])}{latex_index(indices[k_loop-1])})")
            Christoffel(i, j, k_loop)

\end{lstlisting}