\appendix
\chapter{Programa para el calculo de símbolos de  Christoffel}
\label{chap:programa_christoffel}

Este es el programa utilizado para calcular los símbolos de Christoffel de la métrica \ref{eq:genericmetric2} usados en la derivación de la métrica  de Schwarzschild.
%%\lstinputlisting[language=Python]{christofel.py}
\begin{lstlisting}[language=Python, caption=Programa para el calculo de símbolos de  Christoffel]
from sympy import symbols,trigsimp,factor,cancel, Function, Matrix, Rational, diff, simplify, latex, init_printing, sin, cos

# Definición de Símbolos
t, r, theta, phi, Gamma_sym = symbols('t r theta phi Gamma') 

indices = [t, r, theta, phi] 


# Parámetros y expresiones para la métrica de Kerr
a, m, c, delta_s, sigma_s, rho_s = symbols('a m c Delta Sigma rho', real=True) # Delta y Sigma (mayúsculas) son símbolos, delta y sigma (minúsculas) se definen abajo

# Expresiones para la métrica (usando _expr para distinguirlas de los símbolos si fuera necesario)
delta = r**2 - 2*m*r + a**2
sigma = (r**2 + a**2)**2 - a**2*sin(theta)**2*delta 
rho = r**2 + a**2*cos(theta)**2


# Función para convertir índices a su forma de LaTeX
def latex_index(index_symbol): 
    index_dict = {'t': 't', 'r': 'r', 'theta': '\\theta', 'phi': '\\phi'}
    return index_dict.get(str(index_symbol), str(index_symbol))


init_printing(use_unicode=True)

A = Function('A', real=True)(r)
B = Function('B', real=True)(r)


# Métrica genérica para la derivación de la métrica de Schwarzschild
g_ij = [
    [-A, 0, 0, 0],
    [0, B, 0, 0],
    [0, 0, r**2, 0],
    [0, 0, 0, r**2*sin(theta)**2]
]



# Métrica inversa g^ij (tu variable 'gij')
gij = [[Matrix(g_ij).inv()[i, j] for j in range(0, len(g_ij))] for i in range(0, len(g_ij))]

# Función para calcular Símbolos de Christoffel
# upperIndex, lowerIndex1, lowerIndex2 son enteros 1-basados (1, 2, 3, 4)
def Christoffel(upperIndex, lowerIndex1, lowerIndex2, latex_output=True):
    
    # Obtener los símbolos de coordenadas correspondientes a los índices
    # Ahora 'indices' es la lista [t, r, theta, phi]
    upperIndexSymbol = indices[upperIndex ]
    lowerIndexSymbol1 = indices[lowerIndex1 ]
    lowerIndexSymbol2 = indices[lowerIndex2 ]

    Christoffel_sum = 0 # Usé Christoffel_sum para la variable de suma, pero tu 'Christoffel = 0' original también funciona debido al alcance.
    
    # La suma ahora itera correctamente de 0 a 3 (len(indices) es 4)
    for i in range(0, len(indices), 1): # 'i' es el índice de suma (0-basado)
        
       
        term_inv_metric = gij[upperIndex][i]
        
        diff1 = diff(g_ij[i][lowerIndex2], lowerIndexSymbol1)
        diff2 = diff(g_ij[i][lowerIndex1], lowerIndexSymbol2)
        diff3 = diff(g_ij[lowerIndex1][lowerIndex2], indices[i])
        
        Christoffel_sum += Rational(1,2) * term_inv_metric * (diff1 + diff2 - diff3)


    Christoffel_simplified = simplify(Christoffel_sum)  # Simplificación general
    Christoffel_simplified = trigsimp(Christoffel_simplified)  # Simplifica funciones trigonométricas
    Christoffel_simplified = cancel(Christoffel_simplified)  #

    # Imprimir en pantalla los coeficientes no cero
    if (Christoffel_simplified != 0):
        if (latex_output == True):
            latex_str = f"\\Gamma^{{{latex_index(upperIndexSymbol)}}}{{ }}_{{{latex_index(lowerIndexSymbol1)} {latex_index(lowerIndexSymbol2)}}} &= {latex(Christoffel_simplified)} \\\\"
            print(latex_str)
        else:
            latex_str = f"\\Gamma^{{{upperIndex}}}{{ }}_{{{lowerIndex1} {lowerIndex2}}} &= {latex(Christoffel_simplified)} \\\\"
            print(latex_str)
    
    return Christoffel_simplified


def Christoffel_all(latex_output=False):
    """
    Crea un objeto con indices tipo tensor (que no es un tensor) que contiene todos los Christoffel

    """
    Christoffel_list = []
    for i in range(0, 4, 1):      # Índice superior k
        i_list = []
        for j in range(0, 4, 1):  # Primer índice inferior m
            j_list = []
            for k_loop in range(0, 4, 1): # Segundo índice inferior n 
                Christoffel_sym = Christoffel(i, j, k_loop, latex_output=latex_output)
                j_list.append(Christoffel_sym)
            i_list.append(j_list)
        Christoffel_list.append(i_list)

    return Christoffel_list

Christoffel_all()
\end{lstlisting}

\chapter{Programa para el calculo del tensor de Ricci}
\label{chap:programa_ricci}
Este programa calcula la los componentes del tensor de Ricci por la definición de \ref{eq:ricci_tensor}

\begin{lstlisting}[language=Python, caption=Programa para el calculo del tensor de Ricci]
from christoffel import Christoffel_all # importar el programa anterior
from sympy import *

t, r, theta, phi, = symbols('t r theta phi') 

indices = [t, r, theta, phi]

# Lista con los símbolos de Christoffel
Christoffel_symbols = Christoffel_all()


# Por definición el tensor de Ricci es la contracción de 2 indices del tensor de Riemann

# mu , nu siendo los indices del Ricci
for mu in range(0,4,1):
    for nu in range(0,4,1):
        ricci_mu_nu = 0
        ##sobre el indice mudo alpha
        for alpha in range(0,4,1):
            ricci_mu_nu += diff(Christoffel_symbols[alpha][mu][nu],indices[alpha]) -diff(Christoffel_symbols[alpha][mu][alpha],indices[nu]) 
            # indice mudo lambda
            for lam in range(0,4,1):
                ricci_mu_nu +=  Christoffel_symbols[alpha][alpha][lam]*Christoffel_symbols[lam][mu][nu] - Christoffel_symbols[alpha][nu][lam]*Christoffel_symbols[lam][mu][alpha]
        ricci_mu_nu = cancel(ricci_mu_nu)
        ricci_mu_nu = simplify(ricci_mu_nu)
 
        # imprimir en pantalla los componentes no cero
        if ricci_mu_nu != 0:
            print(f"R_{{{mu}{nu}}} &= {latex(ricci_mu_nu)} \\\\")
\end{lstlisting}

\chapter{Formalismo de Hamilton en relatividad general}
\label{chap:hamilton}
El formalismo de Hamilton es una reformulación de la mecánica clásica que se utiliza para describir sistemas dinámicos. En relatividad general, este formalismo se adapta para manejar la dinámica del espacio-tiempo y los campos gravitacionales. 
Clásicamente la acción se define como la integral del lagrangiano $L$ a lo largo del tiempo
\[
S = \int L dt
\]  
y el lagrangiano es una función de las coordenadas generalizadas $q_i$ y sus velocidades $\dot{q}_i$ (derivadas respecto al tiempo)
\[
L = L(q_i, \dot{q}_i, t)
\]

tomemos la acción de una partícula libre de masa $M$ en un espacio-tiempo curvo respecto a su tiempo propio $\tau$
\begin{equation}
    S = \int -Mc \mathbf{d}\tau  \qquad [\text{Energía} \times \text{Tiempo}]. 
\end{equation}
Por definición del tiempo propio $d\tau^2 = -ds^2/c^2$
\begin{equation}
    S = \int -Mc \sqrt{-g_{\mu \nu} \frac{dx^\mu}{d\lambda} \frac{dx^\nu}{d\lambda}} d\lambda
    \label{eq:accionParticulaRaiz}
\end{equation} 
bajo la parametrización arbitraria $\lambda$ (que puede ser el tiempo coordenado $t$, el tiempo propio $\tau$ o cualquier otro parámetro afín).
Por razones de sencillez y para tener congruencia con los textos que están ahi afuera como \cite[p. 45]{wald-1984} que presentan un lagrangiano sin la raíz cuadrada, se tomara el parámetro afín $\lambda = \tau$ tal que la normalización
\begin{equation}
    g_{\mu \nu} \frac{dx^\mu}{d\tau} \frac{dx^\nu}{d\tau} = -c^2 \rightarrow \sqrt{-g_{\mu \nu} \frac{dx^\mu}{d\tau} \frac{dx^\nu}{d\tau}} = c.
\end{equation}
Además de eso por mera conveniencia matemática se multiplicara por el factor un medio, de esta forma la acción queda
\begin{equation}
    S  =  \int \frac{M}{2}  g_{\mu \nu} \frac{dx^\mu}{d\tau} \frac{dx^\nu}{d\tau} d\tau.
\end{equation}
Esta acción es equivalente a la de (\ref{eq:accionParticulaRaiz}) ya que ambas producen las mismas ecuaciones de movimiento (geodésicas) y difieren en una constante multiplicativa(que no afecta las ecuaciones de movimiento).
El lagrangiano es entonces
\begin{equation}
    L = \frac{1}{2} M g_{\mu \nu} \dot{x}^\mu \dot{x}^\nu 
\end{equation}
donde $\dot{x}^\mu = dx^\mu / d\tau$
\begin{note}
     En esta parte intente hacer el enfoque mas natural para proponer el lagrangiano de la acción, ahi afuera hay varias formas de llegar a este lagrangiano, a veces solo se propone el lagrangiano sin la raíz cuadrada sin justificar mucho de donde ha salido.

    Fragmento tomado de \cite[ver p. 45]{wald-1984}:
            \begin{quote}
                 An identical derivation shows that the curves which extremize proper time between two points are precisely the timelike geodesics. These derivations also show that the geodesic equation (with affine parameterization) can be obtained from variation of the Lagrangian,
                 $$
                 L=\sum_{\mu, \nu} g_{\mu \nu} \frac{d x^\mu}{d t} \frac{d x^\nu}{d t}
                 $$
    \end{quote}
\end{note}

Ya que este lagrangiano describe la dinámica de una particula libre en un espacio-tiempo curvo, este mismo debe de ser compatible con la ecuación de geodésicas \ref{eq:geodesic}, es decir, las ecuaciones de movimiento que se obtienen al aplicar el principio de mínima acción i.e. las ecuaciones de Euler-Lagrange a este lagrangiano deben ser las ecuaciones de geodésicas.
\begin{equation}
    \frac{d}{d\tau} \left( \frac{\partial L}{\partial \dot{x}^\beta} \right) - \frac{\partial L}{\partial x^\beta} = 0
\end{equation}

Para el lagrangiano propuesto, las ecuaciones de Euler-Lagrange se convierten en
\begin{equation}
 \begin{aligned}
    \frac{d}{d\tau} \left( \frac{\partial }{\partial \dot{x}^\beta} \left(\frac{1}{2} M g_{\mu \nu} \dot{x}^\mu \dot{x}^\nu \right) \right) - \frac{\partial }{\partial x^\beta}\left(\frac{1}{2} M g_{\mu \nu} \dot{x}^\mu \dot{x}^\nu \right) = 0\\ 
    \frac{d}{d\tau} \left( \frac{1}{2} M g_{\mu \nu} \frac{\partial }{\partial \dot{x}^\beta} \left(  \dot{x}^\mu \dot{x}^\nu \right) \right) 
    - \frac{1}{2} M \left( \partial_\beta g_{\mu \nu} \right) \dot{x}^\mu \dot{x}^\nu = 0\\
    \frac{d}{d\tau} \left( \frac{1}{2}  g_{\mu \nu}  \left(  \dot{x}^\mu \delta_\beta^\nu +\delta_\beta^\mu \dot{x}^\nu \right) \right) 
    - \frac{1}{2}  \left( \partial_\beta g_{\mu \nu} \right) \dot{x}^\mu \dot{x}^\nu = 0\\
    \frac{d}{d\tau} \left( \frac{1}{2}   \left(  g_{\mu \beta} \dot{x}^\mu  + g_{\beta \nu}  \dot{x}^\nu \right) \right) 
    - \frac{1}{2}  \left( \partial_\beta g_{\mu \nu} \right) \dot{x}^\mu \dot{x}^\nu = 0\\
    \frac{d}{d\tau} \left(    g_{\mu \beta} \dot{x}^\mu \right) 
    - \frac{1}{2}  \left( \partial_\beta g_{\mu \nu} \right) \dot{x}^\mu \dot{x}^\nu = 0\\
    \left(\partial_\sigma g_{\mu \beta }\right) \dot{x}^\sigma \dot{x}^\mu+    g_{\mu \beta} \frac{d \dot{x}^\mu}{d\tau}
    - \frac{1}{2}  \left( \partial_\beta g_{\mu \nu} \right) \dot{x}^\mu \dot{x}^\nu = 0\\
    \frac{1}{2}\left(\partial_\sigma g_{\mu \beta } + \partial_\mu g_{\sigma \beta } \right) \dot{x}^\sigma \dot{x}^\mu+    g_{\mu \beta} \frac{d \dot{x}^\mu}{d\tau}
    - \frac{1}{2}  \left( \partial_\beta g_{\mu \nu} \right) \dot{x}^\mu \dot{x}^\nu = 0\\
    g_{\mu \beta}  \ddot{x}^\mu+
    \frac{1}{2}\left(\partial_\nu g_{\mu \beta } + \partial_\mu g_{\nu \beta } -  \partial_\beta g_{\mu \nu} \right) \dot{x}^\mu \dot{x}^\nu = 0\\
    g^{\alpha \beta}g_{\mu \beta}  \ddot{x}^\mu+
    \underbrace{g^{\alpha \beta}\frac{1}{2}\left(\partial_\nu g_{\mu \beta } + \partial_\mu g_{\nu \beta } -  \partial_\beta g_{\mu \nu} \right) \dot{x}^\mu \dot{x}^\nu = 0}_{\Gamma^\alpha_{\mu \nu}}\\
    \ddot{x}^\alpha + \Gamma^\alpha_{\mu \nu} \dot{x}^\mu \dot{x}^\nu = 0
 \end{aligned}   
\end{equation}

que son precisamente las ecuaciones de geodésicas (\ref{eq:geodesic}).

\section{Ecuación de Hamilton-Jacobi para geodésicas}
Una vez que hemos comprobado que el lagrangiano propuesto describe la dinámica de una partícula libre en un espacio-tiempo curvo, podemos proceder a derivar el hamiltoniano por la transformación de Legendre, el momento conjugado se define como
\begin{equation}
p_\alpha=\frac{\partial L}{\partial \dot{x}^\alpha}=\frac{\partial }{\partial \dot{x}^\alpha}\left( \frac{1}{2} M g_{\mu \nu} \dot{x}^\mu \dot{x}^\nu\right)= M g_{\alpha \nu} \dot{x}^\nu
\end{equation}
multiplicando ambos lados por $g^{\alpha \beta}$ se obtiene
\begin{equation}
g^{\alpha \beta} p_\alpha = g^{\alpha \beta} M g_{\alpha \nu} \dot{x}^\nu = M \delta^\beta_\nu \dot{x}^\nu = M \dot{x}^\beta \rightarrow  \dot{x}^\beta= \frac{p^{\beta}}{M}
\end{equation}
Y se escribe el hamiltoniano como

\begin{equation}
\begin{aligned}
    H(x, p)&=p_\mu \dot{x}^\mu-L=  \frac{p_\mu p^{\mu}}{M} -\frac{1}{2} M g_{\mu \nu} \frac{p^{\mu}}{M} \frac{p^{\nu}}{M} \\
    &= \frac{1}{2M}  p_\mu p^{\mu}\\
    &= \frac{1}{2M} g^{\mu \nu} p_\mu p_\nu 
\end{aligned}
\end{equation}
 
La ecuación de Hamilton-Jacobi para la función principal $S\left(x^\mu, \lambda\right)$ es

\begin{equation}
\frac{\partial S}{\partial \lambda}+H(x, \partial S / \partial x)=0
\end{equation}

donde $\lambda$ es un parámetro afín a lo largo de la trayectoria.Con las transformaciones canónicas usuales
\begin{equation}
    \frac{\partial S}{\partial x^\mu} = p_\mu \quad ; \quad \frac{\partial S}{\partial \lambda} = -H,
\end{equation}
la ecuación de Hamilton-Jacobi en este caso se convierte en
\begin{equation}
\frac{\partial S}{\partial \lambda}+\frac{1}{2M} g^{\mu \nu} \frac{\partial S}{\partial x^\mu}  \frac{\partial S}{\partial x^\nu} =0.
\end{equation}
Aquí es útil calcular ahora la cantidad contravariante $g^{\mu \nu} p_\mu p_\nu$ 
\begin{equation}
g^{\mu \nu} p_\mu p_\nu=M^2 g^{\mu \nu} g_{\mu \alpha} g_{\nu \beta} \dot{x}^\alpha \dot{x}^\beta=M^2 g_{\alpha \beta} \dot{x}^\alpha \dot{x}^\beta
\end{equation}
Recuérdese que geodésicas timelike normalizadas con tiempo propio $g_{\alpha \beta} \dot{x}^\alpha \dot{x}^\beta=-c^2 \rightarrow g^{\mu \nu} p_\mu p_\nu=-M^2 c^2$, y por tanto
\begin{equation}
\frac{\partial S}{\partial \lambda}=\frac{1}{2} M c^2,
\end{equation}
podemos buscar una solución separada del tipo
\begin{equation}
S(x, \lambda)=W(x)+\frac{1}{2} Mc^2 \lambda
\end{equation}
(donde la parte lineal en $\lambda$ refleja que el hamiltoniano es constante a lo largo de la evolución afín). 
Sustituyendo:
\begin{equation}
    \frac{1}{2} Mc^2+\frac{1}{2 M} g^{\mu \nu} \partial_\mu S \partial_\nu S=0 \quad \Longrightarrow \quad g^{\mu \nu} \partial_\mu S \partial_\nu S=-M^2 c^2
    \label{eq:geodesicHamiltonJacobi}
\end{equation}
esta es la ecuación de Hamilton-Jacobi para geodésicas.



\section{Codigo para metrica inversa}
\label{sec:codigo_metrica_inversa}
El siguiente código calcula la métrica inversa de Kerr a partir de la métrica covariante.
\begin{lstlisting}[language=Python, caption=Programa para el calculo de la métrica inversa de Kerr]
    from sympy import symbols, Matrix, sin, cos, simplify, pretty_print, factor, cancel, together, trigsimp, latex

# ------------------------------
# Definición de símbolos
# ------------------------------
t, r, theta, phi = symbols('t r theta phi', real=True)
a, m = symbols('a m', real=True)
# Estas son las variables objetivo para el resultado final
delta_s, rho_s, sigma_s = symbols('delta rho sigma', real=True)

def smart_subs(expr, old, new):
    """Sustituye en expr todas las ocurrencias algebraicamente equivalentes a old por new."""
    return expr.replace(
        lambda subexpr: simplify(subexpr - old) == 0,
        lambda subexpr: new
    )

# ------------------------------
# Definición de las expresiones completas
# ------------------------------
delta_expr = r**2 - 2*m*r + a**2
rho_expr   = r**2 + a**2*cos(theta)**2
sigma_expr = (r**2 + a**2)**2 - a**2*sin(theta)**2*(r**2 - 2*m*r + a**2)

# ------------------------------
# Métrica de Kerr (g_ij)
# ------------------------------
g_ij = Matrix([
    [-(1 - 2*m*r/rho_expr),          0, 0, -2*m*a*r*sin(theta)**2/rho_expr],
    [0,                             rho_expr/delta_expr, 0, 0],
    [0,                             0, rho_expr, 0],
    [-2*m*a*r*sin(theta)**2/rho_expr, 0, 0, sigma_expr*sin(theta)**2/rho_expr]
])

# ------------------------------
# Métrica inversa de Kerr (g^ij)
# ------------------------------
gij_inv = g_ij.inv()

# Sustituciones para variables cortas
subs_dict = {
    sigma_expr: sigma_s,
    delta_expr: delta_s,
    rho_expr: rho_s
}

# Simplificación robusta: varias rutinas en cascada
def full_simplify(expr):
    expr = expr.subs(subs_dict)
    expr = together(expr)
    expr = cancel(expr)
    expr = simplify(expr)
    expr = factor(expr)
    expr = trigsimp(expr)
    return expr

gij_inv = gij_inv.applyfunc(full_simplify)
gij_inv = gij_inv.subs(subs_dict)
gij_inv = gij_inv.applyfunc(lambda e: smart_subs(e, sigma_expr, sigma_s))
gij_inv = gij_inv.applyfunc(lambda e: smart_subs(e, delta_expr, delta_s))
gij_inv = gij_inv.applyfunc(lambda e: smart_subs(e, rho_expr, rho_s))


# ------------------------------
# Resultados
# ------------------------------
pretty_print(gij_inv)
print("\n")
print(latex(gij_inv))
\end{lstlisting}



\section{Codigo para la determinacioon de las constantes de movimiento en Kerr}
\label{chap:programa_constantes_kerr}
\begin{lstlisting}[language=Python, caption={Programa para el calculo de las constantes de movimiento en Kerr}]
import numpy as np
from sympy import symbols, Matrix, sin, cos, solveset

# ------------------------------
# 1. Definición de Símbolos
# ------------------------------
r, theta = symbols('r, theta', real=True)
a, M = symbols('a, M', real=True, positive=True)

# ------------------------------
# 2. Definición de Expresiones Auxiliares
# ------------------------------
delta = r**2 - 2 * M * r + a**2
rho2 = r**2 + a**2 * cos(theta)**2
sigma = (r**2 + a**2)**2 - a**2 * sin(theta)**2 * delta

# ------------------------------
# 3. Métrica de Kerr (Covariante g_uv)
# ------------------------------
g_uv = Matrix([
    [-(1 - 2 * M * r / rho2), 0, 0, -2 * M * a * r * sin(theta)**2 / rho2],
    [0, rho2 / delta, 0, 0],
    [0, 0, rho2, 0],
    [-2 * M * a * r * sin(theta)**2 / rho2, 0, 0, sigma * sin(theta)**2 / rho2]
])

# ------------------------------
# 4. Función para Calcular las Constantes de Movimiento Totales
# ------------------------------

def get_total_constants_of_motion(r_0: float, theta_0: float, 
                                  u_r0: float, u_theta0: float, u_phi0: float, 
                                  a_val: float = 0.9, M_val: float = 1.0,
                                  mu: float = 1.0, c: float = 1.0):
    """
    Calcula las constantes de movimiento TOTALES.
    Versión robusta que maneja correctamente el caso ecuatorial (theta = pi/2).
    """
    g_uv_val = g_uv.subs([(r, r_0), (theta, theta_0), (a, a_val), (M, M_val)])

    u_t0_sym = symbols('u_t', real=True, positive=True)
    u_vector = [u_t0_sym, u_r0, u_theta0, u_phi0]
    
    # Ecuación de normalización: g_uv * U^u * U^v = -c^2
    norm_eq = sum(g_uv_val[i, j] * u_vector[i] * u_vector[j] for i in range(4) for j in range(4)) + c**2
    
    solutions = solveset(norm_eq, u_t0_sym)
    u_t0_val = None
    for sol in solutions:
        if sol.is_real and sol > 0:
            u_t0_val = float(sol)
            break
            
    if u_t0_val is None:
        raise ValueError("No se encontró una solución real y positiva. Revisa las condiciones iniciales.")

    # Calcular E y Lz (escalados por la masa mu)
    E = mu * -(g_uv_val[0, 0] * u_t0_val + g_uv_val[0, 3] * u_phi0)
    Lz = mu * (g_uv_val[3, 0] * u_t0_val + g_uv_val[3, 3] * u_phi0)

    # --- Calcular la Constante de Carter Total (Q) - FORMA ROBUSTA ---
    rho2_val = float(rho2.subs([(r, r_0), (theta, theta_0), (a, a_val)]))
    
    # Comprobar si estamos en el ecuador para usar la fórmula simplificada
    # Se usa una pequeña tolerancia para evitar errores de punto flotante
    if abs(theta_0 - np.pi/2) < 1e-9:
        # En el ecuador, cos(theta) es 0, por lo que Q se simplifica drásticamente
        Q = (mu * r_0**2 * u_theta0)**2 #  rho2 en el ecuador es r^2
    else:
        # Si no estamos en el ecuador, usamos la fórmula completa
        cos_theta0_val = cos(theta_0)
        sin_theta0_val = sin(theta_0)

        p_theta_sq = (mu * rho2_val * u_theta0)**2

        if abs(sin_theta0_val) < 1e-9:
            term_Lz = 0 if abs(Lz) < 1e-9 else float('inf')
        else:
            term_Lz = Lz**2 / sin_theta0_val**2

        term_E = a_val**2 * (mu**2 * c**2 - E**2 / c**2)
        Q = p_theta_sq + cos_theta0_val**2 * (term_Lz + term_E)

    return E, Lz, Q

# ------------------------------
# 5. Ejemplo de Uso 
# ------------------------------
print("--- Test partícula en caída ---")
try:
    E_fall, Lz_fall, Q_fall = get_total_constants_of_motion(
        r_0=6.0, theta_0=np.pi/2, 
        u_r0=0.0, u_theta0=0.0, u_phi0=0.0,
        a_val=0.9, M_val=1.0, mu=1.0, c=1.0
    )
    print(f"  - Energía Total: {E_fall:.6f}")
    print(f"  - Momento Angular Total: {Lz_fall:.6f}")
    print(f"  - Constante de Carter Total: {Q_fall:.6f}")

except ValueError as e:
    print(f"Error: {e}")

\end{lstlisting}

\section{Programa geodesicas}
\label{chap:programa_geodesicas}
Este programa calcula las geodésicas en la métrica de Kerr usando las constantes de movimiento 

\begin{lstlisting}[language=Python, caption=Programa para el calculo de las geodesicas en Kerr]
    import numpy as np
from scipy.integrate import solve_ivp
import matplotlib.pyplot as plt
from matplotlib.patches import Patch
from matplotlib.patches import Circle


# --- Variables Globales para el Manejo de Puntos de Inflexión ---
# Guardan el signo de la velocidad para saber si la partícula va hacia adentro/afuera
# o hacia arriba/abajo.
sign_r = 1.0  # Empezamos moviéndonos hacia afuera
sign_theta = 1.0 # Empezamos moviéndonos hacia el polo norte

# Guardan el valor de la derivada anterior para ayudar a detectar el cambio de signo.
dr_dtau_prev = 0.0
dtheta_dtau_prev = 0.0


# Constantes del agujero negro
a = 0.9
m = 1.0
c = 1.0

# Constantes de la geodésica
# Estas se calclulan por separado
# Constantes de movimiento para una órbita inclinada y en precesión
E = 0.95     # Energía 
Lz = 3.0     # Momento angular axial
Q = 15.0     # Constante de Carter (alta para una gran inclinación)
mu = 1.0     # Masa de la partícula (1 para masiva, 0 para fotón)

# Condiciones iniciales [r0, theta0, phi0, t0]
r0 = 6.0*m
theta0 = np.pi / 3  # Inclinación inicial de 60 grados respecto al ecuador
phi0 = 0.0
t0 = 0.0
y0 = [r0, theta0, phi0, t0]


tau_max = 30
tau_span = [0, tau_max]

def kerr_geodesics(tau, y, m, a, E, Lz, Q):
    """
    Define el sistema de EDOs para las geodésicas de Kerr en tiempo Mino (tau).
    El vector de estado es y = [r, theta, phi, t].
    """
    global sign_r, sign_theta, dr_dtau_prev, dtheta_dtau_prev
    r, theta, phi, t = y

    # Términos auxiliares de la métrica
    Delta = r**2 - 2 * m * r + a**2


    Rr = ((r**2 + a**2) * (E/c) - a * Lz)**2 - Delta * ((mu * c * r)**2 + (Lz - a * (E/c))**2 + Q)

    # El término np.sin(theta)**2 + 1e-9 evita la división por cero en los polos.
    Th = Q - np.cos(theta)**2 * (Lz**2 / (np.sin(theta)**2 + 1e-9)+a**2 * ((mu* c)**2 - (E/c)**2))

    # Se truncan a cero valores negativos pequeños que puedan surgir de errores de precisión.
    Rr = max(Rr, 0.0)
    Th = max(Th, 0.0)

    # Detección y manejo simple de puntos de inflexión (turning points)
    # Si el potencial es casi cero y la velocidad anterior era mayor, invertimos la dirección.
    if Rr < 1e-7 and dr_dtau_prev**2 > Rr:
        sign_r *= -1
    if Th < 1e-7 and dtheta_dtau_prev**2 > Th:
        sign_theta *= -1

    # Derivadas con respecto al tiempo Mino (tau)
    dr_dtau = sign_r * np.sqrt(Rr)
    dtheta_dtau = sign_theta * np.sqrt(Th)

    # Actualizamos los valores de las derivadas para la siguiente iteración
    dr_dtau_prev = dr_dtau
    dtheta_dtau_prev = dtheta_dtau

    dphi_dtau = (Lz / (np.sin(theta)**2 + 1e-9) - a * (E/c)) + (a *((r**2 + a**2) * (E/c) - a * Lz) / Delta)

    dt_dtau = - a * (a * (E/c) * np.sin(theta)**2 - Lz) + ((r**2 + a**2) * ((r**2 + a**2) * (E/c) - a * Lz) / Delta)

    return [dr_dtau, dtheta_dtau, dphi_dtau, dt_dtau]

print("Iniciando la integración numérica...")

# Genera 2000 puntos espaciados uniformemente para una curva suave.
t_eval_points = np.linspace(tau_span[0], tau_span[1], 2000) 

# Llamada al solver de EDOs de SciPy
sol = solve_ivp(
    fun=kerr_geodesics,
    t_span=tau_span,
    y0=y0,
    args=(m, a, E, Lz, Q),
    method='Radau',      # Un método robusto para este tipo de problemas
    dense_output=True,   # Permite obtener una solución continua y suave
    rtol=1e-8,           # Tolerancia relativa para alta precisión
    atol=1e-8,           # Tolerancia absoluta
    t_eval=t_eval_points # Puntos donde se evalúa la solución
)


print("Integración completada con éxito.")


# Extraer los resultados de la solución
r, theta, phi, t = sol.y

# --- 3. Visualización de la Órbita ---

print("Generando la visualización 3D...")

# Convertir de coordenadas de Boyer-Lindquist a cartesianas para la gráfica
x = r * np.sin(theta) * np.cos(phi)
y = r * np.sin(theta) * np.sin(phi)
z = r * np.cos(theta)

# Crear la figura y el eje 3D
fig = plt.figure(figsize=(12, 10))
ax = fig.add_subplot(111, projection='3d')

# Malla de parámetros
u = np.linspace(0, 2 * np.pi, 200)
v = np.linspace(0, np.pi, 200)

# Ergoesfera
r_e = m + np.sqrt(m**2 - (a*np.cos(v))**2)
x_e = r_e * np.outer(np.cos(u), np.sin(v))
y_e = r_e * np.outer(np.sin(u), np.sin(v))
z_e = r_e * np.outer(np.ones(np.size(u)), np.cos(v))
ax.plot_surface(x_e, y_e, z_e, color='purple', alpha=0.25, rstride=5, cstride=5)

# Horizonte de sucesos
r_h = m + np.sqrt(m**2 - a**2)
x_bh = r_h * np.outer(np.cos(u), np.sin(v))
y_bh = r_h * np.outer(np.sin(u), np.sin(v))
z_bh = r_h * np.outer(np.ones(np.size(u)), np.cos(v))
ax.plot_surface(x_bh, y_bh, z_bh, color='black', alpha=0.6, rstride=5, cstride=5)

# Círculo del horizonte en el plano ecuatorial
theta = np.linspace(0, 2*np.pi, 400)
x_circ = a * np.cos(theta)
y_circ = a * np.sin(theta)
z_circ = np.zeros_like(theta)
ax.plot(x_circ, y_circ, z_circ, color="red", lw=2, label="Singularidad r=a")

# --- Trayectoria (se dibuja al final, queda arriba de todo) ---
ax.plot(x, y, z, label='Órbita Precesante', lw=1, color='cyan')

# Configuración de la gráfica
ax.set_title("Orbita en el espacio-tiempo de Kerr", fontsize=16)
ax.set_xlabel("X")
ax.set_ylabel("Y")
ax.set_zlabel("Z")
ax.legend()

legend_elements = [
    Patch(facecolor='purple', edgecolor='k', alpha=0.25, label='Ergoesfera'),
    Patch(facecolor='black', edgecolor='k', alpha=0.6, label='Horizonte de sucesos'),
]

ax.legend(handles=legend_elements + ax.get_legend_handles_labels()[0], loc="upper right")

# Ajustar límites
max_range = np.max(np.abs([x, y, z])) * 1.2
ax.set_xlim([-max_range, max_range])
ax.set_ylim([-max_range, max_range])
ax.set_zlim([-max_range, max_range])

# Ángulo de vista
ax.view_init(elev=30, azim=45)

# Fondo blanco
ax.set_facecolor("white")

plt.tight_layout()

plt.show()
# --- 4. Visualización 2D en el plano XY ---

fig, ax = plt.subplots(figsize=(8, 8))

# Trayectoria proyectada en XY
ax.plot(x, y, color="cyan", lw=1, label="Órbita Precesante")

# Horizonte de sucesos (círculo en XY)
circle_h = Circle((0, 0), r_h, color="black", alpha=0.6, label="Horizonte de sucesos")
ax.add_patch(circle_h)

# Ergoesfera (círculo en XY)
r_e_eq = m + np.sqrt(m**2 - 0)  # ergo radio en el ecuador
circle_e = Circle((0, 0), r_e_eq, color="purple", alpha=0.25, label="Ergoesfera")
ax.add_patch(circle_e)

#Singularidad
circle_s = Circle((0, 0), a, color="red", alpha=0.9, label="Singularidad", fill=False)
ax.add_patch(circle_s)

# Configuración
ax.set_aspect("equal")
ax.set_xlabel("X")
ax.set_ylabel("Y")

ax.legend()


plt.show()
\end{lstlisting}

\section{programa para funciones de r circulares}
\label{chap:programa_orbitas_circulares_kerr}

\begin{lstlisting}[language=Python, caption={Cálculo numérico de las constantes para órbitas circulares en Kerr}, label={lst:kerr_circular_orbits}]
import numpy as np
import matplotlib.pyplot as plt
from sympy import symbols, diff, lambdify
from scipy.optimize import fsolve

# ------------------------------
# 1. PARÁMETROS DE CONFIGURACIÓN
# ------------------------------
# Constantes físicas y del agujero negro
PARAMS = {
    'm': 1.0,      # Masa del agujero negro
    'a': 0.9,      # Parámetro de espín (spin)
    'mu': 1.0,     # Masa de la partícula de prueba
    'c': 1.0       # Velocidad de la luz (unidades geometrizadas)
}

# Parámetros de simulación
R_START = PARAMS['m'] + np.sqrt(PARAMS['m']**2 - PARAMS['a']**2) + 0.1  # Justo fuera del horizonte
R_END = 6.0
N_POINTS = 5000
Q_VALS = [0, 5, 10] # Valores del parámetro de Carter a simular

# ------------------------------
# 2. DEFINICIÓN SIMBÓLICA (con SymPy)
# ------------------------------
r, a, m, E, Lz, Q, mu, c = symbols('r a m E L_z Q mu c', real=True)

# Ecuaciones de la geodésica de Kerr
delta = r**2 - 2*m*r + a**2
P_func = (r**2 + a**2)*E - a*Lz
R_func = P_func**2 - delta*(mu**2*r**2 + (Lz - a*E)**2 + Q)
R_prime = diff(R_func, r)

# ------------------------------
# 3. CONVERSIÓN A FUNCIONES NUMÉRICAS (con Lambdify)
# ------------------------------
# Se convierten las expresiones de SymPy en funciones de NumPy para un cálculo rápido.
VARIABLES = [E, Lz, r, Q, a, m, mu, c]
R_numeric = lambdify(VARIABLES, R_func, 'numpy')
R_prime_numeric = lambdify(VARIABLES, R_prime, 'numpy')

# ------------------------------
# 4. FUNCIÓN SOLUCIONADORA (con SciPy)
# ------------------------------
def solve_equations(initial_guess, r_val, Q_val, params):
    """
    Resuelve el sistema R=0 y R'=0 para E y Lz usando fsolve de SciPy.
    
    Args:
        initial_guess (list): Valores iniciales para [E, Lz].
        r_val (float): Valor del radio.
        Q_val (float): Valor del parámetro de Carter.
        params (dict): Diccionario con los parámetros del agujero negro.
        
    Returns:
        tuple: (E, Lz) o (nan, nan) si la solución no se encuentra.
    """
    # Función objetivo para fsolve. Debe devolver [0, 0] en la solución.
    def objective_func(variables):
        E_sol, Lz_sol = variables
        args = (r_val, Q_val, params['a'], params['m'], params['mu'], params['c'])
        
        eq1 = R_numeric(E_sol, Lz_sol, *args)
        eq2 = R_prime_numeric(E_sol, Lz_sol, *args)
        
        return [eq1, eq2]

    try:
        # fsolve es el equivalente numérico de nsolve
        solution, _, success_flag, _ = fsolve(objective_func, initial_guess, full_output=True)
        if success_flag == 1:
            return float(solution[0]), float(solution[1])
        else:
            return np.nan, np.nan
    except RuntimeError:
        # fsolve puede lanzar un error si no converge
        return np.nan, np.nan


#r_consulta = 3
#q_consulta = 4
#E_consulta, Lz_consulta = solve_equations([0.9, 2.0], r_consulta, q_consulta, PARAMS)
#print(f"Para r={r_consulta} y Q={q_consulta}, se obtiene E={E_consulta:.6f}, Lz={Lz_consulta:.6f}")

# ------------------------------
# 5. BUCLE PRINCIPAL DE CÁLCULO
# ------------------------------
print("Iniciando cálculos...")
r_vals = np.linspace(R_START, R_END, N_POINTS)
results = {}

for q in Q_VALS:
    print(f"Calculando para Q = {q}...")
    E_vals, Lz_vals = [], []
    # Valores iniciales para el primer punto del radio
    current_guess = [0.9, 2.0] 
    
    for rv in r_vals:
        E_sol, Lz_sol = solve_equations(current_guess, rv, q, PARAMS)
        E_vals.append(E_sol)
        Lz_vals.append(Lz_sol)
        
        # Optimización: Usa la solución actual como el valor inicial para el siguiente punto.
        # Esto hace que el solucionador sea mucho más rápido y estable.
        if not np.isnan(E_sol):
            current_guess = [E_sol, Lz_sol]
            
    results[q] = {'E': E_vals, 'Lz': Lz_vals}

print("Cálculo finalizado.")

# ------------------------------
# 6. VISUALIZACIÓN DE RESULTADOS
# ------------------------------
# Usamos el modo orientado a objetos de Matplotlib para mayor control



# Líneas verticales de referencia
r_horizonte = PARAMS['m'] + np.sqrt(PARAMS['m']**2 - PARAMS['a']**2)
r_ergoesfera = 2 * PARAMS['m']

plt.figure()
plt.axvline(x=r_horizonte, color='red', linestyle='--', label='Horizonte de eventos ($r_+$)')
plt.axvline(x=r_ergoesfera, color='black', linestyle=':', label='Ergoesfera estática ($r_E$)')
plt.grid(True, linestyle='--', alpha=0.6)

# Gráfica para E(r)
for q, data in results.items():
    plt.plot(r_vals, data['E'], label=f'Q={q}')
plt.ylabel("E")
plt.xlabel("r")
plt.legend()
plt.ylim(0, 6)

plt.show()


plt.figure()
plt.axvline(x=r_horizonte, color='red', linestyle='--', label='Horizonte de eventos ($r_+$)')
plt.axvline(x=r_ergoesfera, color='black', linestyle=':', label='Ergoesfera estática ($r_E$)')
plt.grid(True, linestyle='--', alpha=0.6)   
# Gráfica para Lz(r)
for q, data in results.items():
    plt.plot(r_vals, data['Lz'], label=f'Q={q}')
plt.ylabel("$L_z$")
plt.xlabel("r")
plt.legend()
plt.ylim(0, 6)

plt.show()
\end{lstlisting}