\usepackage[most]{tcolorbox}
\usepackage{amsthm} % Theorem Formatting
\usepackage{amssymb}    % Math symbols such as \mathbb


\tcbuselibrary{theorems}
%configs pf colorbox
\tcbset {
  base/.style={
    arc=0mm, 
    bottomtitle=0.5mm,
    boxrule=0mm,
    colbacktitle=black!10!white, 
    coltitle=black, 
    fonttitle=\bfseries, 
    left=2.5mm,
    leftrule=1mm,
    right=3.5mm,
    title={#1},
    toptitle=0.75mm, 
  }
}
\definecolor{brandblue}{rgb}{0.34, 0.7, 1}
\renewcommand\qedsymbol{$\blacksquare$}
%%%%%%%%%%%%%%%%%%%%%%%%%%%%%%%%%%%%%%%%%%%%%%%%%%%%%%%%
%theorems 

%\newtcbtheorem[number within=section]{example}{Example}%
%{ colframe=brandblue, 
%base={#1}}{th}

\newtcbtheorem[number within=section]{theorem}{Teorema}%
{ colframe=black!30!white,
base={#1}}{th}
\newtcbtheorem[number within=section]{task}{Tarea}%
{ colframe=red!, 
base={#1}}{tsk}

\newtheorem{problem}{Problema}[]
\usepackage{cancel}
\newtheorem{proposition}{Proposición}
\newtheorem{axiom}{Axioma}[section]
\newtheorem{lemma}{Lema}
\theoremstyle{definition}
\newtheorem{definition}{Definición}
\newenvironment{sol}{%\small%
        \begin{trivlist} \item \textbf{Solución}. }{%
            \hspace*{\fill}\end{trivlist}}%








